\documentclass[11pt, a4paper]{letter}
\input{structure.tex}
\usepackage{url}
\longindentation=0pt

\Who{Jordan C. Hanson}

\authordetails{
	Assistant Professor\\
	Department of Physics and Astronomy\\
	13406 East Philadeplphia Street\\
	Whittier CA 90602\\
	Email: jhanson2@whittier.edu\\​
}

\logo{WhittierCollegeSeal.png}
\headerlineone{Whittier}
\headerlinetwo{College}

\begin{document}

\begin{letter}{
	Cover Letter: Application for Tenure
}

\opening{Greetings,}

My name is Prof. Jordan C. Hanson, and I am currently an Assistant Professor in the Department of Physics and Astronomy at Whittier College.  We are contacting you for your consideration as an external reviewer for our tenure and promotion process at Whittier College.  This dossier contains my curriculum vitae, tenure and promotion guidelines, and my recent scholarship.  Whittier College is a 130-year-old liberal arts college in Whittier, California, near East Los Angeles.  We are recognized as a Title-V Hispanic Serving Institution (HSI) with amission to provide access to higher education for historically marginalized students.  We currently serve about 2,000 undergraduates.

I have focused my scholarship on three broad areas: advancing the field of ultra-high energy neutrino (UHE-$\nu$) research as part of IceCube Gen2 (\url{https://wipac.wisc.edu}), RF engineering research through the Office of Naval Research (ONR: \url{https://www.nre.navy.mil}), and contributions to diversity, equity, and inclusion (DEI).  It is important to note that, at Whittier College, we typically teach six courses per year. I make my research contributions while teaching physics, math, and computer science courses.  I have found creative ways to deliver a unique and intellectually enriching environment for my students, often by connecting my research and teaching. I regularly involve undergraduate students in my research, and I have published scholarship with undergraduate authors.

Within this dossier, you will find several types of examples of my scholarship.  These are primarily peer-reviewed journal articles in the UHE-$\nu$ area, but also include \textit{applied} scholarship produced with undergraduates.  I provide descriptions of how the applied science or engineering concept connects to our mission at Whittier College, the ONR, and IceCube Gen2.  The UHE-$\nu$ research is categorized into five areas: computational electromagnetism (CEM), the Askaryan effect, RF antenna design and fabrication, Antarctic ice properties, and RF drones.  The ONR research is categorized into four areas: CEM, RF antenna design, workforce development, and reliability analysis.

Finally, I have included three examples of DEI contributions.  First, I highlight my participation in the Artemis program, a STEM recruitment and research program for young women from local high schools.  The second describes an internal DEI grant I have earned to develop a mobile application that will help foster inclusion and belonging in introductory STEM courses.  The third is an example of my participation in the Whittier Scholars Program (WSP: \url{https://scholars.domains/}).  This program allows students to customize a major within our curriculum, and the interdisciplinary results tend to diversify scholarship within fields by forming new connections.

I am grateful for your consideration, and we at Whittier College want to thank you for taking the time to complete this evaluation.

\closing{Sincerely,}
\end{letter}

\end{document}
