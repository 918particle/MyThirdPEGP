\documentclass[11pt, a4paper]{letter}
%%%%%%%%%%%%%%%%%%%%%%%%%%%%%%%%%%%%%%%%%
% Professional Formal Letter
% Structure Specification File
% Version 1.0 (12/2/17)
%
% This file originates from:
% http://www.LaTeXTemplates.com
%
% Authors:
% Brian Moses
% Vel (vel@LaTeXTemplates.com)
%
% License:
% CC BY-NC-SA 3.0 (http://creativecommons.org/licenses/by-nc-sa/3.0/)
%
%%%%%%%%%%%%%%%%%%%%%%%%%%%%%%%%%%%%%%%%%

%----------------------------------------------------------------------------------------
%	PACKAGES AND OTHER DOCUMENT CONFIGURATIONS
%----------------------------------------------------------------------------------------

\usepackage{graphicx} % Required for including pictures

\usepackage[T1]{fontenc} % Output font encoding for international characters
\usepackage[utf8]{inputenc} % Required for inputting international characters
%\usepackage{gfsdidot} % Use the GFS Didot font: http://www.tug.dk/FontCatalogue/gfsdidot/
\usepackage{microtype} % Improves typography

\pagestyle{empty} % Suppress headers and footers

\setlength\parindent{1cm} % Paragraph indentation

% Create a new command for the horizontal rule in the document which allows thickness specification
\makeatletter
\newcommand{\vhrulefill}[1]{\leavevmode\leaders\hrule\@height#1\hfill \kern\z@}
\makeatother

%----------------------------------------------------------------------------------------
%	DOCUMENT MARGINS
%----------------------------------------------------------------------------------------

\usepackage{geometry} % Required for adjusting page dimensions

\geometry{
	top=1cm, % Top margin
	bottom=1.5cm, % Bottom margin
	left=3cm, % Left margin
	right=3cm, % Right margin
	%showframe, % Uncomment to show how the type block is set on the page
}

%----------------------------------------------------------------------------------------
%	DEFINE CUSTOM COMMANDS
%----------------------------------------------------------------------------------------

\newcommand{\logo}[1]{\renewcommand{\logo}{#1}}

\newcommand{\Who}[1]{\renewcommand{\Who}{#1}}
\newcommand{\Title}[1]{\renewcommand{\Title}{#1}}

\newcommand{\headerlineone}[1]{\renewcommand{\headerlineone}{#1}}
\newcommand{\headerlinetwo}[1]{\renewcommand{\headerlinetwo}{#1}}

\newcommand{\authordetails}[1]{\renewcommand{\authordetails}{#1}}

%----------------------------------------------------------------------------------------
%	AUTHOR DETAILS STRUCTURE
%----------------------------------------------------------------------------------------

\newcommand{\authordetailsblock}{
	\hspace{\fill} % Move the author details to the far right
	\parbox[t]{0.48\textwidth}{ % Box holding the author details; width value specifies where it starts and ends, increase to move details left
		\footnotesize % Use a smaller font size for the details
		\Who\\ % Author name
		\textit{\authordetails} % The author details text, all italicised
	}
}

%----------------------------------------------------------------------------------------
%	HEADER STRUCTURE
%----------------------------------------------------------------------------------------

\address{
	\includegraphics[width=1in]{\logo} % Include the logo of author institution
	\hspace{0.82\textwidth} % Position of the institution logo, increase to move left, decrease to move right
	\vskip -0.1\textheight~\\ % Position of the large header text in relation to the institution logo, increase to move down, decrease to move up
	\Large\hspace{0.2\textwidth}\headerlineone\hfill ~\\[0.006\textheight] % First line of institution name, adjust hspace if your logo is wide
	\hspace{0.2\textwidth}\headerlinetwo\hfill \normalsize % Second line of institution name, adjust hspace if your logo is wide
	\makebox[0ex][r]{\textbf{\Who\Title}}\hspace{0.01\textwidth} % Print author name and title with a little whitespace to the right
	~\\[-0.01\textheight] % Reduce the whitespace above the horizontal rule
	\hspace{0.2\textwidth}\vhrulefill{1pt} \\ % Horizontal rule, adjust hspace if your logo is wide and \vhrulefill for the thickness of the rule
	\authordetailsblock % Include the letter author's details on the right side of the page under the horizontal rule
	\hspace{-0.25\textwidth} % Horizontal position of the author details block, increase to move left, decrease to move right
	\vspace{-0.1\textheight} % Move the date and letter content up for a more compact look
}

%----------------------------------------------------------------------------------------
%	COMPOSE THE ENTIRE HEADER
%----------------------------------------------------------------------------------------

\renewcommand{\opening}[1]{
	{\centering\fromaddress\vspace{0.05\textheight} \\ % Print the header and from address here, add whitespace to move date down
	\hspace*{\longindentation}\today\hspace*{\fill}\par} % Print today's date, remove \today to not display it
	{\raggedright \toname \\ \toaddress \par} % Print the to name and address
	\vspace{1cm} % White space after the to address
	\noindent #1 % Print the opening line
}

%----------------------------------------------------------------------------------------
%	SIGNATURE STRUCTURE
%----------------------------------------------------------------------------------------

\signature{\Who\Title} % The signature is a combination of the author's name and title

\renewcommand{\closing}[1]{
	\vspace{2.5mm} % Some whitespace after the letter content and before the signature
	\noindent % Stop paragraph indentation
	\hspace*{\longindentation} % Move the signature right to the value of \longindentation
	\parbox{\indentedwidth}{
		\raggedright
		#1 % Print the signature text
		\vskip 1.65cm % Whitespace between the closing text and author's name for a physical signature
		\fromsig % Prints the value of \signature{}, i.e. author name and title
	}
}

\usepackage{url}
\longindentation=0pt

\Who{Jordan C. Hanson}

\authordetails{
	Assistant Professor\\
	Department of Physics and Astronomy\\
	13406 East Philadeplphia Street\\
	Whittier CA 90602\\
	Email: jhanson2@whittier.edu\\​
}

\logo{WhittierCollegeSeal.png}
\headerlineone{Whittier}
\headerlinetwo{College}

\begin{document}

\begin{letter}{
	Cover Letter: Application for Tenure
}

\opening{Greetings,}

My name is Prof. Jordan C. Hanson, and I am currently an Assistant Professor in the Department of Physics and Astronomy at Whittier College.  We are contacting you for your consideration as an external reviewer for our tenure and promotion process at Whittier College.  This dossier contains my curriculum vitae, tenure and promotion guidelines, and my recent scholarship.  Whittier College is a 130-year-old liberal arts college in Whittier, California, near East Los Angeles.  We are recognized as a Title-V Hispanic Serving Institution (HSI) with a mission to provide access to higher education for historically marginalized students.

I have focused my scholarship on three broad areas: advancing the field of ultra-high energy neutrino (UHE-$\nu$) research as part of IceCube Gen2 (\url{https://wipac.wisc.edu}), RF engineering research through the Office of Naval Research (ONR: \url{https://www.nre.navy.mil}), and contributions to diversity, equity, and inclusion (DEI).  It is important to note that, at Whittier College, we typically teach six courses per year. I make my research contributions while teaching physics, math, and computer science courses.  I have found creative ways to deliver a unique and intellectually enriching environment for my students, often by connecting my research and teaching. I regularly involve undergraduate students in my research, and I have published scholarship with undergraduate authors.

Within this dossier, you will find several types of examples of my scholarship.  These are primarily peer-reviewed journal articles in the UHE-$\nu$ and RF engineering areas, but also include \textit{applied} scholarship produced with undergraduates.  I provide descriptions of how the applied science or engineering concept connects to our mission at Whittier College, the ONR, and IceCube Gen2.  The UHE-$\nu$ and RF engineering research for the ONR is categorized into five areas: computational electromagnetism (CEM), the Askaryan effect, RF antenna design and fabrication, Antarctic ice properties, and drones.  The ONR research also adds workforce development in the form of interactive engineering courses.

Finally, I have included three examples of DEI contributions.  First, I highlight my participation in the Artemis program, a STEM recruitment and research program for young women from local high schools.  The second describes an DEI grant I have earned to develop a mobile application that will help foster inclusion and belonging in introductory STEM courses.  The third is an example of my participation in the Whittier Scholars Program (WSP: \url{https://scholars.domains/}).  This program allows students to customize a major within our curriculum, and the interdisciplinary results tend to diversify scholarship within fields by forming new connections.

I am grateful for your consideration, and we at Whittier College want to thank you for taking the time to read and process this dossier.

\closing{Sincerely,}
\end{letter}

\end{document}
