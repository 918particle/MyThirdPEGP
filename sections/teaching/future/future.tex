\documentclass[../../../main.tex]{subfiles}

\begin{document}

\begin{table}[ht]
\scriptsize
\centering
\begin{tabular}{| c | c | c | c | c |}
\hline \hline
Semester & Course & Credits & Students & Curriculum feature \\ \hline
Fall 2017 & PHYS135A-01 & 4.0 & 24 & Intro \\ \hline
Fall 2017 & PHYS150-01 & 4.0 & 17 & COM1/Intro \\ \hline
Spring 2018 & PHYS135B-01 & 4.0 & 18 & Intro \\ \hline
Spring 2018 & PHYS180-02 & 5.0 & 19 & COM1/Intro \\ \hline
Spring 2018 & COSC330/PHYS306 & 3.0 & 6 & Advanced \\ \hline
Fall 2018 & PHYS135A-01 & 4.0 & 24 & Intro \\ \hline
Fall 2018 & PHYS135A-02 & 4.0 & 26 & Intro \\ \hline
Jan 2019 & COSC390 & 3.0 & 8 & Advanced \\ \hline
Spring 2019 & PHYS135B-01 & 4.0 & 25 & Intro \\ \hline
Spring 2019 & PHYS180-02 & 4.0 & 9 & Intro/COM1 \\ \hline
Fall 2019 & PHYS135A-01 & 4.0 & 24 & Intro \\ \hline
Fall 2019 & PHYS150-02/03 & 4.0 & 26 & COM1/Intro \\ \hline
Fall 2019 & INTD255 & 3.0 & 23 & CON2 \\ \hline
Spring 2020 & COSC330/PHYS306 & 3.0 & 13 & Advanced \\ \hline
Spring 2020 & PHYS135B-01 & 4.0 & 23 & Intro \\ \hline
Spring 2020 & PHYS180-02 & 4.0 & 24 & COM1/Intro \\ \hline
Summer 2020 (Session II) & MATH080 & 3.0 & 11 & Intro \\ \hline
Fall 2020 (Module 1) & INTD100-21 & 3.0 & 14 & Intro \\ \hline
Fall 2020 (Module 2) & PHYS330 & 3.0 & 11 & Advanced \\ \hline
Spring 2021 (Module 1) & INTD290 & 3.0 & 26 & CON2,CUL3 \\ \hline
Spring 2021 (Module 2) & PHYS135B-02 & 4.0 & 17 & Intro \\ \hline
Spring 2021 (Module 3) & PHYS135B-01 & 4.0 & 25 & Intro \\ \hline
Fall 2021 & COSC330/PHYS306 & 3.0 & 16 & Advanced \\ \hline
Fall 2021 & PHYS135A-01 & 4.0 & 24 & Intro \\ \hline
Fall 2021 & PHYS135A-02 & 4.0 & 25 & Intro \\ \hline
Jan 2022 & COSC360 & 3.0 & 16 & Advanced \\ \hline
Spring 2022 & PHYS135B-02 & 4.0 & 25 & Intro \\ \hline
Spring 2022 & PHYS330 & 3.0 & 12 & Advanced \\ \hline
Summer 2022 & MATH080 & 3.0 & 3 & Intro \\ \hline \hline
\textbf{Students/Credit:} 5.1 & \textbf{Credits/Course:} 3.6 & \textbf{Students/Course:} 18.4 & \textbf{Credits/year:} 21 & \textbf{Advanced/Total:} 24\% \\ \hline
\hline
\end{tabular}
\caption{\label{tab:classes} This table is a summary of courses taught in five years, plus Summer sessions.  Not included: PHYS396 (Physics Research for Credit), PHYS499 (Senior Seminar), and PHYS495 (Independent Studies).}
\end{table}

The data in Tab. \ref{tab:classes} are meant to share a portrait of my teaching over the last five years, not including Fall 2022.  Appropriate benchmarks in my department are to teach 20 credits per year, and I have hit 21 credits per year, without including physics and engineering research for credit (PHYS396).  Our average student-to-faculty ratio is 12, meaning my courses tend to have 6 or more students above average (18.6), or about a 55\% increase.  All of my courses are either 3 or 4 credits, excluding things like PHYS396 (variable 1-3).  The average is 3.6 credits per course, meaning a slight majority of my credits come from 4-credit courses.  I have achieved a students/credit ratio of 5.1, again indicating that I teach more students than most of our colleagues at Whittier.  For example, a standard 3.0 credit course with 12 students would have a students/credit ratio of 4.0.  A ratio of 5.1 indicates that I teach $\approx 20$\% more students/credit.  I have taught a mixture of introductory STEM courses, liberal arts courses, and advanced technical subjects.  About three quarters of my courses have not been advanced technical subjects.  In summary, this represents a healthy mixture of teaching for an assistant professor at Whittier College.

\subsection{Summary Reflection on Teaching}

Before moving on to the other facets 

%1. Teaching in 2017-2018 introductory physics was a challenge
%	a) AAPT helped
%	b) Colleagues helped
%	c) Changing the way I did testing helped, developing a good relationship with the students
%	d) Summarize three changes made
%2. Now teaching introductory physics courses was a breeze
%	a) Course evaluations reflect this
%	b) Pandemic did not lower my scores, and in some cases, raised them
%3. Applying the lessons to my advanced courses has led to great success overall
%4. I have created new advanced courses and liberal arts courses that have generated great success
%5. Teaching INTD100 is challenging for a physical scientist, but I have managed to do it well
%	a) Scores from 1st INTD100 were good, but it was during the module system and fully remote
%	b) I am currently teaching INTD100 and it is going well, but time will tell
%	
%I have not forgotten your suggestion to bring back my CON2 course about the history of science in Latin America.  I can make this happen after I return from sabbatical, should I be granted tenure.  I welcome any other suggestions you might have.

\end{document}