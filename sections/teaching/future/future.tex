\documentclass[../../../main.tex]{subfiles}

\begin{document}

\begin{table}[ht]
\scriptsize
\centering
\begin{tabular}{| c | c | c | c | c | c |}
\hline \hline
Semester & Course & Credits & Students & Curriculum feature & Q16, Q25 Scores \\ \hline \hline
Fall 2017 & PHYS135A-01 & 4.0 & 24 & Intro & $3.19$, $3.24$ \\ \hline
Fall 2017 & PHYS150-01 & 4.0 & 17 & COM1/Intro & $3.56$, $3.13$ \\ \hline
Spring 2018 & PHYS135B-01 & 4.0 & 18 & Intro & $2.94$, $3.12$ \\ \hline
Spring 2018 & PHYS180-02 & 5.0 & 19 & COM1/Intro & $3.83$, $3.61$ \\ \hline
Spring 2018 & COSC330/PHYS306 & 3.0 & 6 & Advanced & $3.29$, $2.88$ \\ \hline
Fall 2018 & PHYS135A-01 & 4.0 & 24 & Intro & $3.92$, $4.13$ \\ \hline
Fall 2018 & PHYS135A-02 & 4.0 & 26 & Intro & $3.88$, $3.96$ \\ \hline
Jan 2019 & COSC390 & 3.0 & 8 & Advanced & $4.50$, $4.75$ \\ \hline
Spring 2019 & PHYS135B-01 & 4.0 & 25 & Intro & $4.33$, $4.46$ \\ \hline
Spring 2019 & PHYS180-02 & 4.0 & 9 & Intro/COM1 & $5.00$, $5.00$ \\ \hline
Fall 2019 & PHYS135A-01 & 4.0 & 24 & Intro & $4.50$, $4.70$ \\ \hline
Fall 2019 & PHYS150-02/03 & 4.0 & 26 & COM1/Intro & $4.80$, $4.80$ \\ \hline
Fall 2019 & INTD255 & 3.0 & 23 & CON2 & $4.60$, $4.80$ \\ \hline
Spring 2020 & COSC330/PHYS306 & 3.0 & 13 & Advanced & $4.80$, $4.60$ \\ \hline
Spring 2020 & PHYS135B-01 & 4.0 & 23 & Intro & $4.70$, $4.50$ \\ \hline
Spring 2020 & PHYS180-02 & 4.0 & 24 & COM1/Intro & $4.90$, $5.00$ \\ \hline
Summer 2020 & MATH080 & 3.0 & 11 & Intro & $4.80$, $4.80$ \\ \hline
Fall 2020 (Mod. 1) & INTD100-21 & 3.0 & 14 & Intro/Writing & $4.80$, $5.00$ \\ \hline
Fall 2020 (Mod. 2) & PHYS330 & 3.0 & 11 & Advanced & $4.40$, $4.90$ \\ \hline
Spring 2021 (Mod. 1) & INTD290 & 3.0 & 26 & CON2,CUL3 & $4.60$, $4.80$ \\ \hline
Spring 2021 (Mod. 3) & PHYS135B-01 & 4.0 & 25 & Intro & $3.90$, $4.50$ \\ \hline
Spring 2021 (Mod. 2) & PHYS135B-02 & 4.0 & 17 & Intro & $4.90$, $4.90$ \\ \hline
Fall 2021 & COSC330/PHYS306 & 3.0 & 16 & Advanced & $3.70$, $4.30$ \\ \hline
Fall 2021 & PHYS135A-01 & 4.0 & 24 & Intro & $4.90$, $4.90$ \\ \hline
Fall 2021 & PHYS135A-02 & 4.0 & 25 & Intro & $4.60$, $5.00$ \\ \hline
Jan 2022 & COSC360 & 3.0 & 16 & Advanced & $4.60$, $5.00$ \\ \hline
Spring 2022 & PHYS135B-02 & 4.0 & 25 & Intro & $4.80$, $4.90$ \\ \hline
Spring 2022 & PHYS330 & 3.0 & 12 & Advanced & $4.60$, $5.00$ \\ \hline
Summer 2022 & MATH080 & 3.0 & 3 & Intro & $--$ \\ \hline \hline
\textbf{Students/Cr.:} 5.1 & \textbf{Cr./Course:} 3.6 & \textbf{Students/Course:} 18.4 & \textbf{Cr./year:} 21 & \textbf{Adv./Total:} 24\% & \\ \hline
\hline
Fall 2019 & PHYS396 & 1.0-3.0 & 3 & Research/Cr. & $5.00$, $5.00$ \\ \hline
Spring 2020 & PHYS396 & 1.0-3.0 & 1 & Research/Cr. & $5.00$, $5.00$ \\ \hline
Spring 2022 & PHYS396 & 1.0-3.0 & 3 & Research/Cr. & $5.00$, $5.00$ \\ \hline
\end{tabular}
\caption{\label{tab:classes} A summary of all courses taught at Whittier College.  Physics Research, PHYS396, is included below the summary statistics.  The definitions of Q16 and Q25 from the course evaluation are: Q16 ``Overall, I would recommend this course to others,'' and Q25 ``Overall, I would recommend this professor to others.''  The Q16 and Q25 data were reported with standard deviations in the original paper system, but just the mean and median in the electronic system.  The means are reported here for consistancy.  For MATH080 in Summer 2022, my three students did not provide data.}
\end{table}

The data in Tab. \ref{tab:classes} are meant to share a portrait of my teaching over the last five years, not including Fall 2022.  Appropriate benchmarks in my department are to teach 20 credits per year, and I have achieved 21 credits per year, without including Physics Research, PHYS396.  Our average student-to-faculty ratio is 12, meaning my courses tend to have $\approx 6$ students above average (18.6 students per course), or about a 55\% increase.  All of my courses are either 3 or 4 credits, excluding PHYS396 (variable 1-3).  The average is 3.6 credits per course, meaning a slight majority of my credits come from 4-credit courses.  I have achieved a students per credit ratio of 5.1.  A 3.0-credit course with 12 students would have a students/credit ratio of 4.0.  A ratio of 5.1 indicates that I teach $\approx 20$\% more students/credit.  I have taught a mixture of introductory STEM courses, liberal arts courses, and advanced technical subjects.  About three quarters of my courses have not been advanced technical subjects (advanced ratio: 24\%).  The results in Tab. \ref{tab:classes} show a healthy mixture of courses for a professor at Whittier College.  In the next section, I share with you my summary reflection on trends in my teaching.

\subsection{Summary Reflection on Teaching}

Before turning to my scholarly work, it is important to reflect on my teaching as a whole during my time at Whittier College\footnote{The chair of FPC has informed me that I should include a 4-year history of course evalations.  See Supplemental Material.}.  This reflection highlights my long-term growth and devlopment as an educator.  This reflection is intended to provide context for those who have not encountered my previous teaching results and analysis.  There are three main themes.  The first theme is that learning to teach introductory physics courses in my first several semesters required major adjustments.  In my teaching and mentorship roles during my post-doctoral fellowships at The University of Kansas and The Ohio State University, I was exposed to high-performing physics students.  The transition to teaching students at Whittier College required three changes in my style that we made based on student evaluation data.  The positive effect these changes made was analyzed in fine numerical detail in my supplemental PEGP submitted in Fall 2019.  The transition is visible in Tab. \ref{tab:classes} after January 2019. Since then, my introductory physics scores have been solid.
\\
\vspace{0.15cm}
The second theme is that I have applied a balanced version of these lessons to advanced physics and computer science courses with great success.  I have earned high scores consistently in these courses, with few exceptions.  I have taught advanced physics courses, and advanced electrical engineering and computer science courses.  The most difficult course to maintain in this regard is Computer Logic and Digital Circuit Design (COSC330/PHYS306).  The first time I taught this course was during my transition to teaching introductory physics at Whittier College.  We struggled with the material, primarily because the textbook and lab workbooks I chose were too difficult and tailored for engineering majors at MIT.
\\
\vspace{0.15cm}
Since then, I have chosen a much better course text, \textit{Digital Fundamentals}, 11th ed. by T. Floyd \cite{digitalFund} (see Secs. 2.5 and 2.6 of my prior PEGP)\footnote{I am grateful to Prof. Zorba for suggesting this book.}.  I have incorporated an open-source digital circuit and code platform, the PYNQ-Z1 from Xilinx University Program (XUP).  The students loved this material the second and third time I taught COSC330/PHYS306.  There were problems, however, when the course grew to 16 students (see Sec. \ref{sec:adv_eval}), and these are easy to fix.  These effects are visible in Tab. \ref{tab:classes}, where the students gave average results for \textit{the course,} but higher scores for \textit{the professor.}  Both times I taught Digital Signal Processing (DSP, COSC390 and later COSC360), the students had wonderful experiences.  The students ask that the course be made into a semester-long course.  I propose that these courses form a two-semester sequence.  Those who take them in order would benefit from the synergy.  Digital logic, however, should not be a formal pre-requisite to DSP.  DSP attracts non-majors interested in learning more about image or audio processing, in addition to majors.  Other benefits would include reusing the PYNQ-Z1 system and \textit{Digital Fundamentals} in DSP.
\\
\vspace{0.15cm}
The third theme is that I have achieved a proven track record teaching outside my comfort zone.  This includes two sections of INTD100 (one in Fall 2022) and two CON2 and CUL3 courses.  The first CON2 course I taught, ``Safe Return Doubtful: History and Current Status of Modern Science in Antarctica,'' inspired my student Sophie Frizzell to become a leader in wildlife conservation.  This course also inspired my WSP student, Nicolas Bakken-French, to travel all over the world studying climate science (see Sec. \ref{sec:advising_mentoring}).  The course was very well received overall.  The second CON2/CUL3 course I taught was History of Science in Latin America.  As I covered in great detail in my prior PEGP, this subject is close to my heart given the Mexican-American heritage of much of my family.  One student, Scout Mucher, found the course so inspiring from the standpoint of exploration that Scout decided to venture to Antarctica this Fall!
\\
\vspace{0.15cm}
My INTD100 sections tend to focus on technical and professional writing mechanics, with the themes of popular science writing and the philosophy of science.  Much like my course on Antarctic science, I ask the students to perform a significant amount of reflection on the topic of leadership in journals.  This is also true in my CON2 and CUL3 courses.  In INTD255, I literally had the students keep a journal that led them through exploration and leadership reflections.  Understanding proper leadership is essential for both exploration and science.  The students seem to enjoy their time with me in these writing courses even though I have no training or experience as an English or literature professor.  Finally, I recall your suggestion to bring back the History of Science in Latin America course.  I assure you that I have not forgotten, and I plan to bring this course back after my potential sabbatical.
\\
\vspace{0.15cm}
Returning to the first theme, I share here the three major changes we made to my introductory physics courses.  More detail is provided in the numerical analysis I performed for my supplemental PEGP submitted in Fall 2019.  I also summarized them in my PEGP submitted in Fall 2021 (delayed one year by the pandemic).  The first adjustment was to \textit{control the pace.}  In 2017, I tended to deliver topics the way I experienced them as an undergraduate.  Anecdotally, it seems to me this pace and style is similar to our introductory mathematics courses.  The students shared that it was too fast, so I built in natural pace control mechanisms (like peer instruction \cite{mazur2013peer}).  The second adjustment was to \textit{include more example problems}.  This eventually led to the modification of tenet (1) of my teaching philosophy to include the structured warm up exercises (see Sec. \ref{sec:teaching_philosophy}).  This is an example of how Whittier students have positively influenced and enhanced my teaching philosophy.
\\
\vspace{0.15cm}
The third adjustment was to include more \textit{traditional lecture} content.  I learned several active learning techniques that comprise tenets (1)-(6) from an AAPT\footnote{Association of American Physics Teachers.} conference in Washington, DC in Fall 2017 \cite{AAPTPI}.  On of these was known as Just-in-Time Teaching, or JITT.  The students had never seen anything like it in high school, and soundly rejected it in favor of a more traditional style.  I followed their suggestion, and there was a positive response.  I have implemented these three adjustments diligently since January term 2019, and first reported them to FPC in Fall 2019 in my supplemental PEGP.  The effect on the data is visible in Tab. \ref{tab:classes}.  One can see an example of what happens when these adjustments are not implemented in Spring 2021.  I taught three courses in the three seven-week modules we offered that semester (see Sec. 2.4 of my prior PEGP).  During that semester, course evaluations were hidden until the end of the semester.  During the second module, I controlled my pace in PHS135B-02 (Module 2) such that the students were pleased and gave me high marks.
\\
\vspace{0.15cm}
We did not, however, cover all of the material traditionally covered in a PHYS135B.  This included dropping the topics of AC circuits and magnetic induction\footnote{The pandemic took my magnets!  Remember that?  From the curricular discussions?  Don't take my magnets!  Good times.}.  I felt it was a matter of integrity to deliver the whole course for my students. I tried to increase my pace during Module 3 (PHYS135B-01) to reach these topics by Week 7, and course evaluations were still hidden at this point.  Had they not been hidden, I would have known that increasing the pace was a mistake because the students were already saying the pace was fast in Module 2.  The faster pace just caused burnout for my students, and the score associated with the course (Q16) dropped to 3.9.  The score for the professor (Q25) was 4.5.  The students reported that the course felt rushed and lacked polish.  They acknowledged I was trying but that it was a difficult situation.  Delivering the whole course was just not possible in seven weeks.
\\
\vspace{0.15cm}
There were two other setbacks: grading and vectors.  Module 3 was a perfect storm of course work, committee service, finishing a research paper, being advisor to no less than 30 students, and caring for a one-year-old. In retrospect, I should have exchanged my written midterm for a test that was automatically graded using OpenStax Tutor\footnote{\url{https://tutor.openstax.org/?}}.  When the students submitted midterms, I realized I was not going to be able to grade them quickly.  The mathematical topic of vectors is usually introduced in 135A, and used extensively in 135B.  I was not the instructor for PHYS135A in Fall 2020.  The students from 135A who continued to 135B with me informed me with no uncertainty that they did not understand vectors.  Reviewing vectors on-the-fly cost me more time.  Despite these cases, we can see that the data has risen markedly since January 2019.  My supplemental PEGP submitted in Fall 2019 contained extensive analysis demonstrating the increases were already taking place.  My scores on Q16 and Q25, for example, were rising by a minimum of three standard deviations.  They are now usually between 4.5 and 5.0, which I take to mean that my teaching philosophy helps me guide the introductory students and to success.
\\
\vspace{0.15cm}
Returning to the second theme, I note that when I implement my teaching philosophy in advanced PHYS and COSC courses, I almost always receive high marks.  For many years, Prof. Lagan has taught Electromagnetic Theory (PHYS330).  I have recently taken over duties for this course as Seamus nears retirement.  We have also agreed that I will teach Computational Physics (PHYS325) eventually.  Electromagnetic theory (PHYS330) is a course with no less than seven pre-requisites: PHYS150, PHYS180, PHYS185, PHYS275, MATH141, MATH142, and MATH214.  This course is an application of vector calculus to the topic of charge, electric fields, voltage, magnetic fields, and electromagnetic waves.  Thus, this is a challenging course designed to deliver deep insights into the universe for physics and engineering majors.  I received high marks for this course when I taught it in the module system, and in-person in Spring 2022.
\\
\vspace{0.15cm}
As I shared in Sec. \ref{sec:teaching_philosophy}, I apply a remixed version of the tenets (1)-(6) of my philosophy to my advanced courses.  Because I have taught these students with these techniques before, the students respond positively. Tenet (1), with a structured warm up exercise, is an efficient way to guide students through new and difficult material. Electromagnetic theory (PHYS330) is a theory course, so tenets (3) and (4) apply much more rarely.  Tenets (1) and (2) I use much in the same way as my introductory courses, but (1) is emphasized more than (2).  Tenet (6) takes on even more importance than in the introductory courses.  The students must complete challenging theoretical and computational problems and explain them to their peers in a way that is useful.  This is a critical skill that must be emphasized whenever possible.  It is both a learning goal on my syllabi, and a department learning goal.
\\
\vspace{0.15cm}
My teaching in COSC330/PHYS306 and COSC360 hews more closely to my application of tenets (1)-(6) in my introductory courses.  A structured warm up (1) is given, followed by some traditional lecture.  Peer instruction (2) is less effective for these courses, because the material is not always cast in the form of word problems.  Though there are no PhET simulations for digital logic or DSP, I can write my own in GNU Octave or MATLAB.  Some exercises and concepts in these courses are best casted as design problems and laboratory activities (4).  My syllabi are structured such that we test in the laboratory (4) the very ideas that we encounter on the structured warm up (1) or simulation (3) within the same class session.  Peer instruction (2) is sometimes possible in DSP, but rarely so in computer logic and digital circuit design.  In the former, there are some math problems that make sense as PI modules.  In the latter, this is only true for the first few weeks of the course.  The students prefer to learn computer logic and digital circuits through tenet (4) rather than (2).  In both courses, tenet (6) takes on more importance than in introductory courses.  Students are designing systems and writing code that has to accomplish a task in a way that they can later explain to peers.  I place a high value on this skill given my experience in organizations like ONR and IceCube.
\\
\vspace{0.15cm}
Finally, returning to my third theme, I have had success teaching INTD100 and CON2/CUL3 courses.  My students have learned about exploration, the history of science, and qualities of good leadership.  I think it is particularly important, given the times, that my students learn about the philosophy of science.  How many times do we read in the news that someone made a claim ``without evidence?''  My students must be explicitly trained to recognize the difference between a scientific or evidence-based claim and one that is pseudo-scientific.  Perhaps the philosophy of science can be interwoven through my next version of History of Science in Latin America.  I focused on how people of diverse cultures all performed science throughout the 1600s-1800s.  In a semester-long setting (as opposed to the module system) we would be able to finish with 20th century and modern scientific endeavors by people of diverse cultures in Latin America.  I tried to include some material about this during the module system by introducing the students to the large, modern physics and astronomy projects taking place in Latin America: The Pierre Auger Observatory, ALMA, etc.  I also included discoveries in physics and astronomy we take for granted, like the aurora borealis, that were first understood correctly by Mexican physicists rather than European ones (as some tend to assume).  However, when discussing scientific history as far back as the 1500s, a solid foundation in the philosophy of science will be useful.  This is because modern medical and scientific methodology bears little resemblance to both indigenous and European science several hundred years ago.
\\
\vspace{0.15cm}
I have reflected here on three main stories regarding my teaching over the years at Whittier College.  First, I retold the story of the evolution of my teaching in introductory physics courses.  Second, I shared how what I learned teaching introductory courses has helped me to teach advanced courses.  Third, I discussed my experiences teaching College Writing Seminar and liberal arts courses.  As always, I look forward to your insights and wisdom.

\end{document}
