\documentclass[../../../main.tex]{subfiles}

\begin{document}

\begin{table}[ht]
\scriptsize
\centering
\begin{tabular}{| c | c | c | c | c | c |}
\hline \hline
Semester & Course & Credits & Students & Curriculum feature & Q16, Q25 Scores \\ \hline \hline
Fall 2017 & PHYS135A-01 & 4.0 & 24 & Intro & $3.19$, $3.24$ \\ \hline
Fall 2017 & PHYS150-01 & 4.0 & 17 & COM1/Intro & $3.56$, $3.13$ \\ \hline
Spring 2018 & PHYS135B-01 & 4.0 & 18 & Intro & $2.94$, $3.12$ \\ \hline
Spring 2018 & PHYS180-02 & 5.0 & 19 & COM1/Intro & $3.83$, $3.61$ \\ \hline
Spring 2018 & COSC330/PHYS306 & 3.0 & 6 & Advanced & $3.29$, $2.88$ \\ \hline
Fall 2018 & PHYS135A-01 & 4.0 & 24 & Intro & $3.92$, $4.13$ \\ \hline
Fall 2018 & PHYS135A-02 & 4.0 & 26 & Intro & $3.88$, $3.96$ \\ \hline
Jan 2019 & COSC390 & 3.0 & 8 & Advanced & $4.50$, $4.75$ \\ \hline
Spring 2019 & PHYS135B-01 & 4.0 & 25 & Intro & $4.33$, $4.46$ \\ \hline
Spring 2019 & PHYS180-02 & 4.0 & 9 & Intro/COM1 & $5.00$, $5.00$ \\ \hline
Fall 2019 & PHYS135A-01 & 4.0 & 24 & Intro & $4.50$, $4.70$ \\ \hline
Fall 2019 & PHYS150-02/03 & 4.0 & 26 & COM1/Intro & $4.80$, $4.80$ \\ \hline
Fall 2019 & INTD255 & 3.0 & 23 & CON2 & $4.60$, $4.80$ \\ \hline
Spring 2020 & COSC330/PHYS306 & 3.0 & 13 & Advanced & $4.80$, $4.60$ \\ \hline
Spring 2020 & PHYS135B-01 & 4.0 & 23 & Intro & $4.70$, $4.50$ \\ \hline
Spring 2020 & PHYS180-02 & 4.0 & 24 & COM1/Intro & $4.90$, $5.00$ \\ \hline
Summer 2020 & MATH080 & 3.0 & 11 & Intro & $4.80$, $4.80$ \\ \hline
Fall 2020 (Mod. 1) & INTD100-21 & 3.0 & 14 & Intro & $4.80$, $5.00$ \\ \hline
Fall 2020 (Mod. 2) & PHYS330 & 3.0 & 11 & Advanced & $4.40$, $4.90$ \\ \hline
Spring 2021 (Mod. 1) & INTD290 & 3.0 & 26 & CON2,CUL3 & $4.60$, $4.80$ \\ \hline
Spring 2021 (Mod. 2) & PHYS135B-02 & 4.0 & 17 & Intro & $4.90$, $4.90$ \\ \hline
Spring 2021 (Mod. 3) & PHYS135B-01 & 4.0 & 25 & Intro & $4.90$, $4.90$ \\ \hline
Fall 2021 & COSC330/PHYS306 & 3.0 & 16 & Advanced & $3.70$, $4.30$ \\ \hline
Fall 2021 & PHYS135A-01 & 4.0 & 24 & Intro & $4.90$, $4.90$ \\ \hline
Fall 2021 & PHYS135A-02 & 4.0 & 25 & Intro & $4.60$, $5.00$ \\ \hline
Jan 2022 & COSC360 & 3.0 & 16 & Advanced & $4.60$, $5.00$ \\ \hline
Spring 2022 & PHYS135B-02 & 4.0 & 25 & Intro & $4.80$, $4.90$ \\ \hline
Spring 2022 & PHYS330 & 3.0 & 12 & Advanced & $4.60$, $5.00$ \\ \hline
Summer 2022 & MATH080 & 3.0 & 3 & Intro & $--$ \\ \hline \hline
\textbf{Students/Cr.:} 5.1 & \textbf{Cr./Course:} 3.6 & \textbf{Students/Course:} 18.4 & \textbf{Cr./year:} 21 & \textbf{Adv./Total:} 24\% & \\ \hline
\hline
Fall 2019 & PHYS396 & 1.0-3.0 & 3 & Research/Cr. & $5.00$, $5.00$ \\ \hline
Spring 2020 & PHYS396 & 1.0-3.0 & 1 & Research/Cr. & $5.00$, $5.00$ \\ \hline
Spring 2022 & PHYS396 & 1.0-3.0 & 3 & Research/Cr. & $5.00$, $5.00$ \\ \hline
\end{tabular}
\caption{\label{tab:classes} A summary of all courses taught at Whittier College.  Physics Research, PHYS396, is included below the summary statistics.  The definitions of Q16 and Q25 from the course evaluation are: Q16 ``Overall, I would recommend this course to others,'' and Q25 ``Overall, I would recommend this professor to others.''  The Q16 and Q25 data were reported with standard deviations in the original paper system, but just the mean and median in the electronic system.  The means are reported here for consistancy.  For MATH080 in Summer 2022, my three students did not provide data.}
\end{table}

The data in Tab. \ref{tab:classes} are meant to share a portrait of my teaching over the last five years, not including Fall 2022.  Appropriate benchmarks in my department are to teach 20 credits per year, and I have achieved 21 credits per year, without including Physics Research, PHYS396.  Our average student-to-faculty ratio is 12, meaning my courses tend to have $\approx 6$ students above average (18.6 students per course), or about a 55\% increase.  All of my courses are either 3 or 4 credits, excluding PHYS396 (variable 1-3).  The average is 3.6 credits per course, meaning a slight majority of my credits come from 4-credit courses.  I have achieved a students per credit ratio of 5.1.  A 3.0-credit course with 12 students would have a students/credit ratio of 4.0.  A ratio of 5.1 indicates that I teach $\approx 20$\% more students/credit.  I have taught a mixture of introductory STEM courses, liberal arts courses, and advanced technical subjects.  About three quarters of my courses have not been advanced technical subjects (advanced ratio: 24\%).  The results in Tab. \ref{tab:classes} show a healthy mixture of courses for a professor at Whittier College.  In the next section, I share with you my summary reflection on trends in my teaching.

\subsection{Summary Reflection on Teaching}

Before turning to my scholarly work, it is important to reflect on my teaching as a whole during my time at Whittier College\footnote{The chair of FPC has informed me that I should include a 4-year history of course evalations, which are included in the Supplemental Material.}.  This reflection highlights my long-term growth and devlopment as an educator.  This reflection is intended to provide context for those who have not encountered my previous teaching results and analysis.  In the reflections below, I have chosen to remain as concise as possible.  There are three main themes.  The first theme is that learning to teach introductory physics courses in my first several semesters required major adjustments.  In my teaching and mentorship roles during my post-doctoral fellowships at The University of Kansas and The Ohio State University, I was exposed to high-performing students in physics.  The transition to teaching students at Whittier College required three changes in my style that we made based on student evaluation data.  The positive effect these changes made was analyzed in fine numerical detail in my supplemental PEGP submitted in Fall 2019.  The transition is visible in Tab. \ref{tab:classes}. Since then, my scores in introductory physics courses have indicated successful outcomes.
\\
\vspace{0.25cm}
The second theme is that I have applied a balanced version of these lessons to advanced physics and computer science courses with great success.  I have earned high scores consistently in these courses, with few exceptions.  I have taught advanced physics courses, and advanced electrical engineering and computer science courses.  The most difficult course to maintain in this regard is Computer Logic and Digital Circuit Design (COSC330/PHYS306).  The first time I taught this course was during my transition to teaching introductory physics at Whittier College.  We struggled with the material, primarily because the textbook and lab workbooks I chose were too difficult and tailored for engineering majors at MIT.
\\
\vspace{0.25cm}
Since then, I have chosen a much better course text, \textit{Digital Fundamentals}, 11th ed. by T. Floyd \cite{digitalFund} (see Secs. 2.5 and 2.6 of my prior PEGP).  I have incorporated an open-source digital circuit and code platform, the PYNQ-Z1 from Xilinx for Higher Education.  The students loved this material the second and third time I taught COSC330/PHYS306.  There were problems, however, when the course grew to 16 students (see Sec. \ref{sec:adv_eval}), and these are easy to fix.  These effects are visible in Tab. \ref{tab:classes}, where the students gave average results for \textit{the course,} but higher scores for \textit{the professor.}  Both times I taught Digital Signal Processing (DSP, COSC390 and later COSC360), the students had wonderful experiences and requested (both times) that the course be made into a semester-long course.  I propose that COSC330/PHYS306 and COSC360 be formed into a two-semester sequence, each worth 4.0 credits.  Of course students should be able to take either course separately, but those that choose to take them in order should benefit from the synergy between the two subjects.  Other benefits would include using the PYNQ-Z1 system in DSP, and re-using the COSC330/PHYS306 for maximum efficiency.
\\
\vspace{0.25cm}
The third theme is that I have achieved a proven track record of being able to teach outside my comfort zone.  This includes two sections of INTD100 and two CON2 and CUL3 courses.  The first CON2 course I taught, ``Safe Return Doubtful: History and Current Status of Modern Science in Antarctica,'' inspired my student Sophie Frizzell to become a leader in wildlife conservation and inspired my WSP student, Nicolas Bakken-French, to travel all over the world studying climate science.  The course was very well received overall.  The second CON2/CUL3 course I taught was History of Science in Latin America.  As I covered in great detail in my prior PEGP, this subject is close to my heart given the Mexican-American heritage of much of my family.  One student, Scout Mucher, found the course so inspiring from the standpoint of exploration that Scout decided to venture to Antarctica this Fall!
\\
\vspace{0.25cm}
My INTD100 sections tend to focus on technical and professional writing mechanics, with the themes of popular science writing and lately, the philosophy of science.  The students seem to enjoy their time with me in these writing courses even though I have no training or experience as a English or literature professor.  Finally, I recall your suggestion to bring back the History of Science in Latin America course.  I assure you that I have not forgotten, and I plan to bring this course back after my potential sabbatical.
\\
\vspace{0.25cm}
Returning to the first theme, I share here the three major changes we made to my introductory physics courses.  More detail can be found in the numerical analysis I performed for my supplemental PEGP submitted in Fall 2019, which was also summarized in my PEGP submitted in Fall 2021.

%1. Teaching in 2017-2018 introductory physics was a challenge
%	a) AAPT helped
%	b) Colleagues helped
%	c) Changing the way I did testing helped, developing a good relationship with the students
%	d) Summarize three changes made
%2. Now teaching introductory physics courses was a breeze
%	a) Course evaluations reflect this
%	b) Pandemic did not lower my scores, and in some cases, raised them
%3. Applying the lessons to my advanced courses has led to great success overall
%4. I have created new advanced courses and liberal arts courses that have generated great success
%5. Teaching INTD100 is challenging for a physical scientist, but I have managed to do it well
%	a) Scores from 1st INTD100 were good, but it was during the module system and fully remote
%	b) I am currently teaching INTD100 and it is going well, but time will tell
%	
%I have not forgotten your suggestion to bring back my CON2 course about the history of science in Latin America.  I can make this happen after I return from sabbatical, should I be granted tenure.  I welcome any other suggestions you might have.

%  I also have discussed taking on the responsibility of teaching Computational Physics from Prof. Lagan, who plans to retire in the next few years.
% Taking over the 3-2 program

\end{document}