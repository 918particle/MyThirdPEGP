\documentclass[../../../main.tex]{subfiles}

\begin{document}

What follows is a description of the advanced physics and computer science I taught in AY 2021-2022, with connections to departmental goals, learning focuses listed in Appendix xxx, and tenets of my teaching philosophy from Sec \ref{sec:teaching_philosophy}.
\\
\vspace{0.25cm}
\textbf{\textit{Computer Logic and Digital Circuit Design (PHYS306/COSC330)}}. Computer Logic and Digital Circuit Design is cross-listed as PHYS306/COSC330.  My first task for the design of this course was to use the advanced learning focus of \textit{strength in all phases of science}, and \textit{to satisfy departmental goals 4-7}.  The topic of electrical engineering has many phases, from the underlying mathematics to the hands-on experience of designing and building logic circuits.  This is a 300-level course that satisfies requirements in the following majors: Physics, ICS/Physics, ICS/Economics, 3-2 Engineering/Math, 3-2 Engineering/Computer Science.  Courses like COSC330 that serve such a variety of students should touch on at least the following sub-topics: (i) binary mathematics, non-decimal base systems, and boolean logic, (ii) basic digital components, clocks and gates, (iii) implementation of boolean algebra with complex logic functions (iv) creation of digital circuits and projects, and (v) analysis of digital data, analogue-to-digital and digital-to-analogue conversion (ADC and DAC).
\\
\vspace{0.25cm}
In the design of this course, I kept in mind my advanced learning focus of \textit{mental discipline} by attempting to make the course challenging for the students within each of the above topics.  As with any physical science or engineering course, the design must include the following phases: \textit{mathematics, computer programming and modeling, hardware design and testing, and digital data analysis} (strength in all phases of science)\footnote{An example syllabus is in the supporting material.}.  As with many of my courses, I include student-driven final projects in the design, to meet my advanced learning focus of \textit{communication} within a technical subject.  As I teach the course I've designed, I use the active learning strategies (1)-(6) developed in my teaching philosophy (Sec. \ref{sec:teaching_philosophy}).  One key difference is that I rarely use learning activity (2) (peer-instruction) in this advanced course, because PI works best for larger courses.
\\
\vspace{0.25cm}
The first half of COSC330/PHYS306 class sessions are spent learning number systems like binary and hexidecimal, boolean algebra and logic functions, and increasingly complex logic circuits.  We apply these concepts to designing digital circuits, and we also cover how these tools have applications in business.  One example is how to simplify a logic function, and how that technique can simplify a business workflow.  The digital design labs rely on the PYNQ-Z1 system-on-a-chip (SoC).  To set up a system the student, I install a Linux operating system on a laptop, and use it to download an operating system for the SoC.  The PYNQ-Z1 is called a SoC because it has both programmable logic firmware (PL) and a processing system (PS) like a small computer.  The default operating system for PYNQ-Z1 contains example code the students use to learn.  Students create digital circuits inside the SoC by writing code in Python3.  When traditional lecture and lab activities are woven together in this course (learning activities (1) and (4) of my teaching philosophy), the students achieve an understanding of embedded digital systems.
\\
\vspace{0.25cm}
I provide a bridge to one of my other advanced courses, COSC360 (digital signal processing - DSP), by finishing the lab activities (tenet (4) of my teaching philosophy) on an ADC lab.  The students capture analogue voltage signals from an external source, digitize it using the SoC, and graph it using Python3.  This lab activity serves as model for how sensor data is digitized, and the starting point for DSP.  For the final projects, the students design sensors, timing-based systems like traffic-light controllers, video and sound processing systems, and robotic systems.  As with the final projects in my introductory courses, the results can be presented with digital storytelling or live.  The students leap at the chance to express themselves scientifically with these projects.
\\
\vspace{0.25cm}

\textbf{\textit{Digital Signal Processing (COSC390)}}.  Digital Signal Processing (DSP) is now listed as COSC360, and it is a 300-level ICS course satisfying requirements in the following majors: Physics, ICS/Physics, ICS/Economics, 3-2 Engineering/Math, 3-2 Engineering/Computer Science.  Similar to the design of COSC330/PHYS306, I used \textit{Physics Department goals 4-7} and my advanced coures learning focuses to design (and re-design) this course.  One key difference is that I \textit{also} use two aspects of my introductory learning focuses when designing this course (\textit{curiosity and applications to society}).  The reason is that a portion of my COSC360 students are non-majors taking a January-term course simply because it sounds interesting.
\\
\vspace{0.25cm}
DSP encompasses the myriad of ways we capture, process, and filter analogue signals like audio signals, medical sensors on the heart and brain, and images.  DSP gives students vital practice with loading, cleaning, manipulating, and graphing data.  DSP is used by a wide variety of students in fields beyond engineering.  Our students also practice a vital STEM skill in DSP: Fourier analysis.  This technique allows a student rearrange data comprised of \textit{time-dependent} signals into \textit{frequency-dependent} signals.  One economics student, for example, was able to find periodic trends in the stock market using this technique.
\\
\vspace{0.25cm}
My course design touches upon the first advanced learning focus (\textit{mental discipline}), by including both the underlying mathematics of DSP, and practical applications.  One such application is the sampling, digital filtering, and visualization of electronic music.  I have taught this course twice, once in January 2019, and again this past January 2022.  We meet for three hours each morning for three weeks in the January term format.  Homework sets are assigned each day, and kept short but challenging.  The students find this style refreshing and efficient.  I would like to re-design the course for a semester-length format, and the students have requested that in the past.  If given the chance to do that, I would incorporate the first advanced learning focus through more extended applications to music, the stock market, and image analysis.
\\
\vspace{0.25cm}
The second learning focus for advanced course design is \textit{strength in all phases of science.}  COSC360 follows COSC330/PHYS306 conceptually.  One can think of COSC330/PHYS306 as learning the building blocks of digital components.  Some of those components help create scientific instruments that sample and digitize analog data.  DSP is the subject of what follows \textit{after} sampling and digitization.  The students have creatively applied the material: my student Noah Wilson created an analysis of Federal Reserve interest rate data over many decades using DSP, and my student Jake Householder used Fourier analysis to study periodic trends in the stock market.  As with COSC330/PHYS306, we conclude the course with final projects (learning activity (6) of my teaching philosophy) that foster communication of technical ideas in line with department goals.
\\
\vspace{0.25cm}
\textbf{\textit{Electromagnetic Theory (PHYS330)}}.  Electromagnetic theory is a course taught in every department of physics.  It is an advanced theoretical physics course utilized by 3-2 engineering, physics, and math students.  It builds upon vector calculus (MATH241) and calculus-based physics II (PHYS180), and vector calculus is built from single-variable calculus (MATH141 and MATH142).  I taught this course in Spring 2022 for the first time in-person.  I utilized my standard toolkit of learning activities (1)-(6) from Sec. \ref{sec:teaching_philosophy}.  However, the students in this course are more advanced, and ready for less of learning activity (2) (peer instruction) and more of learning activity (1) (traditional lecture with warm ups).
\\
\vspace{0.25cm}
The first learning focus for advanced course design is \textit{mental discipline}, so I begin the course with a rigorous review of vector calculus, augmented with online tutorial videos.  We use the ubiquitous textbook for this course entitled \textit{Introduction to Electrodynamics, 3rd ed.} by David Griffiths.  The book contains a thorough mathematical warm-up chapter, and we choose to complete it and draw upon the concepts throughout the semester.  When I was in college, we used the 2nd edition and spent one 14-week semester covering most of the book.  Other institutions spend two full semesters covering the whole book.  My task set by my department is to cover the first half of the book in one semester.  This exposes the students to (i) review of vector calculus, (ii) the distribution of electric charge and the electric fields created by charge, (iii) the energy and voltage corresponding to those fields, (iv) how those fields behave inside materials, (v) moving charge, current, and magnetic fields, and the associated energy, (vi) the behavior of magnetic fields behave in materials, and (vii) how light is actually radiating electromagnetic waves.
\\
\vspace{0.25cm}
The second and third learning focuses for advanced course design are \textit{strength in all phases of science,} and \textit{communication.}  PHYS330 is one of our upper-division electives that is centered on abstract problem-solving and numerical prediction/modeling, so laboratory activities do not play a role.  We build beautiful, intricate theoretical ideas that the students will use in future engineering and scientific contexts.  I try to include demonstrations of modern computational electromagnetism (CEM), with experience drawn from my research with the Office of Naval Research (ONR).  See Sec. \ref{sec:scholarship} for more details.  As with COSC330/PHYS306 and COSC360, we conduct final projects in line with learning activity (6) of my teaching philosophy, department goals, and my third learning focus for advanced courses.

\end{document}
