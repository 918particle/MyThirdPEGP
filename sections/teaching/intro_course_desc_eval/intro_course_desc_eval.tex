\documentclass[../../main.tex]{subfiles}
 
\begin{document}

\begin{table}[ht]
\scriptsize
\centering
\begin{tabular}{| p{1.5cm} | p{14cm} |}
\hline
\hline
Question & Text \\ \hline
10 & This course had clear goals and objectives. \\ \hline
11 & This course was academically challenging. \\ \hline
12 & This course offered useful learning tools (such as lectures, discussions, readings, assignments and/or examinations). \\ \hline
13 & This course had grading criteria that were clearly identified. \\ \hline
14 & This course improved my understanding of the material. \\ \hline
15 & This course increased my interest in the subject matter. \\ \hline
16 & Overall, I would recommend this course to others. \\ \hline
17 & The professor used class time effectively and demonstrated preparation for class. \\ \hline
18 & The professor's teaching style and/or enthusiasm for the material strengthened my interest in the subject matter. \\ \hline
19 & The professor was able to explain complicated ideas. \\ \hline
20 & The professor challenged students to think critically and/or imaginatively about the course material. \\ \hline
21 & The professor provided clear and timely feedback. \\ \hline
22 & The professor encouraged meaningful class discussions. \\ \hline
23 & The professor was receptive to differing views. \\ \hline
24 & The professor was available for help outside of class. \\ \hline
25 & Overall, I would recommend this professor to others. \\ \hline
\hline
\end{tabular}
\caption{\label{tab:questions} The listing of standard course evaluation quesions.}
\end{table}

The course evaluations for PHYS135A/B taught in AY 2021-2022 are shared below.  These courses were the only introductory courses I taught this past year.  On the course evaluations, questions 10-16 pertain to the course, and questions 17-25 pertain to the professor.  Table \ref{tab:questions} lists the standard questions, and the course evaluation data for algebra-based physics is shown in Tab. \ref{tab:eval_135}.  The data cover three courses: sections 1 and 2 of 135A (Fall), and section 2 of 135B (Spring).  Fall courses (135A) cover units and vectors, kinematics, forces, energy, and momentum.  Spring courses (135B) cover electromagnetism topics like charge, fields, current, DC circuits, and magnetism.
\\
\vspace{0.25cm}

\begin{table}
\scriptsize
\centering
\begin{tabular}{| c | c | c | c |}
\hline
\hline
Question & Fall 2021 (1) & Fall 2021 (2) & Spring 2022 (2) \\ \hline
10 & 5.0 & 4.8 & 4.9 \\ \hline
11 & 4.8 & 5.0 & 4.9 \\ \hline
12 & 5.0 & 5.0 & 4.9 \\ \hline
13 & 5.0 & 4.6 & 4.7 \\ \hline
14 & 4.9 & 4.6 & 4.8 \\ \hline
15 & 4.5 & 4.1 & 4.7 \\ \hline
16 & 4.9 & 4.6 & 4.8 \\ \hline
17 & 4.8 & 4.9 & 4.9 \\ \hline
18 & 4.8 & 4.8 & 4.8 \\ \hline
19 & 4.7 & 5.0 & 4.9 \\ \hline
20 & 4.8 & 4.9 & 4.9 \\ \hline
21 & 4.8 & 4.9 & 4.8 \\ \hline
22 & 4.6 & 4.6 & 4.8 \\ \hline
23 & 4.9 & 5.0 & 4.8 \\ \hline
24 & 4.9 & 5.0 & 5.0 \\ \hline
25 & 4.9 & 5.0 & 4.9 \\ \hline
\hline
\end{tabular}
\caption{\label{tab:eval_135} Course evaluation results for PHYS135A sections 1 and 2, and PHYS135B section 2.  Questions 10-16 refer to the course, and questions 17-25 refer to the professor.}
\end{table}

I have reflected on the student feedback for these courses.  I find that the numerical results are in line with recent years.  My implementation of the teaching philosophy in Sec \ref{sec:teaching_philosophy} is generating very positive feedback from the students.  Looking for numerical trends in Tab. \ref{tab:eval_135}, I do see that I still struggle occasionally to boost student interest in physics (Question 15).  This is true of the course, historically, as many students are nervous to take this required course.  This year-long sequence of algebra-based physics is a must-pass course for pre-medical students and KNS majors.
\\
\vspace{0.25cm}
The written responses provide some good clues as to how to fine-tune algebra-based physics.  In Sec. \ref{sec:teaching_philosophy}, I was reflecting on whether or not to include more online laboratory activities, in addition to the laboratory activities done on the lab bench.  There seems to be a common theme in the written responses: the students enjoy the lab activities and even want more of them.  The online labs sound like they do not hurt, but there is definitely a preference for hands-on labs.  This could help boost the interest and excitement of those last few students who are less interested in physics at first.
\\
\vspace{0.25cm}
The other student comments were about little adjustments they would appreciate: more consistent uploading of grades to Moodle, taking a break in the middle class sessions, and removing the custom questions from the online homework system.  Regarding the latter, I have done that already.  The online homework system is fully integrated with our OpenStax textbooks (see Sec. 2.2.1 of my prior PEGP), and artificial intelligence is included.  The AI tries to assign extra problems to homework sets that are tailored to the student.  The idea is to nudge the student to study areas in which they need more practice.  We found this feature to be glitchy, and did not add much value.  Overall, it is very heartwarming to read the students' positive encouragements in their written responses.

\end{document}
