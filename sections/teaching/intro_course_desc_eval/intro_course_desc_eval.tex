\documentclass[../../main.tex]{subfiles}
 
\begin{document}

\begin{table}[ht]
\scriptsize
\centering
\begin{tabular}{| p{1.5cm} | p{15cm} |}
\hline
\hline
Question & Text \\ \hline
10 & This course had clear goals and objectives. \\ \hline
11 & This course was academically challenging. \\ \hline
12 & This course offered useful learning tools (such as lectures, discussions, readings, assignments and/or examinations). \\ \hline
13 & This course had grading criteria that were clearly identified. \\ \hline
14 & This course improved my understanding of the material. \\ \hline
15 & This course increased my interest in the subject matter. \\ \hline
16 & Overall, I would recommend this course to others. \\ \hline
17 & The professor used class time effectively and demonstrated preparation for class. \\ \hline
18 & The professor's teaching style and/or enthusiasm for the material strengthened my interest in the subject matter. \\ \hline
19 & The professor was able to explain complicated ideas. \\ \hline
20 & The professor challenged students to think critically and/or imaginatively about the course material. \\ \hline
21 & The professor provided clear and timely feedback. \\ \hline
22 & The professor encouraged meaningful class discussions. \\ \hline
23 & The professor was receptive to differing views. \\ \hline
24 & The professor was available for help outside of class. \\ \hline
25 & Overall, I would recommend this professor to others. \\ \hline
\hline
\end{tabular}
\caption{\label{tab:questions} The listing of standard course evaluation quesions.}
\end{table}

The course evaluations for PHYS135A/B taught in AY 2021-2022 are shared below.  These courses were the only introductory courses I taught this past year.  On the course evaluations, questions 10-16 pertain to the course, and questions 17-25 pertain to the professor.  Table \ref{tab:questions} lists the standard questions.

\begin{table}
\scriptsize
\centering
\begin{tabular}{| c | c | c | c | c |}
\hline
\hline
Question & Fall 2021 (1) & Fall 2021 (2) & Spring 2022 (1) & Spring 2022 (2) \\ \hline
10 & x & y & a & b \\ \hline
11 & x & y & a & b \\ \hline
12 & x & y & a & b \\ \hline
13 & x & y & a & b \\ \hline
14 & x & y & a & b \\ \hline
15 & x & y & a & b \\ \hline
16 & x & y & a & b \\ \hline
\hline
\end{tabular}
\begin{tabular}{| c | c | c | c | c |}
\hline
\hline
Question & Fall 2021 (1) & Fall 2021 (2) & Spring 2022 (1) & Spring 2022 (2) \\ \hline
17 & x & y & a & b \\ \hline
18 & x & y & a & b \\ \hline
19 & x & y & a & b \\ \hline
20 & x & y & a & b \\ \hline
21 & x & y & a & b \\ \hline
22 & x & y & a & b \\ \hline
23 & x & y & a & b \\ \hline
24 & x & y & a & b \\ \hline
25 & x & y & a & b \\ \hline
\hline
\end{tabular}
\caption{\label{tab:eval_135} (Left) Course evaluation results for PHYS135, course questions.  (Right) Course evaluation results for PHYS135, professor questions.}
\end{table}

The course evaluation data for algebra-based physics is shown in Tab. \ref{tab:eval_135}.  Table \ref{tab:eval_135} (left) contains the results for the course, while Tab. \ref{tab:eval_135} (right) contains the results for the professor.  The question definitions are listed in Tab. \ref{tab:questions}.  The data cover four courses: sections 1 and 2 of 135Av(Fall), and sections 1 and 2 of 135B (Spring).  Fall courses (135A) cover units and vectors, kinematics, forces, energy, and momentum.  Spring courses (135B) cover electromagnetism topics like charge, fields, current, DC circuits, and magnetism.

\end{document}
