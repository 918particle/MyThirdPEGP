\documentclass[../../main.tex]{subfiles}

\begin{document}

The course evaluations for advanced STEM courses described in Sec. \ref{sec:adv} are shared below.

\subsubsection{Computer Logic and Digital Circuit Design}

\begin{table}
\footnotesize
\centering
\begin{tabular}{| c | c |}
\hline
\hline
Question & Fall 2021 \\ \hline
10 & 4.9 \\ \hline
11 & 4.9 \\ \hline
12 & 3.9 \\ \hline
13 & 4.3 \\ \hline
14 & 3.9 \\ \hline
15 & 3.7 \\ \hline
16 & 3.7 \\ \hline
17 & 5.0 \\ \hline
18 & 4.4 \\ \hline
19 & 4.0 \\ \hline
20 & 4.4 \\ \hline
21 & 5.0 \\ \hline
22 & 4.8 \\ \hline
23 & 5.0 \\ \hline
24 & 5.0 \\ \hline
25 & 4.3 \\ \hline
\hline
\end{tabular}
\caption{\label{tab:eval_330_306} Course evaluation results for COSC330/PHYS306.}
\end{table}

The course evaluation results for COSC330/PHYS306 are shown in Tab. \ref{tab:eval_330_306}.  The questions pertaining to the professor (questions 17-25) are generally higher than those pertaining to the course (questions 10-16).  When I saw the numbers in Tab. \ref{tab:eval_330_306}, I was not surprised and I know of three reasons why the course went the way it did.  First, some history is relevant.  This is the third time I have taught this course, and I have built experience teaching it.  The last time I taught this course was during Spring 2020, and the scores were very high despite the rapid transition to online teaching that semester (see Sec. 2.6 of my prior PEGP).  In Spring 2020, I found the right balance of laboratory activities and traditional lecture, and the transition to DSP topics worked well.  The students felt like hackers from the movies and were generally impressed by the material.  Students gave me high marks and praise in the written responses.  For Fall 2021, some students felt the exact opposite about the course material as their peers from 2020.
\\
\vspace{0.25cm}
I think there are three reasons why this occurred.  First, I was asked to move the course to the Fall semester, which placed it after some of the COSC students had taken Prof. Lutgen's course on microprocessors.  Some students asked me if my course would be too similar to microprocessors.  I told them that my course was about what happens (essentially) beneath the hood of microprocessors: the digital circuits that make up the logic of computer chips.  Since I was the one who created this course from scratch in Spring 2018, I've always had a mindset of recruitment so that enough students take it to justify the work required to put together a complex integrated lecture-lab course on engineering.  This led me to accept students that I should not have accepted into the course, namely those who already took microprocessors, for they would be disappointed with the introductory material.  It seems like one student wrote along these lines in the evaluations.
\\
\vspace{0.25cm}
The second reason is that I accepted too many students into the course (16) when it worked well at 12 students.  It turns out that at least one of my students was not prepared to take this advanced course, and I should have provided better guidance about when they should enroll.  Also, the integrated lecture-lab format is highly effective when I had two students per lab group in this course.  However, with the higher number of students, some lab groups had to be three people which left one person passively observing.  The lab component of this course drove its success last time, so the larger class made the students' lab activities more awkward.
\\
\vspace{0.25cm}
The third reason is related to the first.  With the larger class size, I could not devote the time I would have liked to the students who were struggling.  Toward the end of the course, some were mastering the material, some had seen it already in microprocessors (with Prof. Lutgen), and some where struggling.  I had to advise the students' final projects in addition to helping with homework and labs, and it felt a lot harder than in Spring 2020.  As I planned the transition to the DSP material in COSC330, and began to pitch the Jan Term COSC360 (DSP) course, I noticed that a subset of my students were eager to take it, while others were not.  The DSP students later gave me stellar reviews in COSC360 (see below).
\\
\vspace{0.25cm}
Thus, I recommend four modifications for COSC330.  First, I think a smaller class size ($\approx 12$) will serve students better due to the nature of the integrated lecture-lab format.  Second, we need better coordination with the Math department so that the course does not overlap with microprocessors (COSC390).  Third, we should spend more time in the laboratory, as it seems students prefer learning activity (4) (from Sec. \ref{sec:teaching_philosophy}) as opposed to learning activity (1) for this type of course.  Finally, since the students seemed to appreciate the DSP content so much, we should further integrate it into COSC330.  I would like to form a two-semester sequence from these two courses, COSC330 and COSC360.  I do not mean one would be pre-requisite for the other, but that the two courses should be more integrated.

\subsubsection{Digital Signal Processing}

\begin{table}
\footnotesize
\centering
\begin{tabular}{| c | c |}
\hline
\hline
Question & Jan 2022 \\ \hline
10 & 4.9 \\ \hline
11 & 5.0 \\ \hline
12 & 5.0 \\ \hline
13 & 4.8 \\ \hline
14 & 4.9 \\ \hline
15 & 4.5 \\ \hline
16 & 4.6 \\ \hline
17 & 5.0 \\ \hline
18 & 4.9 \\ \hline
19 & 4.6 \\ \hline
20 & 4.8 \\ \hline
21 & 4.6 \\ \hline
22 & 4.5 \\ \hline
23 & 4.9 \\ \hline
24 & 4.9 \\ \hline
25 & 5.0 \\ \hline
\hline
\end{tabular}
\caption{\label{tab:eval_dsp} Course evaluation results for COSC360.}
\end{table}

The results for DSP (COSC390) are shown in Tab. \ref{tab:eval_dsp}.  This is the second time I have taught this course.  The students rewarded me with high marks.  There is a common refrain through the written responses, and it is familiar from the last time I taught DSP (Jan 2019).  The students are asking that we make this a full-semester course.  Now that this course is in the course catalog as a full course, with a designated course number (as opposed to COSC390 - special topics), we are prepared to offer it regularly.  The final hurdle is to sort out with the Department of Mathematics when we should teach COSC330 and when we should teach COSC360.  Normally, this would cause tension because there are a finite number of students who would take advanced programming and engineering courses.
\\
\vspace{0.25cm}
According to our institutional research, this is changing.  COSC360 is a course suited for students in the ICS/Math major.  The ICS/Math major has grown by more than 900\% since 2018.  Given the demand for graduates in engineering, data science, and software development, there should be plenty of students to support a semester-long version of DSP.  Both sets of students in 2019 and 2022 requested it, and now there will be no January term going forward.  There is a wrinkle in this argument, because Prof. Glenn Piner now teaches a data science course (COSC180).  COSC360 is at the 300-level, however, with more advanced mathematics.  Thus, I expect first-years and sophomores to take COSC180, and juniors and seniors to take COSC360.  Overall, I am very pleased with the results of DSP, including the fact that we have served non-majors who have performed interesting final projects despite not having the same technical background as the majors.

%\subsubsection{Electromagnetic Theory}
%
%\begin{table}
%\footnotesize
%\centering
%\begin{tabular}{| c | c |}
%\hline
%\hline
%Question & F2020 \\ \hline
%10 & 4.9 \\ \hline
%11 & 5.0 \\ \hline
%12 & 4.9 \\ \hline
%13 & 4.9 \\ \hline
%14 & 4.6 \\ \hline
%15 & 4.6 \\ \hline
%16 & 4.4 \\ \hline
%\hline
%\end{tabular}
%\begin{tabular}{| c | c |}
%\hline
%\hline
%Question & F2020 \\ \hline
%17 & 4.9 \\ \hline
%18 & 4.8 \\ \hline
%19 & 4.6 \\ \hline
%20 & 4.6 \\ \hline
%21 & 4.5 \\ \hline
%22 & 4.7 \\ \hline
%23 & 4.7 \\ \hline
%24 & 4.9 \\ \hline
%25 & 4.9 \\ \hline
%\hline
%\end{tabular}
%\caption{\label{tab:eval_330} (Left) Course evaluation results for PHYS330 pertaining to the course.  (Right) Course evaluation results for PHYS330 pertaining to the professor.}
%\end{table}
%
%The course evaluation results for PHYS330 are shown in Tab. \ref{tab:eval_330}.  In general, the results are good.  I was aiming for a higher mark on Question 16 (recommend this course to others), which prompted me to search through the written responses for clues. Several students shared that this course is not really appropriate for seven weeks.  I gave examples of how this course is normally distributed through the weeks of a semester in Sec. \ref{sec:adv}.  There were other suggestions, though, that were interesting and useful for next time.  One student related that if there are going to be videos, they would be better if they included traditional teaching in addition to example problems.  Another student suggested that in an online advanced course, warm-ups should be shortened, or done as a group.
%\\
%\vspace{0.15cm}
%Most of my reflection on my teaching lately has been geared towards the introductory physics courses.  The students in advanced physics courses have slightly different preferences.  These issues are made easier in a semester system, and I will take them into account the next time I teach PHYS330.  More traditional teaching, tutorial videos, and perhaps more CEM, will satisfy the students.  When asked if they would change anything, the students mostly remarked that two modules, or more time, would have been better.  Finally, one student gave me a boost when he wrote: ``I miss the physics boys.  This was nice.''  I miss you too, bro.

\end{document}
