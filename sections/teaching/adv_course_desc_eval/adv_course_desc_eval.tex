\documentclass[../../main.tex]{subfiles}

\begin{document}

The course evaluations for advanced STEM courses described in Sec. \ref{sec:adv} are shared below.

\subsubsection{Computer Logic and Digital Circuit Design}

\begin{table}
\footnotesize
\centering
\begin{tabular}{| c | c |}
\hline
\hline
Question & Fall 2021 \\ \hline
10 & 4.9 \\ \hline
11 & 4.9 \\ \hline
12 & 3.9 \\ \hline
13 & 4.3 \\ \hline
14 & 3.9 \\ \hline
15 & 3.7 \\ \hline
16 & 3.7 \\ \hline
17 & 5.0 \\ \hline
18 & 4.4 \\ \hline
19 & 4.0 \\ \hline
20 & 4.4 \\ \hline
21 & 5.0 \\ \hline
22 & 4.8 \\ \hline
23 & 5.0 \\ \hline
24 & 5.0 \\ \hline
25 & 4.3 \\ \hline
\hline
\end{tabular}
\caption{\label{tab:eval_330_306} Course evaluation results for COSC330/PHYS306.}
\end{table}

The course evaluation results for COSC330/PHYS306 are shown in Tab. \ref{tab:eval_330_306}.  The questions pertaining to the professor (questions 17-25) are generally higher than those pertaining to the course (questions 10-16).  When I saw the numbers in Tab. \ref{tab:eval_330_306}, I was not surprised and I know of three reasons why the course went the way it did.  First, some history is relevant.  This is the third time I have taught this course, and I have built experience teaching it.  The last time I taught this course was during Spring 2020, and the scores were very high despite the rapid transition to online teaching that semester (see Sec. 2.6 of my prior PEGP).  In Spring 2020, I found the right balance of laboratory activities and traditional lecture, and the transition to DSP topics worked well.  The students felt like hackers from the movies and were generally impressed by the material.  Students gave me high marks and praise in the written responses.  For Fall 2021, some students felt the exact opposite about the course material as their peers from 2020.
\\
\vspace{0.25cm}
I think there are three reasons why this occurred.  First, I was asked to move the course to the Fall semester, which placed it after some of the COSC students had taken Prof. Lutgen's course on microprocessors.  Some students asked me if my course would be too similar to microprocessors.  I told them that my course was about what happens (essentially) beneath the hood of microprocessors: the digital circuits that make up the logic of computer chips.  After creating the course from scratch in Spring 2018, I've always had a mindset of recruitment for COSC330.  I want to ensure that enough students take the course to justify the work required to put together such a large, complex integrated lecture-lab course on engineering.  This led me to accept students that I should not have accepted into the course, namely those who already took microprocessors.  I think these students felt they were repeating the same material on a more introductory level.  It seems like one student wrote along these lines in the evaluations.
\\
\vspace{0.25cm}
The second reason is that I accepted too many students into the course (16) when it worked well at 12 students.  It turns out that at least one of my students was not prepared to take this advanced course, and I should have provided better guidance about when they should enroll.  Also, the integrated lecture-lab format is highly effective when I had two students per lab group in this course.  However, with the higher number of students, some lab groups had to be three people which left one person passively observing.  The lab component of this course drove its success last time, so the larger class made the students' lab activities more awkward.
\\
\vspace{0.25cm}
The third reason is related to the second.  With the larger class size, I could not devote the time I would have liked to the students who were struggling.  Toward the end of the course, some were mastering the material, some had seen it already in microprocessors (with Prof. Lutgen), and some where struggling.  I had to advise the students' final projects in addition to helping with homework and labs, and it felt a lot harder than in Spring 2020.  As we transitioned to the DSP material in COSC330, and as I began to pitch the Jan Term COSC360 (DSP) course, a subset of my students showed eagerness to take COSC360 while others did not.  The students who did take COSC360 later gave me stellar reviews (see below).
\\
\vspace{0.25cm}
Thus, I recommend four modifications for COSC330.  First, I think a smaller class size ($\approx 12$) will serve students better due to the nature of the integrated lecture-lab format.  Second, we need better coordination with the Math department so that the course does not overlap with microprocessors (COSC390).  Third, we should spend more time in the laboratory, as it seems students prefer learning activity (4) (from Sec. \ref{sec:teaching_philosophy}) as opposed to learning activity (1) for this type of course.  Finally, since the students seemed to appreciate the DSP content so much, we should further integrate it into COSC330.  I would like to form a two-semester sequence from these two courses, COSC330 and COSC360.  I do not mean one would be pre-requisite for the other, but that the two courses should be more integrated.

\subsubsection{Digital Signal Processing}

\begin{table}
\footnotesize
\centering
\begin{tabular}{| c | c |}
\hline
\hline
Question & Jan 2022 \\ \hline
10 & 4.9 \\ \hline
11 & 5.0 \\ \hline
12 & 5.0 \\ \hline
13 & 4.8 \\ \hline
14 & 4.9 \\ \hline
15 & 4.5 \\ \hline
16 & 4.6 \\ \hline
17 & 5.0 \\ \hline
18 & 4.9 \\ \hline
19 & 4.6 \\ \hline
20 & 4.8 \\ \hline
21 & 4.6 \\ \hline
22 & 4.5 \\ \hline
23 & 4.9 \\ \hline
24 & 4.9 \\ \hline
25 & 5.0 \\ \hline
\hline
\end{tabular}
\caption{\label{tab:eval_dsp} Course evaluation results for COSC360.}
\end{table}

The results for DSP (COSC390) are shown in Tab. \ref{tab:eval_dsp}.  This is the second time I have taught this course.  The students rewarded me with high marks.  There is a common refrain through the written responses, and it is familiar from the last time I taught DSP (Jan 2019).  The students are asking that we make this a full-semester course.  Now that this course is in the course catalog as a full course, with a designated course number (as opposed to its former number, COSC390: special topics), we are prepared to offer it regularly.  The final hurdle is to sort out with the Department of Mathematics when we should teach COSC330 and when we should teach COSC360.  Normally, this would be a logistical problem because there are a finite number of students who would take advanced programming and engineering courses.
\\
\vspace{0.25cm}
According to our institutional research, this is changing.  COSC360 is a course suited, for example, to students in the ICS/Math major.  The ICS/Math major has grown by more than 900\% since 2018.  Given the demand for graduates in engineering, data science, and software development, I forsee sufficient demand for a semester-long version of DSP.  Both sets of students in 2019 and 2022 requested it, and now there will be no January term going forward.  There is a wrinkle in this argument, because Prof. Glenn Piner now teaches a data science course (COSC180).  COSC360 is at the 300-level, however, with more advanced mathematics.  Thus, I expect first-years and sophomores to take COSC180, and juniors and seniors to take COSC330 and COSC360.  Overall, I am very pleased with the results of DSP, including the fact that we have served non-majors who have performed interesting final projects despite not having the same technical background as the majors.

\subsubsection{Electromagnetic Theory}

\begin{table}
\footnotesize
\centering
\begin{tabular}{| c | c |}
\hline
\hline
Question & F2020 \\ \hline
10 & 4.8 \\ \hline
11 & 5.0 \\ \hline
12 & 4.6 \\ \hline
13 & 4.6 \\ \hline
14 & 4.6 \\ \hline
15 & 4.6 \\ \hline
16 & 4.6 \\ \hline
17 & 4.6 \\ \hline
18 & 4.6 \\ \hline
19 & 4.8 \\ \hline
20 & 5.0 \\ \hline
21 & 5.0 \\ \hline
22 & 4.8 \\ \hline
23 & 5.0 \\ \hline
24 & 5.0 \\ \hline
25 & 5.0 \\ \hline
\hline
\end{tabular}
\caption{\label{tab:eval_330} Course evaluation results for PHYS330.}
\end{table}

The course evaluation results for PHYS330 are shown in Tab. \ref{tab:eval_330}.  In general, the results are good.  Spring 2022 was the first semester in which I taught this course outside the module system, and the first time I've taught it in person.  After reflecting on our experience in PHS330 for Spring 2022, I'm relieved and happy.  This is one of our most advanced theoretical physics courses, and it is based on no fewer than seven pre-requisite courses.  Students must pass Calculus I-III, Calculus-based Physics I-III, and Modern Physics, before taking Electromagnetic Theory.  My students in Spring 2022 took Calculus III during the module system.  As I related in my prior PEGP, asking \textit{human beings} (let along college students who sometimes work jobs) to finish a course like Calculus III in just seven weeks is not reasonable.  People can only absorb so much technical or abstract material in a short time.  I was aware of this issue, however, so I utilized the vector calculus review chapter (Chapter 1) of our textbook to help the students prepare.  In addition to learning activity (1) (Sec. \ref{sec:teaching_philosophy}), I created many video tutorials for my students based on Chapter 1.
\\
\vspace{0.25cm}
One student wrote in the course evaluations that they would like to see more visualization, more 3D modeling and animation of the electromagnetic fields.  It just so happens that my research has been focused lately on computational electromagnetism (CEM) (see Sec. xxx).  I recently gave an invited lecture at MIT on my CEM work.  One of the organizers, a long-time MIT professor of applied physics and math, explained how we can use CEM codes to enhance our teaching of electromagnetic theory.  Incorporating CEM into PHYS330 would be very similar to incorporating the Python3 code that operates the PYNQ-Z1 board in COSC330.  I can use Jupyter\footnote{https://jupyter.org/} notebooks to run CEM calculations hosted on Moodle and run via the web browser.  I think this will enhance PHYS330 for the students by connecting it to modern engineering applications.

\end{document}
