\documentclass[../../../main.tex]{subfiles}

\begin{document}

The following is a reflection on the six main active learning techniques I use in the physical sciences.  I also apply them, as appropriate, to my liberal arts and college writing courses.  Each activity is derived from the over-arching principles of \textit{order} and \textit{shared meaning.}  I wrote about these principles in Sec. 2.1 of my prior PEGP, and I have included relevant passages from my prior PEGP in the Supplemental Material.  Briefly, the term \textit{order} refers to the idea that physics is a logical description of the Universe, and that physics instruction must reflect this logical order.  The term \textit{shared meaning} refers to the unifying idea that physics is a social activity that produces the same correct results for diverse peoples, confirmed by a scientific community committed to the truth.  For each of the six tenets, I answer the following three questions you posed: (a) \textit{For this teaching activity, can you describe your interpretation of the learning process?} (b) \textit{For this teaching activity, how do you incorporate teaching tools and practices?} (c) \textit{For this teaching activity, can you show how the tenets of your teaching philosophy help achieve learning goals you set for your courses?} These questions address the \textit{why} behind my teaching practices, as you suggested.  This exercise has been fruitful, showing me where these tenets work well, and where growth needs to happen.

\subsection{General Approach: How I Teach with Six Easy Pieces (1)-(6)}

Each of the following six tenets is an active learning technique that keeps students engaged in the processes of science.  As I shared in Sec. 2.1 of my prior PEGP\footnote{See Supplemental Material.}, physicists classify students into \textit{majors} and \textit{non-majors}.  The broadest definition of a \textit{major} student is someone who takes physics or engineering courses above the introductory level.  Physics education research (PER) usually covers courses designed for \textit{non-majors}, or introductory courses.  PER provides evidence for \textit{why} active learning techniques are effective.  Most of my students are non-majors, so my teaching philosophy is first focused on them.  When I teach advanced courses, I remix the ingredients into a different recipe.  Advanced physics and engineering courses are built from introductory ones.  Students in advanced courses have already experienced ingredients (1)-(6), and they are ready for something new.  I combine some of the activities (1)-(6) with writing assignments when I teach liberal arts courses and college writing seminar.
\\
\vspace{0.15cm}
I begin introductory course sessions with (1): traditional lecture format.  I often start with a written warm-up exercise drawn from textbook readings assigned 1-2 days prior.  Before giving the solutions, I present the agenda for the session so the students know what to expect.  I then solve the warm-up exercises as a demonstration, and build on them in traditional style.  Next, I proceed to (2): peer-instruction \cite{mazur2013peer}.  I pose conceptual multiple-choice questions to the students, based on content from activity (1).  Students record anonymous answers electronically, and we view the answer distribution.  We discuss our responses as peers in small groups.  I help stuggling students to stimulate their thinking by re-phrasing the question or giving them clues\footnote{I discussed in Sec. 2.2 of my prior PEGP, and Sec. 2 in general, about how this phase of PI also helps to boost equity and inclusion in my courses.  Recall that if one student calls WAT, then we slow down and ensure everyone is ready to proceed.  I can also identify students that appear excluded, and make sure they are included in the discussion personally.  See Supplemental Material for details.}.  The students respond again, and we move forward when a super-majority of the students get it right.  Next, we arrive at (3): PhET simulations.  Physics education technology, or PhET \cite{phet}, consists of PER-based simulation activities.  I provide written activities the students complete while operating the simulations.
\\
\vspace{0.15cm}
I use the second half of our session to conduct a laboratory activity, ingredient (4).  We as a department maintain our lab pedagogy and infrastructure.  The students complete labs that cover the same content presented in activities (1)-(3).  Sometimes tenet (5) is possible, in which tenet (1) lecture content is testable in both a PhET and a lab.  We align theoretical predictions with simulation and empirical data\footnote{Ideally, we would do this every time, but there are not yet PhET simulations for all labs.}.  The students perform activity (6) near the end of the semester, when they propose, build, execute, and present experiments as small groups to the class.  Teaching activities (1)-(6) represent the \textit{average} recipe for my sessions, but I do not repeat the same routine each session.  As a cookbook contains a diverse collection of healthy recipes using common ingredients, I mix these ingredients in new and interesting ways to maintain student engagement.

\subsection{(1) Traditional Lecture Format}

Having reflected on tenet (1), I find three facets \textbf{(i)-(iii)} that make it effective for students. \textbf{(i)} Solving problems on the whiteboard \textbf{displays the components of physics}.  These include variables, estimation, units, functions, algebra/calculus, solutions and graphs, and checking results by examining units, limiting cases, and symmetry.  Facet (i) is derived from the principle of \textit{order} because scientific statements about nature must be ordered using consistent terminology and mathematics. \textbf{(ii)} Traditional lecture gives the students \textbf{memorized examples} that serve as concrete anchor points, a form of \textit{shared meaning}.  \textbf{(iii)} By \textbf{linking basic examples together}, students begin to solve harder problems for more complex systems that may be reduced to sub-systems they already understand (\textit{order}).  I combine the warm-up exercises with my traditional lecture content to ensure it is an active learning strategy.
\\
\vspace{0.15cm}
\textbf{Facet (i): displaying the components of physics problem solving.}
\\
\vspace{0.15cm}
(a) \textit{For this teaching activity, can you describe your interpretation of the learning process?}  As with any subject, students must first understand how to use its components before solving complex problems.  One example that arises every Fall semester is the usage of physical units with variables.  Solving for the speed of an object, some students write ``the speed is 12.''  My question is ``twelve what? Kilometers per hour, feet per second ...'' Students become aware that there are components of physics calculations they do not yet understand when I demonstrate full solutions in traditional style.  Once students master the concrete examples I demonstrate, they can adapt them to similar systems.  Students appreciate this style.  In semesters where I have reduced the amount of tenet (1), students request more of it.
\\
\vspace{0.15cm}
(b) \textit{For this teaching activity, how do you incorporate teaching tools and practices?}  The students engage warm-up exercises on their own before examining the components of the solution.  This activity becomes an anchor point that can be copied and studied.  This technique keeps traditional lecture \textit{active}, as the students compare the steps of their solutions to mine.  I control the \textit{pace} of this content to maximize the learning of a diverse group of students.  This technique also boosts equity, in that we solve the same problem as one group and share the solution.  \textit{Passive} traditional lecture content, in which the instructor simply works a few examples and then moves on, can actually reduce equity.  Students already familiar with the content will excel, whereas students with no prior experience will fall behind.
\\
\vspace{0.15cm}
(c) \textit{For this teaching activity, can you show how the tenets of your teaching philosophy help achieve learning goals you set for your courses?} Two relevant learning goals\footnote{Learning goals are always listed in my course syllabi.  See supplemental material for details.} set for my recent two-semester sequence of algebra-based physics are (A) ``to solve word problems pertaining to physics and mathematics,'' and (B) ``to construct mathematical models of mechanical systems.''  Facet (i) of traditional lecture addresses (A) by breaking a word problem into its components, while translating the words into mathematical statements.  This is especially useful for students who grew up in a bilingual setting.  Facet (i) of traditional teaching addresses (B) because I demonstrate how complex systems can be broken into manageable components.  The \textit{active} nature of the tutorial keeps the students engaged, and we all share the same understanding (\textit{shared meaning}).
\\
\vspace{0.15cm}
\textbf{Facet (ii): memorizable examples.}
\\
\vspace{0.15cm}
(a) \textit{For this teaching activity, can you describe your interpretation of the learning process?}  The learning process for physics includes memorization and repetition, as a new student of music or a new language learns.  Compare the way someone whose first language is English begins to learn Spanish, as opposed to \textit{linguistics}.  Repetition and memorization again play a stronger role in the former.  Introductory physics students have a similar experience: beginning with active learning techniques involving repetition and memorization in problem solving is highly useful.  Students gradually gain the confidence to engage with increasingly complex problems.  Practically, implementing facet (ii) helps students complete their homework and to study for exams.  Their confidence, and the equity of the class, are boosted.  I recall a time I saw my student Deninson Cortez-Cruz stapling together all of my warm up exercises.  I remarked that it looked like his studying was going well, and he replied: ``Oh yes, professor.  This packet is like our Bible.''  I smiled, because I knew that their lab group was using my material the right way, and they all aced the course.
\\
\vspace{0.15cm}
(b) \textit{For this teaching activity, how do you incorporate teaching tools and practices?}  At the beginning of class, I provide a written handout with warm-up exercises that I solve after the students have attempted them on their own for 12-14 minutes.  The students then have a completed document of example problems to study.  I select problems for homework and exams that share some connection to the example base I have built with the students (\textit{shared meaning}).
\\
\vspace{0.15cm}
(c) \textit{For this teaching activity, can you show how the tenets of your teaching philosophy help achieve learning goals you set for your courses?}  Facet (ii) addresses learning goal (A) above by demonstrating specifically how the words of a problem lead to the mathematical actions we take to solve it.  When examples are memorized, problems involving similar logical language with different numbers can be solved with ease\footnote{Sometimes such collections of word problems are called ``homomorphic.''}.  Facet (ii) addresses learning goal (B) above by providing memorizable models of simple systems that can be combined to build mathematical models (see also facet (iii)).  Thus, facet (ii) is both a form of \textit{order} (goal B) and \textit{shared meaning} (goal A) within physics instruction.
\\
\vspace{0.15cm}
\textbf{Facet (iii): linking basic examples together}
\\
\vspace{0.15cm}
(a) \textit{For this teaching activity, can you describe your interpretation of the learning process?} Part of the learning process for technical subjects is that an understanding of complex systems comes after understanding simpler ones.  For example, suppose I have covered friction and momentum transfer.  The students know how to predict the deceleration as a car slides against friction.  They also understand how to predict the velocities of objects that collide.  Suppose I ask them to predict the final velocity of a vehicle \textit{struck by} another that had been sliding. Implementing facet (iii) helps the students see this problem as two simpler, connected ones.  The wrong approach is to assume students \textit{should} be able to solve the the harder problem because they know the two basic concepts.  Rather, \textit{why} facet (iii) is so important is that the students actively learn how to couple simple ideas into complex ones.  This process is analogous to first learning scales in music before learning to compose songs with chord progressions.
\\
\vspace{0.15cm}
(b) \textit{For this teaching activity, how do you incorporate teaching tools and practices?} Towards the end of traditional lecture time in class sessions, I occasionally link two basic examples or concepts together to solve a harder problem for the students.  These examples become possible as the semester progresses.  I have offered optional bonus problems that link three or more basic concepts.
\\
\vspace{0.15cm}
(c) \textit{For this teaching activity, can you show how the tenets of your teaching philosophy help achieve learning goals you set for your courses?}  The purpose of facet (iii) is to address learning goal (B): ``to construct mathematical models of mechanical systems.''  This intentional development of coupling basic ideas into complex ones empowers the students to tackle more complex systems. 

\subsection{(2) Peer-Instruction (PI)}

Having reflected on activity (2), I have identified three facets \textbf{(i)-(iii)} that make PI effective for students.  Students must \textbf{(i) form an argument in their own scientific language} (\textit{order}), mentally ordering concepts in a way that they understand internally.  Students must then \textbf{(ii) practice explaining concepts to others.}  PI naturally connects to \textit{shared meaning} via facet (ii).  Finally, PI enables \textbf{(iii) efficient teaching.}  PI has built-in pace control that keeps students engaged by covering lightly concepts the students understand well, while focusing more intentionally on concepts they find more challenging.  For more detail on PI, see Appendix A, Sec. 7.2 of my prior PEGP report (excerpts included in Supplemental Material).  An interesting resource for cognitive research into the benefits of collective reasoning such as PI, see \cite{infotopia}.
\\
\vspace{0.15cm}
\textbf{Facet (i): forming an argument in one's own scientific language.}
\\
\vspace{0.15cm}
(a) \textit{For this teaching activity, can you describe your interpretation of the learning process?}  I hear the following every Fall semester: ``I understand the formulas and concepts in physics.  I just need help `translating' the words of the problem into the formulas.''  Facet (i) is vital for physics instruction because students discover that they must practice this ``translation.''  Causing a student to absorb abstract ideas in a way that they understand, and to confirm that understanding with peers, is critical to the learning process.  PI relies on short, conceptual word problems posed in a multiple choice format the students must answer by themselves by thinking conceptually in 1-2 minutes.  Practicing facet (i) of PI causes students to confront their lack of understanding of a concept, and to establish order over it in their mind.
\\
\vspace{0.15cm}
(b) \textit{For this teaching activity, how do you incorporate teaching tools and practices?}  Since PI is such a well-studied technique in PER, I refer the reader to Appendix A, Sec. 7.2 of my prior PEGP, and the following book, website, and conference resources \cite{mazur2013peer,infotopia,PhysPort,AAPTPI}.
\\
\vspace{0.15cm}
(c) \textit{For this teaching activity, can you show how the tenets of your teaching philosophy help achieve learning goals you set for your courses?}  The relevant learning goal (taken from my introductory course syllabi) is (C): ``to apply logical thinking to conceptually-posed physics problems.''  Students \textit{order} the concepts and problems in their minds in a way that is the most useful for them.  Facet (i) also serves goals (A) and (B) because it enables more intricate and meaningful problem-solving.  Physics instructors often encounter introductory students mis-applying an equation to solve a problem because it is the only relevant formula they understand.  PI facet (i) helps break this habit by practicing conceptual thinking.
\\
\vspace{0.15cm}
\textbf{Facet (ii): practice explaining concepts to others} 
\\
\vspace{0.15cm}
(a) \textit{For this teaching activity, can you describe your interpretation of the learning process?} Our own conceptual understanding of a concept must be compared to that of others (\textit{shared meaning}), and physical reality (\textit{order}).  The learning process for physics must include the chance to revise one's conceptual understanding by fixing faulty logic through discussion with peers.  Facet (ii) of PI also respects our psychology: we are more likely to absorb the thinking of a trusted peer who uses language we already understand.
\\
\vspace{0.15cm}
(b) \textit{For this teaching activity, how do you incorporate teaching tools and practices?}  See Appendix A, Sec. 7.2 of my prior PEGP, and the following book, website, and conference resource \cite{mazur2013peer,PhysPort,AAPTPI}.  I add here that, during the discussion (facet (ii)) phase of PI, I move through my class, focusing my attention on struggling students to help boost class equity.  I listen to my students and share clues with them as their peer.  This technique requires a pre-established relationship of trust with the students, so early in the semester I focus on building this relationship.
\\
\vspace{0.15cm}
(c) \textit{For this teaching activity, can you show how the tenets of your teaching philosophy help achieve learning goals you set for your courses?}  The purpose of facet (ii) of PI is literally to develop a \textit{shared meaning} between peers and instructors.  Learning goal (C) is directly addressed, and learning goals (A) and (B) are augmented.  Students begin to think conceptually rather than guessing the formula and plugging in numbers.  As an example, consider a student attempting to predict the final velocity of a falling object by plugging in the distance traveled divided by the time duration.  The student thinks it is the right approach, because the formula produces a velocity.  It is wrong, however, because the object is accelerating (requiring a different equation).  During group discussion, a lab partner might explain why the formula doesn't apply:  ``I don't think you can use that one. This one's more like the accelerating car where you have to times the acceleration by the delta-t.''  Peers develop a shared understanding that leads them away from confusion and towards better problem-solving skill.
\\
\vspace{0.15cm}
\textbf{Facet (iii): teaching efficiently}
\\
\vspace{0.15cm}
Philosophically, I think of facet (iii) of PI as an auxiliary benefit that helps the students.  When a super-majority of students answer a conceptual question correctly in the first round, we move forward.  When this does not happen, we stop and have group discussions.  This forms a natural pace control.  To boost equity, as I've shared in prior PEGPs, I've built in the WAT function.  Students can hit a special button called WAT\footnote{\url{https://knowyourmeme.com/memes/wat}.  For real, click this meme.} that causes me to slow down and give an additional explanation, regardless of how many have solved the problem correctly.  Students know to hit WAT if they are lost, and they know it's OK to do that.

\subsection{(3) PhET Simulations}

I have identified three facets (i)-(iii) that make PhET simulations useful for my students. \textbf{(i) PhETs foster pattern extraction from physical systems} (\textit{order}).  The learning process involves pattern recognition, and PhETs provide a space for students to tinker with a system until they recognize the pattern of its behavior. \textbf{(ii)} PhETs provide an avenue for students \textbf{to construct graphical results.} Data can be generated intuitively with PhET simulations without stressing over constructing the lab apparatus, and students can focus on graphing results in a way the lab group understands (\textit{shared meaning}).  The third facet is philosophically practical: \textbf{(iii)} PhET simulations allow us \textbf{to study physical systems we cannot build,} like the solar system in the study of gravity.
\\
\vspace{0.15cm}
\textbf{Facet (i): extracting patterns from physical systems}
\\
\vspace{0.15cm}
(a) \textit{For this teaching activity, can you describe your interpretation of the learning process?} Part of the learning process in the physical sciences is to recognize patterns in systems (\textit{order}).  PhET simulations allow students to experiment in the absence of statistical error.  For example, consider a PhET simulation of a DC circuit powered by a 5 Volt battery.  Students can measure instantaneously the current flowing to various devices using a tool.  To graph the data and extract Ohm's law (a linear relationship between voltage and current), they simply tune the battery voltage.  The corresponding lab activity requires the students to learn and operate a DC power supply and digital voltmeter.  These skills are part of the course as well, but PhET simulations help students concentrate on \textit{just the pattern recognition.}
\\
\vspace{0.15cm}
(b) \textit{For this teaching activity, how do you incorporate teaching tools and practices?}  I incorporate PhET simulations as group activities completed by lab groups within my integrated lecture-lab format.  I give the students a brief tutorial on the projector screen, and I provide them with a written worksheet.  Working together, they construct a system using the digital components in the PhET, and graph data drawn from the model on the worksheet.  I construct my own version on the large screen, and we all compare results (\textit{shared meaning}).  PhET simulations are designed with extensive PER.  The measurement tools and other systems have obvious controls that resemble the real versions in the lab.  Once the students have completed a PhET simulation activity, doing the real experiment on the lab bench is more straightforward because they recognize the pattern.  This strategy reduces stress and confusion in tenet (4) - laboratory activities.
\\
\vspace{0.15cm}
(c) \textit{For this teaching activity, can you show how the tenets of your teaching philosophy help achieve learning goals you
set for your courses?}  Students are more likely to solve problems correctly using an equation if they have verified that equation experimentally (indirect benefit to (A) and (B)).  Viewed as a pattern of physical behavior, physics equations once extracted from a system can be used to model similar systems by analogy.  Another learning goal is reached by the \textit{order} achieved through pattern recognition is (D) ``to practice scientific experimentation, data analysis, and reporting of results.''  Students must order their thinking by analyzing data from the PhET and constructing shared graphical results in lab groups.
\\
\vspace{0.15cm}
\textbf{Facet (ii): constructing graphical results}
\\
\vspace{0.15cm}
(a) \textit{For this teaching activity, can you describe your interpretation of the learning process?} Physical science requires us to convince others of the validity of a result (\textit{shared meaning}), so the students must learn to share results graphically.  Facet (ii) of PhET activities is about practicing visual representation of experimental patterns derived from simulations or experiments, such that another person can extract the same pattern.  The students learn to pay attention to the details in the graphs they create that allow others to interpret them.
\\
\vspace{0.15cm}
(b) \textit{For this teaching activity, how do you incorporate teaching tools and practices?} Philosophically, I must strike a balance for facet (ii) of PhET activities.  On one hand, I must provide enough scaffolding in the worksheet so that the students know what graphical result I expect.  On the other hand, I want the students to practice constructing the elements of that graph on their own.  These elements include labeled axes, legends, proper numerical upper and lower limits, and data points that include any statistical errors.  We build these skills over the course of the first semester, and our graphics gradually become more sophisticated.  My expectation for the self-designed final projects (tenet (6)) is that all results communicated graphically can be interpreted without ambiguity. 
\\
\vspace{0.15cm}
(c) \textit{For this teaching activity, can you show how the tenets of your teaching philosophy help achieve learning goals you
set for your courses?} Learning goal is (D) ``to practice scientific experimentation, data analysis, and reporting of results.''  The work we do with PhET in facet (ii) is all about reporting of results such that someone else could reasonably understand them (\textit{shared meaning}).  We use PhET activities to build good habits that later benefit tenets (4) and (6), lab activities and final projects.  Science itself goes awry very fast the scientist does not have these habits.  My students practice this important skill, a skill they will use immediately in the scientific and professional communities they join after they graduate.
\\
\vspace{0.15cm}
\textbf{Facet (iii): studying systems we cannot build}
\\
\vspace{0.15cm}
A practical benefit of using PhET simulations is that we cannot always construct systems we would like to study.  Consider the relationship between the pressure, volume, and temperature of an ideal gas.  The \textit{kinetic theory of ideal gases} tells us that the reason these macroscopic quantities are related is because ideal gases are made of molecules that all have kinetic energy and momentum.  Repeating ideal gas experiments done in chemistry courses would not demonstrate how these phenomena are caused by the \textit{molecules,} and we cannot see molecules on the lab bench.  A simulation is more illuminating because we \textit{can} see the molecules in action, and we can tune properties like temperature and observe the corresponding molecular behavior.

\subsection{(4) Laboratory Activities}

I reflected on how we conduct laboratory activities, and three facets have emerged.  In a book we are reading for my INTD100 section, \textit{The Scientific Attitude} \cite{scientific_attitude}, the author draws the \textit{line of demarcation} between science and non-science with two ideas.  First, \textit{science cares about data,} collected from a controlled experiment.  Second, \textit{science is willing to change theories based on new data.}  Facet \textbf{(i)} of tenet (4) is to show the students that \textbf{science cares about data} (\textit{shared meaning}).  Facet \textbf{(ii)} is about \textbf{learning to keep an experiment under control} (\textit{order}), generating data that can be interpreted by establishing cause and effect.  Finally, no experiment has infinite precision, so facet \textbf{(iii)} is about \textbf{error analysis} (\textit{order}).
\\
\vspace{0.15cm}
\textbf{Facet (i): science cares about data}
\\
\vspace{0.15cm}
(a) \textit{For this teaching activity, can you describe your interpretation of the learning process?}  It might seem obvious that physical science courses need labs.  Some students place too much trust in the predictions of theoretical physics, and some are accustomed to being told what to think.  Experiments provide the students with hard proof that what we are teaching them is real.  The students see on a deeper level that physics is built upon incontrovertible evidence.
\\
\vspace{0.15cm}
(b) \textit{For this teaching activity, how do you incorporate teaching tools and practices?}  Constructing a good lab activity for physics courses is an art form\footnote{I am grateful to Prof. Seamus Lagan for taking time to help me grow in this area.}.  Design an apparatus that is too complex, and the students will learn nothing.  To simple, and the apparatus will not tell the students anything.  As a department, we maintain a library of proven labs based on our experience and PER.  It is vital that we maintain the integrated lecture-lab format for our courses.  The lab activity that provides the proof of a theorem immediately follows the traditional lecture and PI that introduced it.  The students are given time and space to apply what they've just learned, and their efforts are rewarded with proof that what they've learned works.
\\
\vspace{0.15cm}
(c) \textit{For this teaching activity, can you show how the tenets of your teaching philosophy help achieve learning goals you set for your courses?}  Goal (D) (taken from my syllabi) is ``to practice scientific experimentation, data analysis, and reporting of results.''  Lab groups (2-4 people) work together and engage in a form of \textit{shared meaning}.  When proctoring these lab activities, I usually observe the following behaviors in my students: one student explaining to another why something is not working, how to compute the result from the data, or even waving their hands to visualize \textit{why} they think the result makes sense.  On the lab worksheets I provide, I require the students to create graphs, calculate numerical results from raw data, and to report results.  Thus, we reach goal (D) as a team.  Periodically, we compare results group by group or with my version of the setup at my desk\footnote{I'm grateful to Prof. Glenn Piner for this suggestion.}.
\\
\vspace{0.15cm}
\textbf{Facet (ii): learning to keep an experiment under control}
\\
\vspace{0.15cm}
(a) \textit{For this teaching activity, can you describe your interpretation of the learning process?}  Sometimes people use the phrase ``if this then that'' (ITTT) to represent causal connections.  For some lab activities, we can give the students a knob, lever, or action that they can use to tune the output data.  One example is my magnetic induction lab, in which a magnet on a spring bounces into and out of a coil of wire.  Students view the voltage induced in the coil with an oscilloscope.  \textit{Faraday's Law} states that the rate of change of the magnetic field in the coil determines the amount of voltage in the coil.  \textit{If} the students tune the velocity of the magnet by compressing the spring more, THEN the size of the oscilloscope signal increases.  When the students experience ITTT they see equations like Faraday's Law not as abstract equations, but simple descriptions of ITTT.
\\
\vspace{0.15cm}
(b) \textit{For this teaching activity, how do you incorporate teaching tools and practices?}  See facet (i), question (b).  All of the labs we perform are controlled experiments, so facet (ii) is built in to the pedagogy.
\\
\vspace{0.15cm}
(c) \textit{For this teaching activity, can you show how the tenets of your teaching philosophy help achieve learning goals you set for your courses?}  Facet (ii) is derived from \textit{order.}  We design our lab activities such that the students change only one input at a time to examine the effect on the outcome (ITTT).  Students quickly learn that if several inputs are changing simultaneously, the data will be useless.  So facet (ii) serves goal (D) (``to practice scientific experimentation, data analysis, and reporting of results'') in the sense that it enables useful data analysis.  Once students draw a connection between ITTT and useful data analysis, they design controlled experiments for their final projects (see tenet (6) below).
\\
\vspace{0.15cm}
\textbf{Facet (iii): error analysis}
\\
\vspace{0.15cm}
(a) \textit{For this teaching activity, can you describe your interpretation of the learning process?}  Part of the learning process for physics students is to notice that experimental results rarely match theoretical predictions with perfect precision and accuracy.  Learning to improve both precision and accuracy is vital for student success.  Introductory physics students usually attribute disagreement between data and theory to ``human error,'' and it takes time for them to learn that statistical error occurs naturally.  I have created special lab activities focusing on statistical error\footnote{See supplemental material.}, and I place them in the middle of the semester after the students are more familiar with lab technique. Because my introductory courses are year-long sequences, I choose to build facet (iii) slowly.  This gives students a chance to learn the calculus that models error propagation first.  For students who do not take calculus, we still must define and understand statistical concepts before discussing error analysis.
\\
\vspace{0.15cm}
(b) \textit{For this teaching activity, how do you incorporate teaching tools and practices?}  I create specific lab activities that teach both physical concepts and focus on statistical error.  We practice error propagation, when multiple measurements with error are combined in a formula to produce a result with compounded error.  We practice comparing results across lab groups to show that error is indeed statistical, and not a function of who is performing the measurement.  Learning error propagation through laboratory work is more applicable than turning the subject into a mathematical or statistical exercise, though that sometimes is necessary if the students are not understanding it in the lab.  Many of my students do not experience proper error analysis until they reach college.  Thus, I create highly scaffolded activities regarding error analysis and error propagation to give the students time and space to learn it.
\\
\vspace{0.15cm}
(c) \textit{For this teaching activity, can you show how the tenets of your teaching philosophy help achieve learning goals you set for your courses?}  The principle of \textit{order} with respect to facet (iii) is critical to achieving goal (D).  My students learn that we can achieve \textit{order} out of the apparent chaos of analyzing raw data.  The students learn results can confirm theoretical predictions despite the presence of statistical error.

\subsection{(5) Synergies}

My reflections give one facet \textbf{(i)} that makes the synergy between traditional lecture, PI, and laboratory activities useful for students: \textbf{solidifcation.}  I focus below on the example of DC circuits.  We are equipped with DC circuit tools in our labs, and an excellent DC circuits PhET is available\footnote{\url{https://phet.colorado.edu/en/simulations/circuit-construction-kit-dc}.}.  I give traditional lecture content about circuits, have the students discuss circuits via PI, and follow that with an integrated PhET and lab activity.  \textbf{Solidification} occurs because the students observe the match between theoretical calculations, simulation, and the lab activity within the class session.  Experiencing this match gives the students a sense of satisfaction.
\\
\vspace{0.15cm}
\textbf{Facet (i): solidification}
\\
\vspace{0.15cm}
(a) \textit{For this teaching activity, can you describe your interpretation of the learning process?}  Students are fully prepared to apply and remix the ideas of physics in their own projects when the ideas solidify in their minds.  The ideas feel right on a deeper level if they have correctly demonstrated them as theoretical predictions, ideally simulated, and tested practically in the lab.  After solidification, their confidence is such that they stop questioning their understanding of a concept and start to use it in their designs.  Consider the example of DC circuits.  Using the ideas that energy and total charge are constant within a circuit, we solve systems of equations that predict currents through multiple devices connected to a battery.  In the PhET, however, current and charge are animated, and the algebra is illustrated.  The students replicate the virtual circuit in the lab, and show that the volts and amps in real life match the simulation.  This confirms their algebra, or exposes algebraic errors.  Apprehension fades to confidence when they see that they got it all to match.
\\
\vspace{0.15cm}
(b) \textit{For this teaching activity, how do you incorporate teaching tools and practices?}  See tenet (4), question (b).  I include synergy between PhET and lab whenever possible, but the limiting factor is the availability of PhET simulations.
\\
\vspace{0.15cm}
(c) \textit{For this teaching activity, can you show how the tenets of your teaching philosophy help achieve learning goals you set for your courses?}  The synergy bonus my students derive from solidification is an important part of \textit{shared meaning.}  As a group, the students construct an apparatus for performing a measurement, match it to simulation, and understand it mathematically.  They build a consensus as to \textit{why} it all works together, and their consensus is confirmed when I gather and display the results from each group.  All parts of learning goal (D) are enhanced by this synergy.  I cannot predict \textit{when} the ``light bulb'' will activate for a given student.  By including the synergy, I am maximizing the number of light bulbs activated in my class session by approaching a topic from diverse perspectives.

\subsection{(6) Student-Designed Final Projects}

My reflections give two facets \textbf{(i)-(ii)} that makes the student-designed final projects useful for students.  Facet (i) is about \textbf{individual creativity}.  The creativity required for a good final project is related to developing both order and shared meaning within a lab group in my courses.  Facet \textbf{(ii)} is about developing a shared meaning with an audience: \textbf{communication of abstract arguments, numerical results, and graphical results.}
\\
\vspace{0.15cm}
\textbf{Facet (i): individual creativity}
\\
\vspace{0.15cm}
(a) \textit{For this teaching activity, can you describe your interpretation of the learning process?} In lieu of a final exam, my students propose, build, execute, and present their own physics project.  During the semester, we practice lab skills and theoretical calculations.  The student-designed projects meet a final learning need left unaddressed by our other activities.  Schemes for brilliant confirmations of physical laws are often concocted with a stroke of creativity.  The students beam when they get a chance to combine their creativity with physics in the final project.  They finally have a chance to express themselves scientifically.  These summative projects allow them to apply what they have learned to a topic about which they are curious.  Their reward is knowing that they can use physics to understand the world. 
\\
\vspace{0.15cm}
(b) \textit{For this teaching activity, how do you incorporate teaching tools and practices?}  I first invite proposals from lab groups that outline what they will measure.  Proposals must include diagrams and lists of parts they will need, including our lab equipment.  Members of lab groups are fully responsible for coordinating the experiment with each other, and creating the final presentation.  Though I actively debug experiments with students, I am cautious in that phase. I provide just enough guidance so they can move forward, while not doing the project for them.
\\
\vspace{0.15cm}
(c) \textit{For this teaching activity, can you show how the tenets of your teaching philosophy help achieve learning goals you set for your courses?} Facet (i) is about creating both order and shared meaning.  Once the students agree on an idea, they take ownership of it to make it a reality.  In our past communications, I have shown you how intricate and exciting these final projects can become.  Learning goal (E), taken from my syllabi, is ``to practice written and oral expression of scientifically technical ideas.''  Asking the students to own their idea and make it reality from inception to presentation is the best way I can imagine to address goal (E).
\\
\vspace{0.15cm}
(d) \textit{For this teaching activity, can you focus on the ``why'' of specific teaching decisions?''}  I can clearly observe how my students shine in this learning activity.  The reason why I include the creative aspect (facet (i)), as opposed to deciding what their project will be, is that they are creatively energized to do science.
\\
\vspace{0.15cm}
\textbf{Facet (ii): communication of abstract arguments, numerical results, and graphical results}
\\
\vspace{0.15cm}
(a) \textit{For this teaching activity, can you describe your interpretation of the learning process?}  Learning to communicate the results of the final projects is just as much art as it is science.  The best way to learn it is to do it and see if people understand you.  Students draw upon experiences in tenet (3), PhET simulations, to help with creating meaningful graphs of data that can be interpreted by the class.  Students can learn by imitating or remixing techniques in past presentations, so I give them example presentations created by me and my students from past courses.
\\
\vspace{0.15cm}
(b) \textit{For this teaching activity, how do you incorporate teaching tools and practices?}  The final presentation can be given live in class, or developed using digital storytelling techniques\footnote{See Secs. 2.1 and 2.2 of my prior PEGP regarding WeVideo.  On my syllabi, the live or WeVideo options are called option A and B.}.  The results are presented to the class and we get a chance to ask the lab group questions.  Given the group size (2-4 students) and the average number of students in my classes ($\approx 25$), we limit the presentations to 15 minutes so that the presentations take 1-2 days.  I review the presentations before they are given to the class, to help the students polish them and eliminate   After some experimentation, I have settled on just one summative project at the end of the semester as it is the most practical plan.
\\
\vspace{0.15cm}
(c) \textit{For this teaching activity, can you show how the tenets of your teaching philosophy help achieve learning goals you set for your courses?}  Facet (ii) is about developing a \textit{shared meaning} between myself and the presenting group, and between the presenting group and their peers.  Showing your results to others is fundamental to science, which is why I've included learning goal (E) in my syllabi.  Facet (ii) of the student-led projects gives the students time and space to practice learning goal (E) (``to practice written and oral expression of scientifically technical ideas'').
\\
\vspace{0.15cm}
(d) \textit{For this teaching activity, can you focus on the ``why'' of specific teaching decisions?''}  See (d) of facet (i).

\subsection{Outlook}

In our last communication, you recognized that I have a plan that appears to be working well for the students, and that my course evaluations reflect that.  As I've been in my room reflecting on why I teach the way I do, I've found that solving this rubix cube helps me to concentrate.  I laughed to myself when I realized that asking me to write about teaching philosophy is like asking me to solve this rubix cube.  I know what the final result should be: students successfully learning and doing science.  I'm less clear on how to lay out the pieces correctly.  When I've asked colleagues over the years to describe a teaching philosophy, I hear many facets.  Some say to express your values.  Others say to describe practices and pedagogy.  Some say that a personal narrative is important.  I have tried to answer your questions in a way that is, if nothing else, easy to understand.  I hope this sheds light on my overall philosophy.
\\
\vspace{0.15cm}
One recent development in my teaching is the inclusion of \textbf{online laboratory activities}.  We began using services like Pivot Interactives during quarantine, when we were barred from doing any in-person teaching.  Online lab activities give the user a chance to play a video of a constructed apparatus in operation.  The user can use on-screen tools to collect data from the video.  Built into the lab web page are tools for creating graphical results from this data.  Data analysis, question prompts regarding conclusions we draw from the results, and grading are all seamlessly integrated.  It is tempting to use these online lab services going forward.  I am of two minds about this strategy.
\\
\vspace{0.15cm}
Online lab activities will not replace activity tenet (4), but they have interesting facets that make them useful for students.  Online labs remove the confusion inherent to constructing lab apparatuses, which makes them more like tenet (3).  In fact, online labs and PhETs share facets (i)-(ii) of tenet (3).  They are not simulations, though, because the apparatuses are real but filmed.  The data is collected from a physical system subject to statistical error.  The potential for statistical error makes online lab activities more like tenet (4).  The systems being filmed in the online labs are systems we \textit{can} build in our labs, but sometimes we have not built them yet.
\\
\vspace{0.15cm}
As with all teaching decisions, the question will be resolved by determining how well the learning activity serves the students.  If the students' experiences of facets (i)-(iii) of tenet (4) are not diminished by including online labs, then we will probably continue to include them.  However, if the students view them like PhET simulations, then the line between simulation and reality would be blurred.  This is not good for science.  In any event, I hope it is clear that I have a conscientious teaching philosophy that continues to serve students well at Whittier College, and I always welcome your insight and wisdom.

\end{document}
