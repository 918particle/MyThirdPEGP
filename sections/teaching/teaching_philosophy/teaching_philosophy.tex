\documentclass[../../../main.tex]{subfiles}

\begin{document}

The following is a reflection on the six main teaching activities I use in the physical sciences.  I also apply them, as appropriate, to my liberal arts courses and college writing seminar sections.  Each activity is derived from the over-arching principles of \textit{order} and \textit{shared meaning.}  I do not cover those principles in detail, since I have already shared that in Sec. 2.1 of my prior PEGP.  For each of my six main teaching activities, I answer the following four questions you posed:
\begin{itemize}
\item (a) \textit{For this teaching activity, can you describe your interpretation of the learning process?}
\item (b) \textit{For this teaching activity, how do you incorporate teaching tools and practices?}
\item (c) \textit{For this teaching activity, can you show how the tenets of your teaching philosophy help achieve learning goals you set for your courses?}
\item (d) \textit{For this teaching activity, can you focus on the ``why'' of specific teaching decisions?}
\end{itemize} 
After completing this exercise, I reflect potential changes going forward, and what seems to work well.  One interesting shift that has taken place within our department is the use of \textit{online laboratory activities.}  I reflect on \textit{why} we have included this practice in recent years.  I review the learning focuses mentioned in Sec. 7 (Appendix A) of my prior PEGP, and reflect in Sec. xxx how these drive my course content selection.

\subsection{General Approach: How I Teach with Six Easy Pieces (1)-(6)}

My teaching philosophy may be broken into the six main teaching activities, or ingredients, below.  Each of them is an active learning technique, keeping students engaged in the processes of science.  As I shared in Sec. 2.1 of my prior PEGP, physicists classify students into \textit{majors} and \textit{non-majors}.  The broadest definition of a \textit{major} student is someone who takes physics or engineering courses above the introductory level.  Physics education research (PER) usually covers courses designed for \textit{non-majors}, or introductory courses.  PER provides evidence for \textit{why} active learning techniques are effective.  Most students who take physics and engineering courses at Whittier College are non-majors, so my teaching philosophy is focused on active learning techniques for non-majors.  When I teach advanced courses, I remix the ingredients into a different recipe.  Advanced physics and engineering courses are built from introductory ones.  Students in advanced courses have already experienced ingredients (1)-(6), and they are ready for something new.  I combine some of the activities (1)-(6) with writing assignments when I teach liberal arts courses and college writing seminar.
\\
\vspace{0.25cm}
I begin introductory course sessions with (1): traditional lecture format.  I start with a warm-up exercise drawn from textbook readings assigned 1-2 days prior.  Before giving the solutions, I present the agenda for the session so the students know what to expect.  I then solve the warm-ups as on the whiteboard, and build on them with more intricate examples and proofs of theorems.  Next, I usually proceed to (2): peer-instruction \cite{mazur2013peer}.  I pose conceptual multiple-choice questions to the students, based on content from activity (1).  Students record anonymous answers electronically, and we view the answer distribution.  We discuss our responses as peers in small groups, and I help stuggling students to stimulate their thinking by re-phrasing the question or giving them clues\footnote{I discussed in Sec. 2.2 of my prior PEGP, and Sec. 2 in general, about how this phase of PI also helps to boost equity and inclusion in my courses.  Recall that if one student calls WAT, then we slow down and ensure everyone is ready to proceed.}.  The students respond again, and we move forward when a super-majority of the students get it right.  Next, we arrive at (3): PhET simulations.  The students find it useful to simulate complex systems.  Physics education technology, or PhET \cite{phet}, consists of HTML5 applications designed with PER.  I provide written activities the students complete while operating the simulation.
\\
\vspace{0.25cm}
I use the second half of our session to conduct a laboratory activity, ingredient (4).  We work as a team in our department to maintain consistent lab pedagogy and equipment.  The students complete labs that cover the same content presented in activities (1)-(3).  Sometimes activity (5) becomes possible, in which we cover traditional lecture content that is testable in both a PhET and a lab.  We align theoretical predictions with simulation and lab experiment\footnote{Ideally, we would do this every time, but there are not yet PhET simulations for all labs.}.  The students perform activity (6) near the end of the semester, when they propose, build, execute, and present experiments as small groups to the class.  Though teaching activities (1)-(6) represent the \textit{average} recipe for my sessions, I do not repeat the exact same routine each day.  As a cookbook contains a diverse collection of healthy recipes using common ingredients, I mix these ingredients in new and interesting ways to maintain student engagement.

\subsection{(1) Traditional Lecture Format}

Having reflected on activity (1), I find three facets \textbf{(i)-(iii)} that make it useful for students. \textbf{(i)} Solving problems on the whiteboard \textbf{displays the components of physics} in step-by-step fashion.  These include variables, estimation, units, functions, algebra/calculus, solutions and graphs, and checking results by examining units, limiting cases, and symmetry.  Facet (i) is derived from the principle of \textit{order} because scientific statements about nature must be ordered using consistent terminology and mathematics. \textbf{(ii)} Traditional lecture gives the students \textbf{memorizable examples} that serve as concrete anchor points, a form of \textit{shared meaning}.  Anchor points assure our mutual understanding of a system, and students use them to solve new problems with similar traits.  \textbf{(iii)} By \textbf{linking basic examples together}, students begin to solve harder problems for more complex systems that may be reduced to sub-systems they already understand.  Students learn to use the intrinsic \textit{order} of physics to model complex systems.  We know from PER how active learning strategies benefit students, but students always request some traditional lecture content.  Philosophically, I make the facets of my traditional lecture as active as possible while responding to this request.
\\
\vspace{0.25cm}
\textbf{Facet (i): displaying the components of physics problem solving.}
\\
\vspace{0.25cm}
(a) \textit{For this teaching activity, can you describe your interpretation of the learning process?}  As with any subject, students must first understand how to use its components before solving problems or creating something new using those components as one.  One example that arises every Fall semester in my courses is the usage of physical units with variables.  Solving for the final speed of a system, some students inevitably write ``the final speed is 12.''  My next question is always ``twelve what? Kilometers per hour, feet per second ...'' Students become aware that there are components of physics calculations they do not yet understand when I break a problem into its components.  Once students master concrete examples by following my lead, they can adapt them to model similar systems.
\\
\vspace{0.25cm}
(b) \textit{For this teaching activity, how do you incorporate teaching tools and practices?}  Solving a problem with nothing but a piece of chalk in front of the group will always have a place in my teaching.  The students engage on their own with a written warm-up before examining the components of the solution.  This activity becomes an anchor point that can be copied and studied.  This technique keeps traditional lecture \textit{active}, as the student evaluate the steps of their solution against mine.  I control the \textit{pace} of this content to maximize the learning of a diverse group of students.  This technique also boosts equity, in that we solve the same problem as one group and share the solution on a warm-up form that the students study with each other\footnote{\textit{Passive} traditional lecture content, in which the instructor simply works a few examples and then moves on, can actually reduce equity.  Students already familiar with the content will excel, whereas students with no prior experience will fall behind.}.
\\
\vspace{0.25cm}
(c) \textit{For this teaching activity, can you show how the tenets of your teaching philosophy help achieve learning goals you set for your courses?} Two relevant learning goals\footnote{Learning goals are always listed in my course syllabi.  See supplemental material for details.} set for my recent two-semester sequence of algebra-based physics are (A) ``to solve word problems pertaining to physics and mathematics,'' and (B) ``to construct mathematical models of mechanical systems.''  Facet (i) of traditional lecture addresses (A) by breaking a word problem into its components, while translating the words into mathematical statements.  As a professor at a Title V HSI like Whittier College, this is especially useful for students who grew up in a bilingual setting.  Facet (i) of traditional teaching addresses (B) because I demonstrate how complex systems can be broken into manageable components.  The \textit{active} nature of the tutorial keeps the students engaged, and we all share the same understanding of the problem (\textit{shared meaning}).
\\
\vspace{0.25cm}
(d) \textit{For this teaching activity, can you focus on the ``why'' of specific teaching decisions?''}  The most basic reason that comes to mind is that the students ask for it.  It's concrete, and feels right to them.  Even if more modern PER methods are shown to be effective, I have learned that a course with too little of facet (i) is disorienting to learners.
\\
\vspace{0.25cm}
\textbf{Facet (ii): memorizable examples.}
\\
\vspace{0.25cm}
(a) \textit{For this teaching activity, can you describe your interpretation of the learning process?}  The learning process for physics includes memorization and repetition, as a new student of music or a new language learns.  Compare the way someone who speaks English begins to learn Spanish, as opposed to \textit{linguistics}.  Memorization and repetition plays a stronger role in the former.  Compare someone learning to play the guitar to a student of \textit{musical theory.}  Repetition and memorization again play a stronger role in the former.  Introductory physics students have a similar experience: beginning with active learning techniques involving repetition and memorization are highly useful.  Students eventually gain enough confidence to engage with more complex physical systems.
\\
\vspace{0.25cm}
(b) \textit{For this teaching activity, how do you incorporate teaching tools and practices?}  At a minimum, once per class session, I provide a written handout (the warm up problems) that I solve after the students have attempted them on their own.  The students then have completed document of example problems to study.  I deliberately select problems for homework and exams that share some connection to the example base I have built with the students (\textit{shared meaning}).
\\
\vspace{0.25cm}
(c) \textit{For this teaching activity, can you show how the tenets of your teaching philosophy help achieve learning goals you set for your courses?}  Facet (ii) addresses learning goal (A) above by demonstrating specifically how the words of a problem lead to the actions we take to solve it with mathematical reasoning.  When examples are memorized, problems involving similar logical language with different numbers can be solved with ease\footnote{Sometimes such collections of word problems are called ``homomorphic.''}.  Facet (ii) addresses learning goal (B) by providing memorizable models of simple systems that can be combined to build mathematical models (see also facet (iii)).  Thus, facet (ii) is both a form of \textit{order} (goal B) and \textit{shared meaning} (goal A) within physics instruction.
\\
\vspace{0.25cm}
(d) \textit{For this teaching activity, can you focus on the ``why'' of specific teaching decisions?''} Practically, these memorizable examples help students complete their homework and to study for exams.  Their confidence, and the equity of the class, are boosted.  I recall a time I saw my student Deninson Cortez-Cruz stapling together all of my warm up problems.  I remarked that it looked like his studying was going well, and he replied: ``Oh yes, professor.  This packet is like our Bible.''  I smiled, because I knew that he was going to use those resources to help his lab group ace the class, and they all did.
\\
\vspace{0.25cm}
\textbf{Facet (iii): linking basic examples together}
\\
\vspace{0.25cm}
(a) \textit{For this teaching activity, can you describe your interpretation of the learning process?} Part of the learning process for technical subjects is that an understanding of complex systems comes after understanding simpler ones.  For example, suppose I have already covered basic examples of friction and momentum transfer.  The students know how to predict the deceleration as a car slides against friction.  They also understand how to predict the velocities of objects that collide.  If I ask them to predict the final velocity of a vehicle \textit{struck by} another that had been sliding, facet (iii) is \textit{why} they can solve this problem.  The wrong approach would be to teach them about friction, then momentum transfer, and conclude that they should be able to solve the complex problem because they know the basic concepts.  Rather, \textit{why} facet (iii) is so important is that the students actively learn how to couple simple ideas into complex ones.
\\
\vspace{0.25cm}
(b) \textit{For this teaching activity, how do you incorporate teaching tools and practices?} Towards the end of traditional lecture time in class sessions, I occasionally link two basic examples or concepts together to solve a harder problem for the students.  When appropriate, I assign \textit{challenge problems} for bonus points that link together three or more basic concepts.
\\
\vspace{0.25cm}
(c) \textit{For this teaching activity, can you show how the tenets of your teaching philosophy help achieve learning goals you set for your courses?}  The purpose of facet (iii) is to address learning goal (B): ``to construct mathematical models of mechanical systems.''  I take the time to practice facet (iii) with my students, to develop further in their minds the \textit{order} within physics.  This development takes time, but empowers the students to tackle more complex systems. 
\\
\vspace{0.25cm}
(d) \textit{For this teaching activity, can you focus on the ``why'' of specific teaching decisions?''}  See (c).  I would compare this part of the process to a piano student first learning scales and chords before beginning to compose songs.  Once students gain confidence with the \textit{order} of physics, they begin to trust it and their problem-solving ability grows stronger.

\subsection{(2) Peer-Instruction (PI)}

Having reflected on activity (2), I have identified three facets \textbf{(i)-(iii)} that make PI useful for students.  Students must \textbf{(i) form an argument in their own scientific language}, which relates to the principle of \textit{order} because they must mentally order concepts in a way that they understand internally.  Students must then \textbf{(ii) practice explaining concepts to others.}  PI naturally connects to \textit{shared meaning} via facet (ii).  Finally, peer-instruction enables \textbf{(iii) teaching efficiently.}  PI has built-in pace control that keeps students engaged by covering lightly concepts the students understand well, while focusing more intentionally on concepts they find more challenging.
\\
\vspace{0.25cm}
\textbf{Facet (i): forming an argument in one's own scientific language.}
\\
\vspace{0.25cm}
(a) \textit{For this teaching activity, can you describe your interpretation of the learning process?}  Causing a student to absorb abstract ideas in a way that they understand, and to confirm that understanding with peers, is critical to the learning process.  PI relies on short, conceptual word problems posed in a multiple choice format.  These problems reveal cases in which a student is getting correct answers but for the wrong reasons.  We ask the students to explain the problem to themselves in their own words while removing confusion added by excessive numbers and formulas.  Practicing facet (i) of PI causes students to confront their lack of understanding of a concept, and to gain control over it in a practical way.  PI is an active learning technique, and facet (i) represents its starting point.
\\
\vspace{0.25cm}
(b) \textit{For this teaching activity, how do you incorporate teaching tools and practices?}  Since PI is such a well-studied technique in PER, I refer the reader to Appendix A, Sec. 7.2 of my prior PEGP, and the following book, website, and conference resource \cite{mazur2013peer,PhysPort,AAPTPI}.
\\
\vspace{0.25cm}
(c) \textit{For this teaching activity, can you show how the tenets of your teaching philosophy help achieve learning goals you set for your courses?}  The relevant learning goal (taken from my introductory course syllabi) is (C): ``to apply logical thinking to conceptually-posed physics problems.''  Within facet (i) of my PI modules, students \textit{order} the concepts and problems in their minds in a way that is the most useful for them.  Facet (i) also serves goals (A) and (B) because it aids in problem-solving.  Often, we observe introductory students mis-applying a physics equation to solve a problem because it is the only relevant formula they understand.  By practicing conceptual thinking, facet (i) helps break students out of this habit.  They instead \textit{order} a problem conceptually in their minds before solving it with the right technique.
\\
\vspace{0.25cm}
(d) \textit{For this teaching activity, can you focus on the ``why'' of specific teaching decisions?''}  I hear the following sentence every single Fall semester in office hours: ``I understand the formulas and concepts in physics.  I just need help `translating' the problem into the formulas.''  Facet (i) is vital for this part of problem-solving growth of the students.  In the phase of PI where facet (i) is occurring, students discover that they become responsible for this ``translation.''
\\
\vspace{0.25cm}
\textbf{Facet (ii): practice explaining concepts to others} 
\\
\vspace{0.25cm}
(a) \textit{For this teaching activity, can you describe your interpretation of the learning process?} Our own conceptual understanding of a concept must be confronted by the conceptual understanding of others (\textit{shared meaning}), and physical reality (\textit{order}).  The learning process for physics must include the chance to revise one's conceptual understanding by confronting faulty logic and absorbing the logic of others.  This process need not be confrontational, however, as in a debate or a sport.  Instead, facet (ii) of PI respects our psychology: we are more likely to absorb the thinking of a trusted peer who uses language we already understand.
\\
\vspace{0.25cm}
(b) \textit{For this teaching activity, how do you incorporate teaching tools and practices?}  See Appendix A, Sec. 7.2 of my prior PEGP, and the following book, website, and conference resource \cite{mazur2013peer,PhysPort,AAPTPI}.  I add here that, during the discussion (facet (ii)) phase of PI, I focus my attention on struggling students to help boost class equity.  In these side discussions, I listen to my students and share clues with them as their peer.  This technique requires a pre-established relationship of trust with the students, so early in the semester I focus more attention on building this relationship.
\\
\vspace{0.25cm}
(c) \textit{For this teaching activity, can you show how the tenets of your teaching philosophy help achieve learning goals you set for your courses?}  The purpose of facet (ii) of PI is literally to develop a \textit{shared meaning} between peers and instructors.  Learning goal (C) is directly addressed, and learning goals (A) and (B) are augmented.  Students begin to think more conceptually before solving problems algebraically rather than guessing the formula and plugging in numbers.  As an example, consider a student attempting to predict the final velocity of a falling object by plugging in the distance traveled divided by the time duration.  This is wrong, because the object is accelerating.  During group discussion, a lab partner might explain why the formula doesn't apply:  ``I don't think you can use that one. This one's more like the accelerating car where you have to times the acceleration by the delta-t.''  Peers are more likely to get a problem right when they develop a shared understanding.
\\
\vspace{0.25cm}
(d) \textit{For this teaching activity, can you focus on the ``why'' of specific teaching decisions?''} I have noticed that my students naturally form lab groups from peer groups in my courses.  They are teammates from varsity sports, friends in the same social society (e.g. Penns, Palmers), or bilingual students.  These connections tend to augment the mechanics of facet (ii) because lab group members are already friends and trust one another.  Thus, a reason \textit{why} I include PI facet (ii) is that it is in alignment with the group dynamic of my courses.
\\
\vspace{0.25cm}
\textbf{Facet (iii): teaching efficiently}
\\
\vspace{0.25cm}
Philosophically, I think of facet (iii) of PI as an auxiliary benefit that helps the students.  When a super-majority of students answer a conceptual question correctly in the first round, we move forward.  When this does not happen, we stop and have group discussions.  This forms a natural pace control.  To boost equity, as I've shared in prior PEGPs, I've built in the WAT function.  Students can hit a special button called WAT\footnote{\url{https://knowyourmeme.com/memes/wat}.} that causes me to stop and give additional clues, regardless of the proportion of students who've solved the problem correctly.  Students know to hit WAT if they are majorly confused, and this is more rare than the sort of confusion that gets resolved in group discussions.

\subsection{(3) PhET Simulations}

Having reflected on my use of PhET simulations, I have identified three facets (i)-(iii) that make them useful for my students. \textbf{(i) PhETs foster the extraction of patterns from physical systems} (\textit{order}).  The learning process involves pattern recognition, and PhETs provide a space for students to tinker with a system until they recognize the pattern of its behavior. \textbf{(ii)} PhETs provide an avenue for students \textbf{to construct graphical results.} Data can be generated intuitively with PhET simulations, because the confusions of building the apparatus and dealing with statistical error are gone.  Students focus on plotting results in a way that the group understands (\textit{shared meaning}).  The third facet is philosophically practical: \textbf{(iii)} PhET simulations allow us \textbf{to study physical systems we cannot build.}  A common example is the topic the solar system, which we use to study gravity.
\\
\vspace{0.25cm}
\textbf{Facet (i): extracting patterns from physical systems}
\\
\vspace{0.25cm}
(a) \textit{For this teaching activity, can you describe your interpretation of the learning process?} Part of the learning process in the physical sciences is to recognize patterns in systems (\textit{order}).  PhET simulations allow students to experiment in the absence of statistical error.  For example, consider a DC circuit powered by a 5 Volt battery.  Students can measure instantaneously the current flowing to various devices in the circuit using a tool.  To graph the data and extract Ohm's law (a linear relationship between voltage and current), they simply tune the battery voltage.  In real life, the students would have to swap batteries, measure the voltage of each battery, and keep track of errors.  These experimental skills are also part of the learning process, but PhET simulations help students to concentrate on \textit{just the pattern recognition.}
\\
\vspace{0.25cm}
(b) \textit{For this teaching activity, how do you incorporate teaching tools and practices?}  I incorporate PhET simulations as group activities completed by lab groups within my integrated lecture-lab format.  I give the students a brief tutorial on the projector screen, and I provide them with a written worksheet.  Working together, they construct a model using the components in the PhET, and graph data drawn from the model on the worksheet.  When appropriate, I construct my own version on the large screen, and we all compare results together (\textit{shared meaning}).
\\
\vspace{0.25cm}
(c) \textit{For this teaching activity, can you show how the tenets of your teaching philosophy help achieve learning goals you
set for your courses?}  Learning goals (A) and (B) are augmented indirectly.  Students are more likely to solve problems correctly using an equation if they have verified that equation experimentally.  Viewed as a pattern of physical behavior, physics equations once extracted from a system can be used to model similar systems by analogy.  Goal (B) is also addressed directly, in that PhET activities are an example of modeling a model to describe physical systems.  The difference is that they are using a computational tool rather than algebraic ones.  Another learning goal is reached by the \textit{order} achieved through pattern recognition is (D) ``to practice scientific experimentation, data analysis, and reporting of results.''  Students must order their thinking by analyzing data from the PhETs and constructing graphical results as a small group.
\\
\vspace{0.25cm}
(d) \textit{For this teaching activity, can you focus on the ``why'' of specific teaching decisions?''}  On a basic level, I include PhET simulations because they are based on extensive PER.  The appearance of the controls and equipment in the PhET programs feels similar to real lab components.  The measurement tools have obvious controls that can be learned quickly.  Once the students have done an experiment with a PhET simulation, doing the real version on the lab bench is more straightforward because they already understand the tools and patterns.
\\
\vspace{0.25cm}
\textbf{Facet (ii): constructing graphical results}
\\
\vspace{0.25cm}
(a) \textit{For this teaching activity, can you describe your interpretation of the learning process?} Part of the learning process in the physical sciences is convincing others of the validity of a concept or result (\textit{shared meaning}).  This is in addition to the ability to master a concept derived from experimentation well enough to apply it to word problems.  Facet (ii) of PhET activities is about creating a visual representation of experimental data such that another person can be expected to interpret the results and extract the pattern.  Thus, the experimenter must think about the details of how results are communicated, and in turn, ask themselves if they really understand the results.
\\
\vspace{0.25cm}
(b) \textit{For this teaching activity, how do you incorporate teaching tools and practices?} Philosophically, I must strike a balance for facet (ii) of PhET.  I need to scaffold the graph on the PhET worksheet so that the students can fill in their data, leaving a graph that can be interpreted by peers.  On the other hand, I want the students to practice constructing the elements of that graph on their own.  These elements include labeled axes, legends, proper numerical upper and lower limits, and data points that include any statistical errors.  We build these skills over the course of the first semester, and our graphics gradually become more sophisticated.  My expectation for the self-designed final projects (see below) is that all results communicated graphically can be interpreted without ambiguity. 
\\
\vspace{0.25cm}
(c) \textit{For this teaching activity, can you show how the tenets of your teaching philosophy help achieve learning goals you
set for your courses?} Learning goal is (D) ``to practice scientific experimentation, data analysis, and reporting of results.''  The work we do with PhET in facet (ii) is all about reporting of results such that someone else could reasonably understand them (\textit{shared meaning}).  We use PhET activities to build good habits, so that when we arrive at lab activities (see below) and the final project, students report their results accurately and unambigiously.
\\
\vspace{0.25cm}
(d) \textit{For this teaching activity, can you focus on the ``why'' of specific teaching decisions?''}  On a basic level, I focus on facet (ii) of PhET because without it, science itself goes awry very fast.  Even though I might know how I arrived at my physical conclusions, if I cannot display them such that someone else draws those same conclusions, then I am not doing science.  Having practiced interpreting each other's graphs in my courses, our students are equipped with an important skill they will use immediately in the scientific and professional communities they join after they graduate.
\\
\vspace{0.25cm}
\textbf{Facet (iii): studying systems we cannot build}
\\
\vspace{0.25cm}
A practical benefit of using PhET simulations is that we cannot always construct systems we would like to study.  Consider the relationship between the pressure, volume, and temperature of an ideal gas.  The \textit{kinetic theory of ideal gases} tells us that the reason these macroscopic quantities are related is because ideal gases are made of molecules that all have kinetic energy and momentum.  Pressure is caused by molecules transferring momentum to vessel walls, temperature is related to the average velocity per molecule, and work done by the molecules on the vessel change the volume.  Repeating ideal gas experiments done in chemistry courses would not demonstrate how these phenomena are caused by the \textit{molecules,} and we cannot see molecules on the lab bench.  A simulation is more illuminating because we \textit{can} see the molecules in action, and we can tune properties like temperature and observe the corresponding molecular behavior.

\subsection{(4) Laboratory Activities}

I have reflected on we conduct laboratory activities in my courses, and three facets emerge: \textbf{(i)-(iii)}.  In a book we are reading for my current INTD100 section \cite{scientific_attitude}, the author draws the \textit{line of demarcation} between science not science with two ideas.  First, \textit{science cares about data,} collected from a controlled experiment.  Second, \textit{science is willing to change theories based on new data.}  Thus, facet \textbf{(i)} of the laboratory tenet of my teaching philosophy is to show the students that \textbf{science cares about data} (\textit{shared meaning}).  However, raw data can be made to ``say'' anything if we fudge it enough.  Facet \textbf{(ii)} is about \textbf{learning to keep an experiment under control} (\textit{order}).  Students learn through controlled experimentation that IF they perform an experimental action and THEN the data responds accordingly, with no other factors being changed, the most convincing results are generated.  Finally, given that no experiment has infinite precision, facet \textbf{(iii)} is about \textbf{error analysis} (\textit{order}).
\\
\vspace{0.25cm}
\textbf{Facet (i): science cares about data}
\\
\vspace{0.25cm}
(a) \textit{For this teaching activity, can you describe your interpretation of the learning process?}  It might seem obvious that physical science courses need laboratory work.  At first, some students are not skeptical when we show them how theoretical physics makes a prediction.  Some are accustomed to being told what to think, and others place too much trust in the order of theoretical physics.  We need to perform experiments to provide the students with hard proof that what we are teaching them is real.  Though facet (i) may or may not help with solving word problems, the students see on a deeper level that physics is built upon incontrovertible evidence.
\\
\vspace{0.25cm}
(b) \textit{For this teaching activity, how do you incorporate teaching tools and practices?}  Constructing a good lab activity for physics courses is an art form\footnote{I am grateful to Prof. Seamus Lagan for taking time to help me grow in this area.}.  Design an apparatus that is too complex, and the students will learn nothing.  To simple, and the apparatus will not tell the students anything.  As a department, we maintain a library of proven labs based on our experience and PER.  It is vital that we maintain the integrated lecture-lab format for our courses.  The lab activity that provides the proof of a theorem immediately follows the traditional lecture and PI that introduced it.  The students are given time and space to apply what they've just learned, and their efforts are rewarded with proof that what they've learned works.  For practical reasons, students work on the lab activities in groups of 2-4 (see below).
\\
\vspace{0.25cm}
(c) \textit{For this teaching activity, can you show how the tenets of your teaching philosophy help achieve learning goals you
set for your courses?}  Goal (D) (taken from my syllabi) is ``to practice scientific experimentation, data analysis, and reporting of results.''  Lab groups (2-4 people) work together to understand the lab activity, so they engage in a form of \textit{shared meaning}.  When proctoring these lab activities, I usually observe the following behaviors in my students: one student explaining to another why something is not working, how to compute the result from the data, or even waving their hands to model \textit{why} they think the result makes sense.  On the lab handouts I provide, I require the students to create graphs, calculate numerical results from raw data, and to report results.  Thus, we reach goal (D) as a team.  Periodically, we compare results group by group or with my version of the setup at my desk\footnote{I'm grateful to Prof. Glenn Piner for this suggestion.}.  This prevents the groups from getting stuck on the wrong path.
\\
\vspace{0.25cm}
(d) \textit{For this teaching activity, can you focus on the ``why'' of specific teaching decisions?''}  Performing experiments to confirm hypotheses from theory is at the core of physics.  It would feel strange to teach an introductory course with \textit{only} theoretical problems.  The reasons why we conduct labs in an integrated lecture-lab format are to reinforce the theory the students learn only moments prior, and to give them the opportunity to prove the theory works.  Encouraging them to generate this proof tells them that we respect their natural skepticism and curiosity.
\\
\vspace{0.25cm}
\textbf{Facet (ii): learning to keep an experiment under control}
\\
\vspace{0.25cm}
(a) \textit{For this teaching activity, can you describe your interpretation of the learning process?}  Sometimes people use the phrase ``if this then that'' (ITTT) to represent causal connections.  For some lab activities, we can give the students a knob, lever, or action that they can use to tune to the effect on the data.  One example is my magnetic induction lab, in which a magnet on a spring bounces into and out of a coil of wire.  Students view the voltage induced in the coil with an oscilloscope.  \textit{Faraday's Law} states that the rate of change of the magnetic field in the coil determines the amount of voltage in the coil.  \textit{If} the students tune the velocity of the magnet by compressing the spring more, THEN the size of the oscilloscope signal increases.  When the students experience ITTT they see equations like Faraday's Law with new eyes.  Laws of physics are not abstract equations, but descriptions grounded in ITTT.
\\
\vspace{0.25cm}
(b) \textit{For this teaching activity, how do you incorporate teaching tools and practices?}  See facet (i), question (b).  It is important to note that ITTT is less obvious in some labs.  One example is a lab covering Snell's Law, which predicts the outgoing angle of a refracted laser beam given the angle of incidence on a glass surface.  The glass has a fixed index of refraction that relates incoming angle to outgoing angle.  While we can change the incoming angle and see the effect on the outgoing angle, we cannot change the index of glass.  However, the students experience ITTT for this property of light via a PhET simulation.  In the simulation, we \textit{can} tune the index, and see that the outgoing angle of the laser light changes, even though the incoming laser beam is not moving.
\\
\vspace{0.25cm}
(c) \textit{For this teaching activity, can you show how the tenets of your teaching philosophy help achieve learning goals you
set for your courses?}  Facet (ii) is derived from the principle of \textit{order.}  We design our lab activities such that the students are supposed to change only one input at a time to examine the effect on the outcome.  Students quickly learn that if too many inputs are changing simultaneously, the data will be useless.  So facet (ii) serves goal (D) (``to practice scientific experimentation, data analysis, and reporting of results'') in the sense that it enables the data analysis.  When students draw a connection between ITTT and data analysis, they tend to design experiments of their own that are kept under control.
\\
\vspace{0.25cm}
(d) \textit{For this teaching activity, can you focus on the ``why'' of specific teaching decisions?''}  If we did not think deliberately about facet (ii) when designing lab activities, then student learning would be diminished because the results would only generate confusion.  Students are disappointed when their lab ``didn't work,'' and they do not gain anything from the activity.  Thus, we design lab activities to be controlled so that the students learn from the results.  By the end of the semester, I prompt the students to design their final projects such that they are kept under control.
\\
\vspace{0.25cm}
\textbf{Facet (iii): error analysis}
\\
\vspace{0.25cm}
(a) \textit{For this teaching activity, can you describe your interpretation of the learning process?}  Part of the learning process for physics students is to notice that experimental results rarely match theoretical predictions exactly.  Understanding why this happens, and learning to improve both the precision and accuracy of our experiments is vital for student success.  Introductory physics students usually attribute disagreement between data and theory to ``human error,'' and it takes time for them to learn that statistical error occurs naturally.  I have created special lab activities focusing on statistical error\footnote{See supplemental material.}, and I place them in the middle of the semester after the students are more familiar with lab technique.  Learning to accept and manage statistical error, in my opinion, takes longer to solidify in the students' minds than facets (i)-(ii).  Because my introductory courses are year-long sequences, I choose not to rush facet (iii) into the process, but to build it slowly.
\\
\vspace{0.25cm}
(b) \textit{For this teaching activity, how do you incorporate teaching tools and practices?}  I create specific lab activities that teach both physical concepts and focus on statistical error.  We practice error propagation, when multiple measurements with error are combined in a formula to produce a result with compounded error.  We practice comparing results across lab groups to show that error is indeed statistical, and not a function of who is performing the measurement.
\\
\vspace{0.25cm}
(c) \textit{For this teaching activity, can you show how the tenets of your teaching philosophy help achieve learning goals you
set for your courses?}  The tenet of \textit{order} with respect to facet (iii) is critical to achieving goal (D).  My students learn that we can achieve order out of the apparent chaos of analyzing raw data, which in turn helps them to produce results capable of convincing others that physical laws hold despite imprecision.
\\
\vspace{0.25cm}
(d) \textit{For this teaching activity, can you focus on the ``why'' of specific teaching decisions?''} Error analysis is usually the most unfamiliar facet of lab activity education for my students.  Many students do not experience it until they reach college.  Thus, I must lead highly scaffolded activities regarding error analysis and propagation to give the students time and space to learn it.

\subsection{(5) Synergies}

My reflections give one facet \textbf{(i)} that makes the synergy between traditional lecture, PI, and laboratory activities useful for students: \textbf{solidifcation.}  In my answers below, I choose as an example my unit on DC circuits.  We are equipped with DC circuit tools in our labs, and an excellent DC circuits PhET is available\footnote{\url{https://phet.colorado.edu/en/simulations/circuit-construction-kit-dc}.}.  Thus, I can give traditional lecture content about circuits, have the students discuss circuits via PI, and follow that with an integrated PhET and lab activity.  I call facet (i) \textbf{solidification} because the students observe the match between there algebraic calculations, the simulation, and the lab activity all in the same class session.  Seeing this match gives the students a sense of satisfaction that they understand the topic completely.
\\
\vspace{0.25cm}
\textbf{Facet (i): solidification}
\\
\vspace{0.25cm}
(a) \textit{For this teaching activity, can you describe your interpretation of the learning process?}  Students are fully prepared to apply and remix the ideas of physics in their own projects when the ideas solidify in their minds.  The ideas feel right on a deeper level if they have correctly demonstrated them as theoretical predictions, ideally simulated, and tested practically in the lab.  After solidification, their confidence is such that they stop questioning their understanding of a concept and start using it without hesitation.  Picture a baseball player who ``just knows'' how to hit a pitch without thinking.  Consider again the example of DC circuit analysis.  Using the ideas that energy and total charge are conserved (constant) within a circuit, students can solve systems of equations that predict currents through multiple devices connected to a battery.  The algebra can become complex and confusing.  In the PhET, however, current and charge are animated, and the rules of the algebra are illustrated.  The students build on the bench the exact same circuit they have constructed virtually, and show that the volts and amps in real life match the simulation.  This leads them back to their algebra, which they can force to agree with their lab results.  Apprehension fades to confidence when they see that they got it all to work.
\\
\vspace{0.25cm}
(b) \textit{For this teaching activity, how do you incorporate teaching tools and practices?}  The incorporation of tools and practices is the same as learning activity (4) (Laboratory Activities).  Whenever this synergy bonus is possible, I put it into practice.  The limiting factor is the availability of PhET simulations corresponding to each laboratory activity we must perform.
\\
\vspace{0.25cm}
(c) \textit{For this teaching activity, can you show how the tenets of your teaching philosophy help achieve learning goals you
set for your courses?}  The synergy bonus my students derive from solidification is an important part of \textit{shared meaning.}  As a group, the students construct an apparatus for performing a measurement, match it to simulation, and understand it mathematically.  They confirm that they share a consensus as to \textit{why} it all works together, and their consensus is confirmed when I gather and display the results from each group.  All parts of learning goal (D) are enhanced by this synergy.
\\
\vspace{0.25cm}
(d) \textit{For this teaching activity, can you focus on the ``why'' of specific teaching decisions?''} I cannot predict when the light bulb will activate for a given student and a given topic.  We all witness these little miracles, when a student's eyes grow wide as an idea solidifies in their mind.  We cannot always know whther traditional lecture, PhET, or lab will be the activity that illuminates a student.  By including the synergy activity, I am maximizing the number of light bulbs activated in my class session.

\subsection{(6) Student-Designed Final Projects}

My reflections give two facets \textbf{(i)-(ii)} that makes the student-designed final projects useful for students.  Facet (i) is about \textbf{individual creativity}.  The creativity required for a good final project is related to developing both order and shared meaning within a lab group in my courses.  Facet \textbf{(ii)} is about developing a shared meaning with an audience: \textbf{communication of abstract arguments, numerical results, and graphical results.}
\\
\vspace{0.25cm}
\textbf{Facet (i): individual creativity}
\\
\vspace{0.25cm}
(a) \textit{For this teaching activity, can you describe your interpretation of the learning process?} In lieu of a final exam, my students propose, build, execute, and present their own physics project.  During the semester, we practice many facets of lab technique and theoretical calculations.  The student-designed projects meet a final learning need left unaddressed by our other activities.  Schemes for brilliant confirmations of physical laws are often concocted when we have a stroke of creativity. Creativity is a mode of thought very different from methodical algebraic derivation or meticulous experimentation.  The students beam when they get a chance to combine their creativity with physics in the final project.  I think this is because they finally have a chance to express themselves scientifically.  These summative projects allow them to apply what they've learned to a topic about which they are curious.  They are confronted by the usual practical concerns and errors of experimentation, but the reward is knowing that they can use physics to understand the world. 
\\
\vspace{0.25cm}
(b) \textit{For this teaching activity, how do you incorporate teaching tools and practices?}  I first invite proposals from lab groups that outline what they will measure.  Proposals must include diagrams and lists of parts they will need, including any lab equipment from our department.  Members of lab groups are fully responsible for coordinating the experiment with each other, and creating the final presentation.  Though I actively debug experiments with students, I am cautious in that phase to provide just enough guidance such that they see the path forward without doing it for them.
\\
\vspace{0.25cm}
(c) \textit{For this teaching activity, can you show how the tenets of your teaching philosophy help achieve learning goals you
set for your courses?} Facet (i) is about creating both order and shared meaning.  Once the students agree on an idea, they take ownership of it to make it a reality.  In our past communications, I have shown you how intricate and exciting these final projects can become.  Learning goal (E), taken from my syllabi, is ``to practice written and oral expression of scientifically technical ideas.''  Asking the students to own their idea and make it reality from inception to presentation is the best way I can imagine to address goal (E).
\\
\vspace{0.25cm}
(d) \textit{For this teaching activity, can you focus on the ``why'' of specific teaching decisions?''}  I can clearly observe how my students shine in this learning activity.  The reason why I include the creative aspect (facet (i)), as opposed to deciding what their project will be, is that they are creatively energized to do science.  The results have been wonderful and fascinating to watch.
\\
\vspace{0.25cm}
\textbf{Facet (ii): communication of abstract arguments, numerical results, and graphical results}
\\
\vspace{0.25cm}
(a) \textit{For this teaching activity, can you describe your interpretation of the learning process?}  Learning to communicate the results of the final projects is just as much art as it is science.  The best way to learn it is to do it and see if people understand you.  Students draw upon experiences in activity (3), PhET simulations, to help with creatingi meaningful graphs of data that can be interpreted by the class.  One semester, I tried to incorporate more than one student-led project, but the students told me that it made the semester too complicated.  I have settled on administering just one summative project.  Students can learn by imitating or remixing techniques in past presentations, so I give them example presentations created by me and my students from past courses.
\\
\vspace{0.25cm}
(b) \textit{For this teaching activity, how do you incorporate teaching tools and practices?}  The final presentation can be given live in class, or developed using digital storytelling techniques\footnote{See Secs. 2.1 and 2.2 of my prior PEGP regarding digital storytelling and the use of WeVideo.  On my syllabi (included in supplemental material), the live or WeVideo options are called option A and B.}.  The results are presented to the class near the end of the semester, and we get a chance to ask the lab group questions.  Given the group size (2-4 students) and the average number of students in my classes ($\approx 25$), we limit the presentations to 15 minutes so that the presentations take 1-2 days.  I review the presentations before they are given to the class, to help the students polish them and eliminate errors.  During the presentation, I encourage the audience to ask the presenting group questions.  Sometimes students feel shy about asking questions, so this is the hardest element of facet (ii) to implement.
\\
\vspace{0.25cm}
(c) \textit{For this teaching activity, can you show how the tenets of your teaching philosophy help achieve learning goals you
set for your courses?}  Facet (ii) is about developing a \textit{shared meaning} between myself and the presenting group, and between the presenting group and their peers.  Showing your results to others is fundamental to science, which is why I've included learning goal (E) in my syllabi.  Facet (ii) of the student-led projects gives the students time and space to practice learning goal (E) (``to practice written and oral expression of scientifically technical ideas'').
\\
\vspace{0.25cm}
(d) \textit{For this teaching activity, can you focus on the ``why'' of specific teaching decisions?''}  See (d) of facet (i).  The students enjoy having a chance to express themselves scientifically as they own these projects and make them shine.

\subsection{Outlook}

In our last communication, you recognized that I have a plan that appears to be working well for the students, and that my course evaluations reflect that.  As I've been in my room reflecting on why I teach the way I do, I've found that solving this rubix cube helps me to concentrate.  I laughed to myself when I realized that asking a physicist and engineer like me to write about teaching philosophy is like asking me to solve this rubix cube.  That is, I know what the final result should be (students successfully learning physics), but I'm less clear on how to lay out the pieces of the philosophy correctly.  When I've asked colleagues over the years to describe a teaching philosophy, I never hear a clear definition, but many facets.  Some say to express your values.  Others say to describe practices and pedagogy.  Some say that a personal narrative is important.  I've tried to align the facets of my teaching for you in a way that is, if nothing else, easy to understand.
\\
\vspace{0.25cm}
One recent development in my teaching is the inclusion of \textbf{online laboratory activities}.  We began using services like Pivot Interactives during quarantine, when we were barred from doing any in-person teaching.  Online lab activities give the user a chance to play a video of a constructed apparatus in operation.  The user can use on-screen tools to collect data from the video.  Built into the lab web page are tools for creating graphical results from this data.  Data analysis, question prompts regarding conclusions we draw from the results, and grading are all seamlessly integrated.  It is tempting to use these online lab services going forward.  I am of two minds about this strategy.
\\
\vspace{0.25cm}
Online lab activities will not replace activity (4) of my philosophy, but they have interesting facets that make them useful for students.  Online labs remove the confusion inherent to constructing lab apparatuses, which makes them more like learning activity (3): PhET simulations.  In fact, online labs and PhETs share facets (i)-(ii) of tenet (3) of my teaching philosophy.  They are not simulations, though, because the apparatuses are real but filmed.  The data is collected from a physical system subject to statistical error.  The potential for statistical error makes online lab activities more like learning activity (4): laboratory activities via facet (iii) of activity (4).  The systems being filmed in the online labs are systems we can build in our labs, but sometimes we have not had the time to build them yet.
\\
\vspace{0.25cm}
I imagine, as with all teaching decisions, the question will be resolved by making an overall determination about how well the activities serve the students.  If the students' experiences of facets (i)-(iii) of activity (4) are not diminished by including online labs, then we will probably continue to include them.  However, if the students begin to view them like PhET simulations, then the line between simulation and reality would be blurred.  This is not good for science, and we would boost the number of standard laboratory activities.  In any event, I hope it is clear that I have a conscientious teaching philosophy that continues to serve students well at Whittier College, and I always welcome your insight and wisdom.

\end{document}
