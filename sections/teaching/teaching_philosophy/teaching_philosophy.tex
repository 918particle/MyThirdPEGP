\documentclass[../../../main.tex]{subfiles}

\begin{document}
I have reflected on the teaching practices I have used in light of the four suggestions I received in our last communication.  What follows is a reflection on the six main teaching activities I use in the physical sciences.  Many of these practices also apply to my liberal arts courses and college writing seminars, but not all of them.  Each activity is derived from the over-arching principles of \textit{order} and \textit{shared meaning.}  However, I have chosen not to cover those principles in detail, since I have already shared that in Sec. 2.1 of my prior PEGP.  Instead, for each of my six main teaching activities, I briefly articulate how they flow from \textit{order} and \textit{shared meaning}.  Next, I answer the following four questions you posed.
\begin{itemize}
\item (a) \textit{For this teaching activity, can you describe your interpretation of the learning process?}
\item (b) \textit{For this teaching activity, how do you incorporate teaching tools and practices?}
\item (c) \textit{For this teaching activity, can you show how the tenets of your teaching philosophy help achieve learning goals you set for your courses?}
\item (d) \textit{For this teaching activity, can you focus on the ``why'' of specific teaching decisions, instead of ``how?''}
\end{itemize} 
After completing this exercise, I reflect upon where I might make changes going forward, and which pieces seem to be working well.  One interesting shift that has taken place within our department is the use of \textit{online laboratory activities.}  I reflect on \textit{why} we have incorporated this practice during the pandemic and going forward.

\subsection{General Approach: How I Teach with Six Easy Pieces (1-6)}

I begin by reflecting on my teaching philosophy for introductory physics courses.  The most succinct expression of my teaching philosophy may be broken into the six main teaching activities, or ingredients, below.  As I shared in Sec. 2.1 of my prior PEGP, physicists tend to classify students into \textit{majors} and \textit{non-majors}.  The broadest definition of a \textit{major} student is a student requires at least some physics or engineering courses above the introductory level.  Most physics education research (PER) is done in the context of courses designed for \textit{non-majors}.  PER provides empirical evidence for \textit{why} teaching practices are effective.  Most students who take physics and engineering courses at Whittier College are non-majors.  Thus, most of the energy I devote to my teaching philosophy focuses on application to introductory courses.  When I teach advanced physics and engineering courses, I simply use the same ingredients with a different recipe.  Advanced physics and engineering courses are built from introductory ones, so students in those courses have already had the first rendering of ingredients (1-6) and are ready for something new.  When I teach liberal arts courses and sections of college writing seminar, I use a subset of activities (1-6), augmented by writing assignments.
\\
\vspace{0.25cm}
I begin introductory course sessions with teaching activity (1): traditional lecture format.  I start with a warm-up exercise drawn from textbook readings assigned 1-2 days prior.  Before giving the solutions, I present a thorough agenda for the session.  I then solve the warm-ups as on the whiteboard, and build on them with more intricate examples and proofs of theorems.  Next, I usually proceed to ingredient (2): peer-instruction \cite{mazur2013peer}.  I pose conceptual multiple-choice questions to the students, based on activity (1).  Students record their anonymous answers electronically, and we view the answer distribution.  We discuss our responses as peers in small groups, and I search for students who appear to be struggling and help stimulate their thinking by re-phrasing the question or giving them clues.  The students give a second response, and we move forward when a super-majority of the students get it right.  Next, we arrive at activity (3): PhET simulations.  As the systems we study grow more complex, it is useful to simulate them.  Physics education technology, or PhET \cite{phet}, consists of HTML5 applications designed using extensive PER.  I provide written activities the students must complete while operating the simulation.
\\
\vspace{0.25cm}
I use the second half of our session to conduct a laboratory activity, ingredient (4).  We work as a team in our department to maintain consistent lab pedagogy and equipment.  The students complete labs that cover the same content presented in activities (1)-(3) during class sessions.  Sometimes activity (5) becomes possible, in which we cover lecture content that is testable in both a PhET and a lab.  We align theoretical predictions with simulation and lab experiment\footnote{Ideally, we would do this every time, but there are not yet PhET simulations for all labs.}.  The students perform activity (6) near the end of the semester, when they propose, build, execute, and present experiments as small groups to the class.  Teaching activities (1)-(6) represent the \textit{average} recipe for my courses.  As a cookbook contains a diverse collection of healthy recipes using common ingredients, I mix these ingredients in new and interesting ways to avoid repetition and maintain student engagement.

\subsection{(1) Traditional Lecture Format}

My reflections provide three facets \textbf{(i)-(iii)} that make traditional lecture useful for students. \textbf{(i)} Solving problems on the whiteboard \textbf{displays the components of physics} in step-by-step fashion.  These components include variables, estimation, units, functions, algebra/calculus, solutions and graphs, and checking results by examining units, limiting cases, and symmetry.  Facet (i) is derived from the principle of \textit{order,} in that statements about the natural world must be ordered using consistent terminology and mathematics. \textbf{(ii)} Traditional lecture gives the students \textbf{memorizable examples} that serve as concrete anchor points, and it is a form of \textit{shared meaning}.  Our mutual understanding of a system is assured, and students use them to solve problems that share traits with the example.  \textbf{(iii)} By \textbf{linking basic examples together}, students begin to solve harder problems for more complex systems.  Because complex physical systems in physics may be reduced to sub-systems we already understand, the students experience the intrinsic \textit{order} of physics and use it to model complex systems.
\\
\vspace{0.25cm}
\textbf{Facet (i): displaying the components of physics problem solving.}
\\
\vspace{0.25cm}
(a) \textit{For this teaching activity, can you describe your interpretation of the learning process?}  As with any subject, students must first understand how to use its components before solving problems or creating something new using those components as one.  One example that arises every Fall semester in my courses is the usage of physical units with variables.  If the students are solving for the final speed of a system, some will write ``the final speed is 12.''  My next question is always ``twelve what? Kilometers per hour, feet per second ...'' By giving explicit examples, students become aware that there are components of physics calculations they do not yet understand.  The students learn from this activity by following our example.  Once students master concrete examples, they can adapt them to model similar systems.
\\
\vspace{0.25cm}
(b) \textit{For this teaching activity, how do you incorporate teaching tools and practices?} While traditional lecture content is considered old-fashioned in PER, solving a problem with nothing but a piece of chalk in front of a group of students will always have a place in my teaching.  As long as I provide a written warm up for the students, solving it in front of them gives them an anchor point that can be copied and studied.  I have also learned to control the \textit{pace} of this content, to maximize the learning of a diverse group of students.
\\
\vspace{0.25cm}
(c) \textit{For this teaching activity, can you show how the tenets of your teaching philosophy help achieve learning goals you set for your courses?} Two relevant learning goals\footnote{Learning goals are always listed in my course syllabi.  See supplemental material for details.} set for my recent two-semester sequence of algebra-based physics are (A) To solve word problems pertaining to physics and mathematics, and (B) to construct mathematical models of mechanical systems.  Facet (i) of traditional lecture addresses (A) by breaking down a word problem into its components, while translating the words into mathematical statements.  This is especially useful for students who grew up in a bilingual setting.  Facet (i) of traditional teaching addresses (B) because I demonstrate how harder problems can be broken into more manageable components.
\\
\vspace{0.25cm}
(d) \textit{For this teaching activity, can you focus on the ``why'' of specific teaching decisions, instead of ``how?''}  The most basic reason that comes to mind is that the students ask for it.  It's concrete, and feels right to them.  Even if more modern PER methods are shown to be effective, I have learned that a course with too little of facet (i) is disorienting to learners.
\\
\vspace{0.25cm}
\textbf{Facet (ii): memorizable examples.}
\\
\vspace{0.25cm}
(a) \textit{For this teaching activity, can you describe your interpretation of the learning process?} On a basic level, the learning process includes memorization and repetition, similar to the way a new student of music or a new language learns.  Compare the way someone who speaks English begins to learn Spanish, as opposed to \textit{linguistics}.  Memorization and repetition plays a stronger role in the former.  Compare a new student of the guitar learns music to a student of \textit{musical theory.}  Repetition and memorization again play a stronger role in the former.
\\
\vspace{0.25cm}
(b) \textit{For this teaching activity, how do you incorporate teaching tools and practices?}  At a minimum, once per class session, I provide a written handout (the warm up problems) that I work as examples.  The students then have completed example problems they take home for study.  Students also take notes as I solve further problems, or elaborate with a proof of a theorem.  I also think very carefully about selecting problems for homework and exams, knowing that they must have some connection to the example base I've given to the students.
\\
\vspace{0.25cm}
(c) \textit{For this teaching activity, can you show how the tenets of your teaching philosophy help achieve learning goals you set for your courses?}  Facet (ii) addresses learning goal (A) above by demonstrating specifically how the words of a problem lead to the actions we take to solve it with mathematical reasoning.  When examples are memorized, problems involving similar logic with different numbers can be solved with ease\footnote{Sometimes such collections of word problems are called ``homomorphic.''}.  Facet (ii) addresses learning goal (B) by providing providing memorizable models of simple systems that can be combined to build mathematical models.
\\
\vspace{0.25cm}
(d) \textit{For this teaching activity, can you focus on the ``why'' of specific teaching decisions, instead of ``how?''} In a practical sense, the fact I make these examples memorizable for the students helps them with their homework, studying for exams.  The confidence of the students who find physics intimidating is boosted.  I recall a time I saw my student Deninson Cortez-Cruz stapling together all of my warm up problems.  I remarked that it looked like his studying was going well, and he replied: ``Oh yes, professor.  This packet is like my Bible.''  I smiled, because I knew that he was going to use those resources to help his lab group ace the class, and they all did.
\\
\vspace{0.25cm}
\textbf{Facet (iii): linking basic examples together}
\\
\vspace{0.25cm}
(a) \textit{For this teaching activity, can you describe your interpretation of the learning process?} One obvious part of the learning process for technical subjects is that an understanding of complex systems comes after understanding simpler ones.  For example, suppose I have already covered the topics of \textit{friction} and \textit{momentum transfer}, with basic examples provided for each.  The students know how to predict the deceleration as the car slides against friction.  They also understand how to predict the velocities of objects that collide.  If I ask them to predict the final velocity of a vehicle \textit{struck by} another that had been sliding, facet (iii) is \textit{why} they can solve this problem.  The wrong approach would be to teach them about friction, then momentum transfer, and conclude that they should be able to solve the harder problem because they know the underlying concepts.  Rather, \textit{why} facet (iii) is so important is that the students learn how to couple simple ideas into complex ones.
\\
\vspace{0.25cm}
(b) \textit{For this teaching activity, how do you incorporate teaching tools and practices?} Towards the end of traditional lecture time in class sessions, I occasionally link two basic examples or concepts together to solve a harder problem for the students.  When appropriate, I assign \textit{challenge problems} for bonus points that link together three or more basic concepts.
\\
\vspace{0.25cm}
(c) \textit{For this teaching activity, can you show how the tenets of your teaching philosophy help achieve learning goals you set for your courses?}  One could argue that the purpose of facet (iii) of traditional lecture is precisely to address learning goal (B): to construct mathematical models of mechanical systems.  This is a learning goal for all of the introductory physics courses, and it does not come easily to non-majors without dedicated training.
\\
\vspace{0.25cm}
(d) \textit{For this teaching activity, can you focus on the ``why'' of specific teaching decisions, instead of ``how?''}  I would compare this part of the process to a piano student first learning scales before beginning to compose songs.  Once students understand the \textit{order} built into physics, they trust that connecting simple models together to form complex ones is a strategy that works.  Though facet (iii) is the most challenging within traditional lecture, it is worth the payoff as the problem-solving capability grows more powerful as a result.

\subsection{(2) Peer-Instruction (PI)}

My reflections provide three facets \textbf{(i)-(iii)} that make peer-instruction useful for students.  Students must \textbf{(i) form an argument in their own scientific language}, which relates to the principle of \textit{order} because they order the concepts of physics in their minds to achieve this.  Students must then \textbf{(ii) practice explaining concepts to others.}  Peer-instruction natural connects to \textit{shared meaning} via facet (ii).  Finally, peer-instruction enables \textbf{(iii) teaching efficiently.}  Finally, peer-instruction has built-in pace control that keeps students engaged\footnote{I don't find that this benefit of peer-instruction relates to \textit{order} or \textit{shared meaning}, but it is important enough to include here.}.
\\
\vspace{0.25cm}
\textbf{Facet (i): forming an argument in one's own scientific language.}
\\
\vspace{0.25cm}
(a) \textit{For this teaching activity, can you describe your interpretation of the learning process?}  Critical to the learning process in technical subjects is the ability of a student to practice absorbing abstract ideas in their minds in a way that they understand, and later confirming their understanding with peers.  Peer-instruction relies on short, conceptual word problems posed in a multiple choice format.  These problems reveal cases in which a student is getting correct answers but for the wrong reasons.  We ask the students to explain the problem to themselves in their own words while removing the added confusion of dealing with the numbers and formulas specific to the current problem.  Practicing facet (i) of PI causes students to confront their lack of understanding of a concept, and to gain control over it in a way that is practical for them.
\\
\vspace{0.25cm}
(b) \textit{For this teaching activity, how do you incorporate teaching tools and practices?}  Since PI is such a well-studied technique in PER, I refer the reader to Appendix A, Sec. 7.2 of my prior PEGP, and the following books, websites, and conference resources \cite{mazur2013peer,PhysPort,AAPTPI}, 
\\
\vspace{0.25cm}
(c) \textit{For this teaching activity, can you show how the tenets of your teaching philosophy help achieve learning goals you set for your courses?}  The relevant learning goal (taken from my introductory course syllabi) is (C): to apply logical thinking to conceptually-posed physics problems.
\\
\vspace{0.25cm}
(d) \textit{For this teaching activity, can you focus on the ``why'' of specific teaching decisions, instead of ``how?''}
\\
\vspace{0.25cm}

%don't forget confronting faulty logic (facet iii) in the book but combined with (ii) here.
\subsection{(3) PhET Simulations}
things
\subsection{(4) Laboratory Activities}
things
\subsection{(5) Synergies}
things
\subsection{(6) Student-Designed Final Projects}
things
\subsection{Outlook}

\end{document}
