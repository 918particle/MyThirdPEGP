\documentclass[../../../main.tex]{subfiles}

\begin{document}
I have reflected on the teaching practices I have used in light of the four suggestions I received in our last communication.  What follows is a reflection on the six main teaching activities I actually use in instruction in the natural sciences at Whittier College.  Many of these practices also apply to my liberal arts courses and college writing seminars, but not all of them.  Each of these activities is derived from the over-arching principles of \textit{order} and \textit{shared meaning.}  However, I have chosen not to cover those over-arching principles in detail, since I have already shared that with you (see Sec. 2.1 of my prior PEGP).  Instead, for each of my six main teaching activities, I briefly articulate how they flow from \textit{order} and \textit{shared meaning}.  Next, I answer the following four questions you posed for each piece in turn:
\begin{itemize}
\item (a) \textit{For this teaching activity, can you describe your interpretation of the learning process?}  To address this item, I show how each piece contributes to the learning process for physics instruction, and attempt to expand it to computer science and mathematics instruction.  The learning process for these topics can be complex for some students.  As such, I do not attempt to articulate \textit{where or when} in the learning process a given teaching activity plays a role, but simply \textit{how} it plays a role.  I do, however, employ a standard order in which I deliver course material using the pieces.  Thus, there is an average order to the learning process, but the ``light-bulb moments'' can always occur in different phases for diverse sets of students.
\item (b) \textit{For this teaching activity, how do you incorporate teaching tools and practices?}  To address this item, I show how specifically I deliver the course content using teaching tools relevant for the activity style.  I chose to imagine this item as being related to the nuts and bolts specific to a given teaching activity.
\item (c) \textit{For this teaching activity, can you show how the tenets of your teaching philosophy help achieve learning goals you set for your courses?} To address this item, I first list the learning goals we set for the courses within the scope of this PEGP.  Next, I draw a connection to the \textit{order} and \textit{shared meaning} of physics and physics instruction to the activity.  Finally, I explain \textit{why} this helps the students reach the goals we set.
\item (d) \textit{For this teaching activity, can you focus on the ``why'' of specific teaching decisions, instead of ``how?''}  I took this question as an opportunity to provide specific examples and references as to why this teaching practice is beneficial to our students.  I draw upon evidence from my 2019 Supplemental PEGP focused on teaching, sources in physics education research (PER), and practical experience.
\end{itemize} 
After completing this exercise, I reflect upon where I might make changes going forward, and which pieces seem to be working well.  One interesting shift that has taken place within our department is the use of \textit{online laboratory activities.}  These activities were very helpful during the quarantine period of the pandemic.  During lockdown, we were not able to conduct in-person lab activities.  Laboratory activities are one of the six major pieces of my teaching practice, so this could have been a major setback for student success.  Prof. Lagan discovered a service called Pivot Interactives that provided interactive, online versions of in-person laboratory activities.  We find this service useful in certain situations even though in-person teaching has resumed.  I reflect on \textit{why} we continue to utilize this practice, and how there might be a role for online laboratory activities as a seventh teaching activity within my teaching philosophy. 

\subsection{General Approach: How I Teach, Six Easy Pieces (1-6)}

As I reflect on my AY 2021-2022 courses, I realize that I most often think of introductory physics courses when I picture my teaching philosophy.  As I have shared with you in prior PEGP reports (Sec. 2.1 of my prior PEGP), physicists tend to classify students into \textit{majors} and \textit{non-majors}.  The broadest definition of a \textit{major} student is a student who has chosen one of the following for their major: Physics, ICS/Physics, ICS/Math, 3-2 Engineering/Physics, 3-2 Engineering/Math, or 3-2 Engineering/Computer Science.  These majors require at least some courses above the introductory level.  The bulk of physics education research (PER) is done in the context of courses designed for \textit{non-majors}, or students who decide upon a major not in the above list.  The purpose of PER is to provide empirical evidence for \textit{why} specific teaching decisions and practices are more effective than others.  A large majority of students who take physics and engineering courses at Whittier College are non-majors, and students who struggle with college-level physics tend not to be majors.  Thus, most of the energy we devote to our teaching philosophy focuses on introductory courses.  It is important to note that our advanced courses are not taught so differently from our introductory ones. However, the students who do take advanced physics courses have already gained extensive mathematical and computational training from us in introductory courses.  Thus, they too benefit from our teaching philosophy.
\\
\vspace{0.25cm}
I begin a typical introductory physics course session with teaching activity (1): traditional lecture format.

\subsection{(1) Traditional Lecture Format}
things
\subsection{(2) Peer-Instruction}
things
\subsection{(3) PhET Simulations}
things
\subsection{(4) Laboratory Activities}
things
\subsection{(5) Synergies}
things
\subsection{(6) Student-Designed Final Projects}
things
\subsection{Outlook}

\end{document}
