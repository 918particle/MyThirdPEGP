\documentclass[../../../main.tex]{subfiles}
 
\begin{document}
\label{sec:intro_course_desc}

Here I describe the introductory courses I taught in AY 2021-20220, with connections to learning activities in my teaching philosophy, and departmental goals.
\\
\vspace{0.15cm}
\textbf{\textit{Algebra-based physics (135A/B)}}. Algebra-based physics, PHYS135A/B, is a two-semester integrated lecture/laboratory sequence covering classical mechanics (Newton's Laws) to electromagnetism\footnote{See supplemental material for example syllabi.}.  PHYS135 is a requirement for majors such as KNS, CHEM, and pre-medical students.  Students practice problem-solving with algebra, trigonometry, and vectors.  I employ the active learning strategies (1)-(6) in Sec. \ref{sec:teaching_philosophy} to satisfy departmental goals 1, 4, and 6.
\\
\vspace{0.15cm}
When selecting course content, one learning focus for non-majors is \textbf{curiosity}.  The content associated with learning activities (4)-(6) in Sec. \ref{sec:teaching_philosophy} are chosen to boost students' curiousity.  This is especially true of learning activity (6), the student-led final project in which they design their own physics experiment.  I also help the students to practice communication of scientific ideas (\textbf{Departmental goal 7}) by giving them an extra credit opportunity in which the communicate the recent findings of scientists at other institutions to the class.  Once the students overcome nerves, they begin to volunteer and choose content connected to their major.  Another introductory learning focus is \textbf{improvement of analysis skill}.  To address this focus, I use tenets (1)-(3) of my teaching philosophy.  PI modules, PhET modules, and traditional lecture content form a flexible and diverse strategy for improving the students' analysis skill (\textbf{Departmental goals 1, 4, and 6}).
\\
\vspace{0.15cm}
My third introductory learning focus is \textbf{applications to society}.  The obvious routes are the applications in the OpenStax texts \cite{openstax1} regarding kinesiology and medicine.  The students experience PI modules and example problems with topics such as motion/work/energy in the human body, and nerve cells as DC circuits.  The modules I select depends on the students' majors.  Including content specifically pertaining to the students' majors is highly beneficial to keep students engaged.

\end{document}
