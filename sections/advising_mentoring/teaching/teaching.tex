\documentclass[../../../main.tex]{subfiles}
 
\begin{document}

Advising and mentoring students resembles our teaching practice, because we must create a sense of \textit{order and shared meaning} in the mind of the student surrounding the curriculum. 
\\
\vspace{0.25cm}
Physics professors often classifiy students into two broad categories: \textit{non-majors} and \textit{majors} (see Sec. \ref{sec:teaching_philosophy}).  Most of our advisees as teachers fall into the first category.
\\
\vspace{0.25cm}
\begin{table}
\centering
\begin{tabular}{| c | c |}
\hline
\hline
Semester & \textbf{Number of First Year Advisees} \\ \hline
Fall 2019 & 15 \\ \hline
Fall 2020 & 14 \\ \hline
\hline
All semesters & \textbf{Physics, ICS, and 3-2 Majors} \\ \hline 
& Cassady Smith (Physics '20) \\ \hline
& John Paul G\'{o}mez-Reed (Math/ICS '21) \\ \hline
& Nicolas Clarizio (Physics, Business Admin. '19) \\ \hline
& Alex Ortiz-Valenzuela (3-2 Engineering/Physics  '22) \\ \hline
& Raymond Hartig (Physics and Math '23) \\ \hline
& Adam Wildanger (3-2 Engineering/Physics '21) \\ \hline
& Matthew Buchanan Garza (ICS/Physics '23) \\ \hline
& Natasha Waldorf (ICS/Physics '24) \\ \hline \hline
All semesters & \textbf{Whittier Scholars Program Majors} \\ \hline
& Nicolas Bakken-French (WSP '21) \\ \hline
\end{tabular}
\caption{\label{tab:advisees} A summary of my advisees, broken into three categories: first-year advisees, STEM majors, and WSP majors.  There are some first year advisees who have chosen ICS/Math for their major, for whom I remain a mentor.  One example is Emily List (ICS/Math '23).}
\end{table}

Advising non-majors follows a basic progression: introducing them to the curriculum and campus (\textit{order}), beginning a conversation surrounding major selection (\textit{shared meaning}), and future course selection.

\end{document}