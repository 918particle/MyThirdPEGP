\documentclass[../../main.tex]{subfiles}

%academic year.  
%
%(a) Armed Services and Govt. Internships, Andrew Householder, Natasha Waldorf, Adam Wildanger, Dane Goodman (volunteering) (b) Ondrasik-Groce Fellowship, Raymond Hartig (c) WSP and Jackson Diamond, and future COSC+WSP connections (d) Matthew Buchanan Garza, and the IDC grant for PhET creation (e) advising and mentoring first years.
%
%If i was so rigid in my advising, why would I encourage folks to join WSP?  WSP is all about thinking productively about a topic from multiple perspectives, creating a path for your own future through development of the educational process, and independently executing an original idea. 
%
%Why not LinkedIn?  In our discussions of a new curriculum, we propose to ask them to create a digital portfolio of projects they can later present future colleagues.  How is this different?
%
%I literally said this ``Sometimes the interests of the student align with my research, and sometimes they do not. I
%advise them nonetheless, and do my best to meet the student where they are to guide them forward.'' And yet you somehow arrived at the conclusion that my practice is too rigid.  Go figure.
%
%Did you not understand the story of my advisee Wyatt?  We literally changed his major from STEM to humanities and he felt that it was the right path.
% 
\begin{document}
\label{sec:advising_mentoring}

I reflect in this section on my role as an advisor and mentor.  On my role as an academic advisor, I highlight four themes in my reflections below.  (1) My ongoing participation in the Whittier Scholars Program is evidence that I am committed to interdisciplinary studies and the liberal arts.  (2) I have shown concrete examples of helping students broaden their vision to include new disciplines.  (3) The proportion of my students selecting ICS/Math and ICS/Physics continues to increase.  (4) According to the self-study completed by my department, I am responsible for a large portion of the first-year advising work my department contributes to the college.  It will be wise to rebalance that work, so I will likely reduce my first-year advising in the short term and increase advising of physical science and WSP majors.
\\
\vspace{0.25cm}
Tab. \ref{tab:advisees}, I summarize my advisees within the first-year orientation program, physical sciences, and Whittier Scholars Program.  After each entry, I indicate the outcome after their time at Whittier College, or their current job, or their plans upon graduation.  I have included my students in Physics Research (PHYS396), which is a course my department maintains to provide students 1-3 credits for doing research with us.  Though these credits do not count towards my teaching load, I provide these opportunities so that students can participate in my research and deepen their scientific and engineering toolkits.

\begin{table}[hb]
\small
\centering
\begin{tabular}{| c | c | c |}
\hline
\hline
Semester & \textbf{Number of First Year Advisees} & Note/Outcome \\ \hline
Fall 2019 & 15 & I taught physics \\ \hline
Fall 2020 & 14 & I taught INTD100 \\ \hline
Fall 2022 & 13 & I am teaching INTD100 \\ \hline
\hline
All semesters & \textbf{Physics, ICS, and 3-2 Majors} & \\ \hline 
& Cassady Smith (Physics '20) & Graduate student at Yale Univ., Keck Fellowship \\ \hline
& John Paul G\'{o}mez-Reed (Math/ICS '21) & Keck Fellow, Ondrasik-Groce Fellow \\ \hline
& Nicolas Clarizio (Physics, Business Admin. '19) & Mechanical Engineer, SMP Engineering Inc. \\ \hline
& Alex Ortiz-Valenzuela (3-2 Engineering/Physics  '22) & Admitted to Cal Poly Pomona, 3-2 program \\ \hline
& Raymond Hartig (Physics and Math '23) & Ondrasik-Groce Fellow, Fletcher-Jones Fellow \\ \hline
& Adam Wildanger (3-2 Engineering/Physics '21) & Admitted to USC, 3-2 program \\ \hline
& Matthew Buchanan Garza (ICS/Physics '23) & Working on DEI project (Sec. \ref{sec:dei}) \\ \hline
& Natasha Waldorf (ICS/Physics '24) & Currently at AFRL internship \\ \hline
& Dane Goodman (Physics with Math Minor '23) & Volunteering in my ONR projects \\ \hline \hline
All semesters & \textbf{Whittier Scholars Program Majors} & \\ \hline
& Nicolas Bakken-French (WSP '21) & Publishing book on glaciology and culture \\ \hline
& Jackson Diamond (WSP '23) & Developing bluetooth positioning system \\ \hline \hline
Current Semster & \textbf{PHYS396 Students}  & \\ \hline
& Matthew Buchanan Garza (ICS/Physics '23) & Working on DEI project (Sec. \ref{sec:dei}) \\ \hline
& Jackson Diamond (WSP '23) & Developing bluetooth positioning system \\ \hline
& Dane Goodman (Physics with Math Minor '23) & Volunteering in my ONR projects \\ \hline
& Raymond Hartig (Physics and Math '23) & Ondrasik-Groce Fellow, Fletcher-Jones Fellow \\ \hline
& Alex Ortiz-Valenzuela (3-2 Engineering/Physics '22) & Admitted to Cal Poly Pomona, 3-2 program \\ \hline
& Riley Sullivan (Physics '23) & Continuing Askaryan radiation research \\ \hline
& Ian Watanabe (ICS/Physics '23) & Continuing Askaryan radiation research \\ \hline
& Natasha Waldorf (ICS/Physics '24) & Currently at AFRL internship \\ \hline
\end{tabular}
\caption{\label{tab:advisees} A summary of my advisees, broken into three categories: first-year advisees, STEM majors, and WSP majors.  There are some first year advisees who have chosen ICS/Math for their major, for whom I remain a mentor.  One example is Emily List (ICS/Math '23).  Listed in the lower portion of the table are my PHYS396 - Physics Research students.}
\end{table}

\section{Advising First Year Students}

Reflecting on my time advising first-year students, I realize the first-year orientation and advising process has evolved substantially since I arrived in 2017.  During my second year serving the Enrollment and Student Affairs Committee (ESAC) with Prof. Gil Gonzalez, we discussed the INTD101 pilot program and I learned more about INTD100.  Since Prof. Gonzalez has fully implemented the INTD101 program, the orientation side of the process has been standardized, with course and curricular advising left to the professors.  I find it easier to understand my role within the new system.  Orientation organization is left to professionals on our staff colleagues in CAAS, Wardman Library, and INTD101 moderators.  The simplification allows me more time to focus on the academics of my students, which is important because fully 80\% of my advisees in my time thus far at Whittier have been first-year advisees.
\\
\vspace{0.25cm}
The majority of my first-year advisees arrive knowing firmly what major they will pursue.  Occasionally I encounter someone who is open to exploring new programs, and my first task is to discern if they are a good match for WSP.  I have advised one WSP student to graduation, and I am advising another this year.  Given my experience with these students, and having seen others that enter WSP, my first move is to assess the intensity and self-motivation of the student.  Though most (but not all) of the advisees steered my way are physical science students, I ask the students who demonstrate intensity and self-motivation if they are interested in customizing their education with interdisciplinary work.  This conversation can spark enough their curiosity enough to explore WSP.  I look for these characteristics because WSP is most fruitful when the student takes charge of the strategy and planning.  I would consider these characteristics necessary but not sufficient to identify a good match for WSP.  Some students with these characteristics will still want a standard major.  If this is the case, the next question in my mind is whether or not they have chosen \textit{right major} for them.
\\
\vspace{0.25cm}
I included the example of my student Wyatt Killien in my prior PEGP report (Secs. 5.1 and 5.2).  In our last communication you shared a concern that my advising is too ``rigid.''  I understand that to mean you do not want me to guide students down a pre-determined path, but foster a sense of exploration in areas other than STEM.  I want you to know that I do have the ability to recognize when a student needs to explore outside their initial plan.  Wyatt is a student who listed physics as the choice of major, but our conversations revealed to us that he did not seek physics as a science, but as framework for world-building within the context of science fiction and anime.  I asked Wyatt to have a conversation with his family about changing his major to digital art and design, and I recommended foundational courses in art and design at Whittier College.  After completing these tasks, Wyatt decided to change his major to digital art and design, in hopes of creating his own digital worlds one day.
\\
\vspace{0.25cm}
Another case of first-year advising that required considerable discernment was that of Grace Cooper, who is now a junior and a varsity swimmer.  Grace had only vague ideas surrounding course selection as a first-year, but clearly retained valuable skill from high school.  As a certified translater (English - Spanish) for her home state of Arizona, Grace had easily met the foreign language requirement of our curriculum and did not plan to take Spanish courses.  She shared vague notions of ``marine biology'' as her field of choice.  I struggled to discern \textit{why} Grace was considering environmental science as a major direction, so we made a key decision.  I convinced Grace to continue Spanish at the college level and potentially develop a minor in Spanish.  We later determined what paths might make sense as an environmental science major.  We discussed education, and scientific and conservation work.  These discussions were strongly informed by my experience with Nicolas Bakken-French, my first WSP student who focused on environmental science.  Grace was finally convinced the path made sense.  Grace is now an ENVS major who takes advanced Spanish courses, all while participating in varsity athletics.  I felt this was a quality result given the level of discernment required.

\section{Advising and Mentoring Majors in Physics, ICS, and 3-2 Engineering}

Within my cohorts of advisees who arrive with certainty in their selection of physical science majors, I have noticed a trend over the last four years.  The number of students who select ICS/Math or ICS/Physics has increased substantially.  Within a typical mixture of students I receive, absent the recent popularity of ICS/Math and ICS/Physics, I might encounter some Physics majors, some 3-2 Engineering Majors, and some ICS majors.  The pathways for students can still be organized into a decision tree, as I attempted to explain to you in my last PEGP report (see Sec. 5.1).  Though you shared that you found this structure rigid, I was trying to share that the students themselves already view these fields through similar structures.  I have decided to share what I think is a more flexible and generalized tree (Fig. \ref{fig:tree}).  For the purposes of this discussion, I use the term physical sciences to refer just to engineering, physics, mathematics, and computer science.  Students that indicate interest in chemistry are steered to my colleagues in other departments. 

\section{Advising and Mentoring Whittier Scholars Program Majors}

\end{document}
