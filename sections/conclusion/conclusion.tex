\documentclass[../main.tex]{subfiles}
 
\begin{document}

This semester marks the beginning of my sixth year with Whittier College.  The past five years have been filled with both wonderfully uplifting experiences, but also great sacrifices by all of us.  I hope that my experiences and contributions have met the bar for tenure from your perspective.  Students in my courses are learning efficiently and reporting that they have good experiences.  I reflected on my teaching philosophy, and answered pedagogical questions surrounding my main teaching practices.  My research students are growing intellectually and achieving professional success.  I accepted my responsibility for shared-governance through service to a committee charged with major decisions about the future of our institution.  Finally, I have reflected on how best I can mentor and advise our students in accordance with the values of Whittier College. 
\\
\vspace{0.25cm}
In the interest of keeping my report concise, I choose to close with one final story rather than a lengthy summary of my five years of service.  My work with the US Navy has revealed a potential boost for Whittier College and our students.  As I shared in the research section, my contacts at NSWC Corona have offered to begin an Educational Partnership Agreement (EPA) with us.  The Dean of the Faculty has given approval, and we are beginning to formulate paperwork that describes the details of the partnership and how it will benefit our students.  The EPA will give our students access to internship and research opportunities in a public sector setting for engineering and business roles.  The statute that created the EPA is specifically designed for HSIs and HBCUs.  My colleagues at NSWC Corona are quick to emphasize that they form these partnerships with small liberal arts schools precisely because they need students who can think independently from a variety of perspectives.
\\
\vspace{0.25cm}
Recently, I was introduced to one of my students who shared the experience of being autistic, and how learning to solve rubics cubes in record time helps with coping.  Rather than see this student merely socially awkward, I chose to be curious, and to get to know the student further.  The student showed me a notebook outlining solutions for the specific rotation combinations that solve the rubics cube with a classification of starting patterns.  This student was not yet a student of physics, but I immediately recognized the formulas in the notebook as those from linear algebra and group theory.  These subjects matter a great deal to physicists who must solve problems in quantum mechanics and quantum field theory.  I was astonished that a student with a diverse cognitive background, who was not initially interested in physics, would stumble across physics equations as part of a hobby.  This student's name is Luka Kenderian, and Luka will be my advisee in physics going forward.  I mention the story of our nacent EPA with the US Navy and my new advisee Luka because they are related.  The EPA with Whittier College would be a wonderful venue for students like Luka.  The EPA is designed to welcome students who are different, giving them a venue to share their talents.  This story is an example of what my colleagues at Whittier College can expect from me going forward: I will create opportunities designed for our students to belong and flourish.

Respectfully submitted,
Jordan C. Hanson

\end{document}
