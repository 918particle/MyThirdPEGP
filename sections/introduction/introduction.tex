\documentclass[../../main.tex]{subfiles}
 
\begin{document}

Dear Friends,
\\
\vspace{0.15cm}
I have compiled a report on my progress as an instructor, scholar, steward, and advisor for Whittier College during the period of 2021-2022.  I reflect on my educational and scholarly practices, and the service I have performed for the College as a mentor, advisor, and committee member.  I am pleased to report that my students are learning and growing at Whittier, and achieving success in the professional world.  In our last communication, after my second major PEGP report (delayed to the fifth year due to the pandemic), you asked me to reflect on my pedagogical practices.  In particular, you posed four questions I could use to discuss my teaching philosophy: (a) describe my interpretation of the learning process, (b) describe how I incorporate tools and practices in my courses, (c) describe how the tenets of my teaching philosophy help us to achieve learning objectives, and (d) focus on the \textit{why} behind specific teaching decisions as opposed to the \textit{how}.
\\
\vspace{0.25cm}
With questions (a)-(d) in mind, I have taken a straightforward and concise approach to the structure of my teaching philosophy in Sec. \ref{sec:teaching_philosophy}.  I reflected on what practices I actually use most often in my teaching, and answered questions (a)-(d) for all practices.  This exercise has been useful and enlightening.  It is my hope that this exercise provides you with useful insight into modern physics instruction.  I have also reflected on the \textit{learning focuses}.  These ideas were derived originally from my colleagues in my department, and I have modified them and made them my own.  I have come to the conclusion that the learning focuses guide my course creation and course content selection, and the \textit{why} behind teaching decisions I make are driven by the tenets of my teaching philosophy.  Though my teaching practices continue to generate positive student feedback, I have also identified areas of courses that need to be adjusted.
\\
\vspace{0.25cm}
Turning to my scholarship, I have new and exciting accomplishments to share with you in Sec. xxx.  I highlight three recent experiences as examples of my progress.  First, I have finally published in \textit{Physical Review D,} the flagship peer-reviewed journal in my field by the American Physical Society (APS)\footnote{J.C. Hanson and R. Hartig. ``Complex analysis of Askaryan radiation: a fully analytic model in the time domain.'' Phys. Rev. D \textbf{105}, 123019 (2022).}.  I am the first professor at Whittier College to achieve this.  This publication was the culmination of two years of work with an undergraduate student who has become a dear friend.  This result marks the first time a professor from my department has published in one of the \textit{Physical Review} journals in the last 16 years\footnote{See, for example, S. Zorba \textit{et al}. ``Fractal-mound growth of pentacene thin films.'' Phys. Rev. B \textbf{74}, 245410 (2006).}.  The piece provides the first fully analytic model of Askaryan radiation.  The results represent a significant contribution to my field, and my undergraduate researcher helped me to improve and finish this work (see Sec. xxx, and also Sec. 3.3.2 of my previous PEGP).
\\
\vspace{0.25cm}
The second experience pertains to my radio-frequency (RF) engineering and radar research with the Office of Naval Research (ONR).  In 2021, I published a paper involving the computational electromagnetism (CEM) of radar design\footnote{J. C. Hanson.  ``Broadband RF Phased Array Design with MEEP: Comparisons to Array Theory in Two and Three Dimensions.'' Electronics Journal \textbf{10}, 415 (2021).}.  This work was ranked Top 10 Most Notable Articles in Electronics Journal for six months, and researchers from four countries contacted me to collaborate.  This Summer, I was invited to speak at a CEM conference held at MIT.  I gave a 45 minute lecture on RF CEM design alongside colleagues from MIT, Google, Georgia Tech, Stanford, and BYU.  It was an incredibly meaningful moment in my career.  I took the opportunity to promote the mission and values of Whittier College.  Having received an ONR Summer Faculty Research Program (SFRP) grant for the third year in a row, I am eligible for ONR Senior Fellowship in Summer 2024 after a mandatory one-year break.  For the past two years, our ONR partners at NSWC Corona Division have granted us money and precision RF equipment to boost the engineering research experiences of our students.  Based on this fruitful collaboration, I'm happy to share that NSWC Corona would like to form an Educational Partnership Agreement (EPA) with Whittier College.  NSWC Corona forms EPAs to strengthen undergraduate engineering research, and to recruit engineering talent.  This project would be wonderfully beneficial to our students.
\\
\vspace{0.25cm}
The third experience is related to both my scholarship and teaching.  In Fall 2019, I taught INTD255, entitled ``Safe Return Doubtful: History and Current Status of Modern Science in Antarctica'' (CON2).  In Spring 2021, I taught INTD290, entitled ``A History of Science in Latin America'' (CON2,CUL3).  I concluded these courses with material related to the connection between modern scientific endeavors by peoples of diverse cultures and exploration literature.  I showed the students how it is possible to travel to Antarctica through the United States Antarctic Program (USAP).  As part of my research with the IceCube Gen2 collaboration\footnote{See, for example, \url{https://icecube.wisc.edu/}, and Secs. 3.1 - 3.3 of my prior PEGP.}, I have conducted research expeditions to Antarctica through USAP.  Thus, there is a deep connection between my teaching and research via the concept of exploring the unknown.  A long-time goal of mine has been to inspire my students to begin their own careers in science and in life with the same organization, mental discipline, and curiosity required of explorers.  I am thrilled to share that we are finally sending our first Poet to Antarctica.  Scout Mucher, who was my student in INTD290, was inspired to go, and we met to go over the application process.  Scout was hired as a contractor to help run USAP operations in McMurdo station on Ross Island.  Scout has now PQ'd (physically qualified), and is slated to begin the journey when the sun rises this Fall!  We are so proud of Scout.
\\
\vspace{0.25cm}
For my service to Whittier College for the 2021-2022 academic year, I joined the Educational Policy Committee (EPC).  I had been in discussions with Prof. Rehn about joining the Whittier Scholars Program (WSP) advisory board.  Near the end of Spring 2021, I was asked to serve on EPC, and I answered the call.  The major task for EPC was to generate consensus around a revised course system proposal to be brought before the full faculty by the end of the year, in light of proposed changes to faculty load.  While discussions of the course system and faculty load proposals continued, we also completed a long list of other tasks.  These included studying and approving changes to ten major programs and changes to programming within the Center for Engagement with Communities (CEC), raising student per-semester credit limits, revising the definition of a credit hour, revising the new course proposal form, revising handbook language pertaining to syllabi, and to study the use of ``tracks'' or ``emphases'' within major programs at Whittier College.
\\
\vspace{0.25cm}
I led a sub-committee dedicated to the study of tracks and emphases, and our goal was to develop a common language for tracks across campus.  I framed the task by comparing ``tracks'' and ``options'' to \textit{partitions}, as a computer hard drive is partitioned.  We examined data from DegreeWorks to accurately determine the number of partitions per program.  I found that, on average, major programs at Whittier College have two partitions per major. Rounded to integers, the natural sciences tend to have three partitions, while the social sciences and humanities tend to have two.  The variances, however, are large.  We used this diverse data set to formulate a survey sent to department chairs.  The data set and chair responses will be used to formulate policy for tracks and options within majors.  Students should find it useful to have a common language describing options within majors to improve their understanding of the curriculum.
\\
\vspace{0.25cm}
With regards to the course system, I tried my very best to aid in the discussions by thinking through the technical implications of the proposal.  One example of how I advanced the discussion was to offer a compromise regarding the maximum number of courses per semester.  One one hand, some wanted to limit first-year students to just four courses per semester, and require all students to obtain permission to take five courses in any semester.  On the other hand, some objected to any restriction.  I describe my compromise idea that we adopted into the final proposal in Sec. xxx.  I hope this demonstrates for you that I can help build consensus, even when the task is technically challenging.  I give much credit to my colleagues on EPC, and especially co-chairs Profs. Camparo and Householder, for setting such a good example.  In Sec. xxx, I describe ways in which I might be of service to Whittier College in the future, given my reflections on my accumulated committee service.  I often find ways to add weight to technical policy discussions by writing code and crunching numbers.  Thus, I propose to help with institutional research, and provide a few ideas in Sec. xxx.  Finally, there is always a special place in my heart for interdisciplinary research, so I have joined the WSP Council and am working with a new WSP advisee.
\\
\vspace{0.25cm}
In Sec. xxx I describe my accomplishments in advising and mentoring.  In our last communication, you gave concrete suggestions for my writing and reflection in this area.  Specifically, you asked about my philosophical approach to advising and mentoring.  You shared a concern that my advising leads students down a rigid path, and that creating digital profiles using services like LinkedIn narrows their path.  You also shared that you would prefer a self-reflective approach to the advising and mentorship section.  I have responded by reflecting on my advising and mentorship more succintly than my previous PEGP report.  I employ a philosophical perspective that motivates my students to take charge of their future in a way that is useful for them and in alignment with their values and interests.  I have also included tangible results of successful student outcomes, such as internships, publications, and fellowships gained by my advisees.  In Sec. xxx I discuss how I am managing first-year advisees again this year, and my new section of INTD100.
\\
\vspace{0.25cm}
To your point about rigidity, I'm surprised that the counter-examples I provided in my last report did not ameliorate your initial reaction.  I do maintain flexibility in my approach to advising, and I gave an example as proof.  I had an advisee who initially decided on majoring in physics, but realized in the midst of our discussions that his path lies in a different direction.  After appropriate reflection, we decided to change his major to Digital Art and Design (see Sec. 5.2.1 of my previous PEGP).  In Sec. xxx, I share how the decision tree I provided (also in Sec. 5.2 of my previous PEGP) actually reflects the thinking of many majors in the physical sciences.  I have infused this decision tree with more detail about the diverse paths my students take towards their career goals.  I also point out in Sec. xxx that our recent curricular discussions include the creation of digital portfolios and encourage sharing senior theses in Poet Commons.  Our advisees will use these to draw connections between projects at Whittier College and their future plans.  Tools liked LinkedIn are a widely accepted mechanism to facilitate sharing work.  I encourage students to use all these tools to take control of their path forward. Finally, I note that my WSP recruitment and WSP advisory board participation serve as evidence that my advising and mentorship is not rigid, but encourages students to explore interdisciplinary fields.  In Sec. xxx, I describe the work of my current WSP student, and my plans with future WSP students.
\\
\vspace{0.25cm}
In Sec. xxx, I describe a project regarding equity and inclusion within the natural sciences at Whittier College.  Back in 2017, I had the idea to create an app that would facilitate inclusion in introductory physics courses infused with machine learning to tailor user experience.  I named the idea ``The Primer,'' in reference to a sci-fi novel in which a young woman aquires a tool that accelerates her education through a narrative tailored to her background.  The visual environment and narrative of the app was to be designed by diverse Whittier College undergraduates to make the experience as inclusive as possible.  I finally found time to encapsulate this idea into an internal DEI grant through IDC.  The IDC members suggested I attend three workshops on inclusivity in introductory STEM courses given by the Cottrell Scholars Network.  In these workshops, we learned about current psychological research within the field of inclusion in introductory STEM courses.  One of my advisees in computer science has (in my humble opinion) the right experience and cultural background to help me create and test the code for the app this year.  We will recruit digital designers this Fall from within the Whittier community to develop the digital storytelling aspect of the tool.  We hope everyone will enjoy the results!
\\
\vspace{0.25cm}
My friends, it is my hope to be granted tenure at Whittier College.  I hope that the effort and passion I have given to our institution for the past several years merits this distinction.  I have created and taught new courses that students have enjoyed and found useful, including liberal arts courses that fused my research interests in the natural sciences with social issues and history.  I have demonstrated tangible successes in the areas of scholarship of discovery and application, and the the future holds several exciting research avenues to explore with our students.  I have served on multiple committees that have tackled complex issues ranging from student admissions data to the course system proposal.  I have continued to be a resource as a first-year advisor to many new Whittier Poets, and I have a track record of leading majors in the physical science to successful career outcomes.  All the while, I have done my best to incorporate your suggestions made with professional candor into my work.  I speak for myself and my department when I say that we are grateful for the work that you do, and we trust in your insight and wisdom.
\\
\vspace{0.25cm}
Sincerely, \\
Jordan C. Hanson \\
Assistant Professor, Dept. of Physics and Astronomy
\end{document}
