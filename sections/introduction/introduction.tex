\documentclass[../../main.tex]{subfiles}
 
\begin{document}

Dear Friends,
\\
\vspace{0.15cm}
I have compiled a report on my progress as an instructor, scholar, steward, and advisor for Whittier College during the period of 2021-2022.  The following is a reflection on the development of my educational and scholarly practices, and the service I have performed for the College as a mentor, advisor, and committee member. I strive to perfect my teaching abilities, and I am pleased to report that my students are learning and growing at Whittier, and achieving success in the professional world.  In our last communication, after my second major PEGP report (delayed to the fifth year due to the pandemic), you asked me to reflect on my pedagogical practices.  In particular, you suggested four concrete topics on which I could reflect.  First, you asked me to describe my interpretation of the learning process.  Second, you asked me to describe how I incorporate tools and practices in my courses.  Third, you asked me to describe how the tenets of my teaching philosophy help us to achieve the learning objectives that we set for our courses.  Finally, you asked me to focus on the \textit{why} behind specific teaching decisions as opposed to the \textit{how}.  You also posed similar questions about the \textit{learning focuses} that I have provided in the past.
\\
\vspace{0.25cm}
With these four simple questions in mind, I have taken what I think is the most straightforward and concise approach to the structure of my teaching philosophy in Sec. xxx.  I reflected on what practices I actually use most often in my teaching, and answered each of the four questions for all practices.  This exercise has been useful and enlightening, as it has encouraged me to think carefully about how the central principles of \textit{order} and \textit{shared meaning} are reflected in the subject of physics, and the instruction of physics.  It is my hope that this exercise provides you with useful insight into modern physics instruction.  I have also reflected on the \textit{learning focuses} I have provided in the past.  These ideas were derived originally from my colleagues in my department, but I have since modified them and made them my own.  In reflecting on how I conduct my courses, I have come to the conclusion that the learning focuses simply guide my course creation and course content selection.  How I conduct my courses, and the \textit{why} behind specific teaching decisions I make are driven primarily by the tenets of my teaching philosophy.
\\
\vspace{0.25cm}
Turning to my scholarship, I have many new and exciting accomplishments to share with you in Sec. xxx.  Three recent experiences come to mind as examples.  First, I have finally published in \textit{Physical Review D,} the flagship peer-reviewed journal in my field by the American Physical Society (APS)\footnote{J.C. Hanson and R. Hartig. ``Complex analysis of Askaryan radiation: a fully analytic model in the time domain.'' Phys. Rev. D \textbf{105}, 123019 (2022).}.  As far as I can tell, I am the first professor at Whittier College to achieve this.  This publication was the culmination of two years of work with an undergraduate student who has become a dear friend.  This result marks the first time a professor from my department has published in one of the \textit{Physical Review} journals in the last 16 years\footnote{See, for example, S. Zorba \textit{et al}. ``Fractal-mound growth of pentacene thin films.'' Phys. Rev. B \textbf{74}, 245410 (2006).}.  The piece provides the first fully analytic model of Askaryan radiation.  I hope to make clear why the results represent a significant contribution to my field, and how my undergraduate researcher helped me to improve and finish this work (see Sec. xxx, and also Sec. 3.3.2 of my previous PEGP).
\\
\vspace{0.25cm}
The second experience pertains to my radio-frequency (RF) engineering and radar research with the Office of Naval Research (ONR).  In 2021, I published a paper involving the computational electromagnetism (CEM) of radar design\footnote{J. C. Hanson.  ``Broadband RF Phased Array Design with MEEP: Comparisons to Array Theory in Two and Three Dimensions.'' Electronics Journal \textbf{10}, 415 (2021).}.  This work was ranked Top 10 Most Notable Articles in Electronics Journal for six months.  The work caught the attention of CEM experts from four different countries who each contacted me for advice and collaboration.  This Summer, I was invited to speak at a CEM conference held at MIT.  I gave a 45 minute lecture on open-source RF CEM design alongside colleagues from MIT, Google, Georgia Tech, Stanford, and BYU.  It was an incredibly meaningful moment in my career.  I took the opportunity to promote the mission and values of Whittier College to our peers at this institutions.  Having received an ONR Summer Faculty Research Program (SFRP) grant for the third year in a row, I am eligible for ONR Senior Fellowship in Summer 2024 after a mandatory one-year break.  For the past two years, our ONR partners at NSWC Corona Division have granted us money and precision RF equipment to boost the engineering research experiences of our students.  Based on this fruitful collaboration, I'm happy to share that they would like to form an Educational Partnership Agreement (EPA) with Whittier College.  NSWC Corona forms EPAs with colleges througout Southern California in order to strengthen undergraduate engineering research, and to recruit engineering talent.  Including Whittier College students in this endeavor will be wonderfully beneficial for our students' career development.
\\
\vspace{0.25cm}
The third and final experience is related to both my scholarship and teaching.  In Fall 2019, I taught INTD255, entitled ``Safe Return Doubtful: History and Current Status of Modern Science in Antarctica'' (CON2).  In Spring 2021, I taught INTD290, entitled ``A History of Science in Latin America'' (CON2,CUL3).  I concluded these courses with material related to the connection between modern scientific endeavors by peoples of diverse cultures and exploration literature.  I showed the students how it is possible to travel to Antarctica through the United States Antarctic Program (USAP).  As part of my research with the IceCube Gen2 collaboration\footnote{See, for example, \url{https://icecube.wisc.edu/}.}, I have conducted research expeditions to Antarctica through USAP.  Thus, there is a deep connection between my teaching and research through the concept of exploring the unknown.  A long-time goal of mine has been to inspire my students to begin their own careers in science and in life with the same organization, mental discipline, curiosity, and confidence required of any explorer.  I am thrilled to report that we are finally sending our first Poet undergraduate to Antarctica.  Scout Mucher, who was my student in INTD290, was inspired to go, and we met to go over the application process in detail.  Scout was hired as a contractor to help run USAP operations in McMurdo station, our flagship base on Ross Island.  Scout has now PQ'd (physically qualified), and is slated to begin work there when the sun rises this Fall!  I am so proud of Scout, who will become the first Whittier Poet to set foot on Antarctic shores.
\\
\vspace{0.25cm}
For my service to Whittier College for the 2021-2022 academic year, I joined the Educational Policy Committee (EPC).  I had been in discussions with Prof. Rehn about joining the Whittier Scholars Program (WSP) advisory board.  Near the end of Spring 2021, I was called to first serve for a year on EPC, and I answered the call.  The major task for EPC was to generate consensus around a revised course sytem proposal to be brought before the full faculty by the end of the year, in light of proposed changes to faculty load.  While discussions of the course system and faculty load proposals continued, we also completed a long list of other tasks.  These included studying and approving changes to ten major programs and changes to programming within the Center for Engagement with Communities (CEC), raising student per-semester credit limits, revising the definition of a credit hour, revising the new course proposal form, revising handbook language pertaining to syllabi, and to study the use of ``tracks'' or ``emphases'' within major programs at Whittier College.  I led a sub-committee dedicated to the study of tracks and emphases, and our goal was to develop a common language for tracks across campus.  I framed the task by comparing ``tracks'' and ``options'' to \textit{partitions}, as a computer hard drive is partitioned.  We examined data from Whittier College websites and DegreeWorks to accurately determine the number of partitions per program.  We used this diverse data set to formulate a survey we sent to department chairs.  The data set and chair responses will be used to formulate common policy surrounding how our major programs are partitioned.  One example of how I helped to advance the discussion of the course system proposal was to offer a compromise between two positions regarding the maximum number of courses a new student could take.  We ultimately adopted this compromise into the final proposal.  Though our course system proposal task was challenging, we worked together find common ground based on sound data.
\\
\vspace{0.25cm}
Finally, in Sec. xxx I describe my accomplishments in advising and mentoring.  In our last communication, you made concrete suggestions for my writing and reflection in this area.  Specifically, you are asking in general about my philosophical approach to advising and mentoring.  You shared a concern that my advising leads students down a rigid path, and that creating digital profiles using services like LinkedIn narrows their outlook.  You also shared that you would prefer a self-reflective approach to the advising and mentorship section, rather than a step-by-step guide through my processes and outcomes.  I have responded by reflecting on my advising and mentorship much more succintly than my previous PEGP report.  I have included relevant details that both show progress in student outcomes and that the philosophical perspective I employ motivates my students to take charge of their future in a way that is useful for them and in alignment with their values and interests.
\\
\vspace{0.25cm}
In Sec. xxx, I will argue that the decision tree I envi

(a) Armed Services and Govt. Internships, Andrew Householder, Natasha Waldorf, Adam Wildanger, Dane Goodman (volunteering) (b) Ondrasik-Groce Fellowship, Raymond Hartig (c) WSP and Jackson Diamond, and future COSC+WSP connections (d) Matthew Buchanan Garza, and the IDC grant for PhET creation (e) advising and mentoring first years.

If i was so rigid in my advising, why would I encourage folks to join WSP?  WSP is all about thinking productively about a topic from multiple perspectives, creating a path for your own future through development of the educational process, and independently executing an original idea. 

Why not LinkedIn?  In our discussions of a new curriculum, we propose to ask them to create a digital portfolio of projects they can later present future colleagues.  How is this different?

I literally said this ``Sometimes the interests of the student align with my research, and sometimes they do not. I
advise them nonetheless, and do my best to meet the student where they are to guide them forward.'' And yet you somehow arrived at the conclusion that my practice is too rigid.  Go figure.

Did you not understand the story of my advisee Wyatt?  We literally changed his major from STEM to humanities and he felt that it was the right path.
\end{document}
