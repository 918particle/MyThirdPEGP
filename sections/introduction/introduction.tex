\documentclass[../../main.tex]{subfiles}
 
\begin{document}

Dear Friends,
\\
\vspace{0.25cm}
I have compiled a report on my progress as an instructor, scholar, steward, and advisor for Whittier College during the period of 2021-2022.  I am pleased to report that my students are growing intellectually and achieving professional success.  After my second major PEGP report (delayed to the fifth year due to the pandemic), you asked me to reflect on my pedagogical practices.  In particular, you posed four questions I could use to discuss my teaching philosophy: (a) describe my interpretation of the learning process, (b) describe how I incorporate tools and practices in my courses, (c) describe how the tenets of my teaching philosophy help us to achieve learning objectives, and (d) focus on the \textit{why} behind specific teaching decisions.
\\
\vspace{0.25cm}
I have answered each question (a)-(d) for each tenet of my teaching philosophy in Sec. \ref{sec:teaching_philosophy}, after reflecting on what teaching practices I use most often.  This exercise has been useful and enlightening.  It is my hope that it provides you with useful insight into modern physics instruction.  I have also concluded that the \textit{learning focuses} I have shared previously serve as guides to my course creation and course content selection.  These ideas were derived originally from my colleagues in my department, and I have modified them and made them my own.  Though my teaching practices continue to generate positive student feedback, I have also identified areas of courses that need to be adjusted.
\\
\vspace{0.25cm}
I have new and exciting sholarship to share with you in Sec. xxx.  I highlight three recent experiences as examples.  First, I have finally published in \textit{Physical Review D,} the flagship peer-reviewed journal in my field by the American Physical Society (APS)\footnote{J.C. Hanson and R. Hartig. ``Complex analysis of Askaryan radiation: a fully analytic model in the time domain.'' Phys. Rev. D \textbf{105}, 123019 (2022).}.  I am the first professor at Whittier College to achieve this.  This publication was the culmination of two years of work with a student who has become a dear friend.  This result marks the first time our department has published in a \textit{Physical Review} journal in 16 years\footnote{See, for example, S. Zorba \textit{et al}. ``Fractal-mound growth of pentacene thin films.'' Phys. Rev. B \textbf{74}, 245410 (2006).}.  In the article, my student and I offer the first fully analytic model of Askaryan radiation.  The results represent a significant contribution to my field (see Sec. xxx, and also Sec. 3.3.2 of my previous PEGP).
\\
\vspace{0.25cm}
The second experience pertains to my radio-frequency (RF) engineering and radar research with the Office of Naval Research (ONR).  In 2021, I published a paper involving the computational electromagnetism (CEM) of radar design\footnote{J. C. Hanson.  ``Broadband RF Phased Array Design with MEEP: Comparisons to Array Theory in Two and Three Dimensions.'' Electronics Journal \textbf{10}, 415 (2021).}.  This work was ranked Top 10 Most Notable Articles in Electronics Journal for six months, and researchers from four countries contacted me to collaborate.  This Summer, I was invited to speak at a CEM conference held at MIT, alongside colleagues from MIT, Google, Georgia Tech, Stanford, and BYU.  It was an incredibly meaningful moment in my career.  I took the opportunity to promote the mission and values of Whittier College.  Having received three ONR Summer Faculty Research Program (SFRP) grants, I am eligible for ONR Senior Fellowship.  Our ONR partners at NSWC Corona Division have granted us money and precision RF equipment to boost engineering research for our students.  Based on this fruitful collaboration, I'm happy to share that NSWC Corona would like to form an Educational Partnership Agreement (EPA) with Whittier College.  The EPA will be wonderfully beneficial to our students (Sec. \ref{sec:naval_research}).
\\
\vspace{0.25cm}
The third experience is related to both my scholarship and teaching.  In Fall 2019, I taught INTD255, entitled ``Safe Return Doubtful: History and Current Status of Modern Science in Antarctica'' (CON2).  In Spring 2021, I taught INTD290, entitled ``A History of Science in Latin America'' (CON2,CUL3).  I concluded these courses with material related to the connection between modern scientific endeavors by peoples of diverse cultures and exploration literature.  I showed the students how it is possible to participate in the United States Antarctic Program (USAP).  As part of my research\footnote{See, for example, \url{https://icecube.wisc.edu/}, and Secs. 3.1 - 3.3 of my prior PEGP.}, I have conducted expeditions to Antarctica through USAP.  There is a deep connection between my teaching and research via the concept of exploring the unknown.  A long-time goal of mine has been to inspire my students to begin their own professional endeavors with the same organization, mental discipline, and curiosity required of explorers.  I am thrilled to share that we are sending our first Poet to Antarctica.  Scout Mucher, who was my student in INTD290, was inspired to go.  I helped Scout qualify as a contractor serving USAP operations in McMurdo station on Ross Island.  Scout has now PQ'd (physically qualified), and will journey South when the sun rises this Fall!
\\
\vspace{0.25cm}
For my committee service in AY 2021-2022, I joined the Educational Policy Committee (EPC).  I had been in discussions with Prof. Andrea Rehn about joining the Whittier Scholars Program (WSP) council.  I agreed to serve on EPC in 2021-2022 because new members were needed, and I put my WSP plans on hold.  The major task for EPC was to generate consensus around a revised course system proposal, in light of proposed changes to faculty load.  While discussions of the course system and faculty load proposals continued, we also completed many other tasks.  These included approving changes to ten major programs and changes to the Center for Engagement with Communities (CEC), raising student per-semester credit limits, revising the definition of a credit hour, and to study the use of ``tracks'' or ``emphases'' within major programs at Whittier College.
\\
\vspace{0.25cm}
I led a sub-committee with a goal to develop a common language for tracks across campus.  I framed the task by comparing ``tracks'' and ``options'' to \textit{partitions}, as a computer hard drive is partitioned.  We found that, on average, major programs at Whittier College have two partitions per major. Rounded to integers, the natural sciences tend to have three partitions, while the social sciences and humanities tend to have two.  We used this diverse data set to formulate a survey sent to department chairs.  The data set and chair responses will be used to inform policy for tracks and options within majors.  Having a common language describing options within majors will hopefully help first-generation students better understand the curriculum.
\\
\vspace{0.25cm}
With regards to the course system, I tried my very best to aid in the discussions by thinking through the technical implications of the proposal.  I offered a compromise regarding the maximum number of courses per semester that was eventually adopted into the final proposal (Sec. \ref{sec:committee_service}).  I hope this demonstrates for you that I can help build consensus, even when the task is challenging.  I give much credit to my colleagues on EPC, and especially co-chairs Profs. Camparo and Householder, for setting a good example.  In Sec. \ref{sec:committee_service}, I propose a few ways in which I can serve Whittier College in the future.  Finally, there is always a special place in my heart for interdisciplinary research, so I have joined the WSP Council and am working with a new WSP advisee (Sec. \ref{sec:committee_service}).
\\
\vspace{0.25cm}
In Sec. xxx I reflect on my advising and mentoring.  You have indicated you are concerned that my advising leads students down a rigid path.  You also shared that you would prefer a self-reflective approach to the advising and mentorship section.  I have responded to these questions in Sec. xxx.  To your point about rigidity, I do not know why the counter-examples to rigidity I provided in my last report did not resonate.  I shared my experience with an Physics advisee who, after a period of discernment with me, switched to Digital Art and Design (see Sec. 5.2.1 of my previous PEGP).  Philosophically, I motivate my students to take charge of their future in a way that aligns with their values and interests.  Our new curricula will include digital portfolios, and our advisees will use these tools and social media tools like LinkedIn to draw connections between projects at Whittier College and future plans that are as diverse as the students themselves.  My service to WSP majors also leads students down interdisciplinary paths.  In Sec. xxx, I also reflect on my tangible student successes, such as internships, publications, and fellowships.
\\
\vspace{0.25cm}
In Sec. xxx, I describe my new DEI project idea.  I had an idea in 2017 to create an app that would facilitate inclusion in introductory STEM courses infused with machine learning to tailor user experience.  The visual environment and narrative of the app was to be designed by diverse Whittier College undergraduates to make the experience as inclusive as possible.  I finally found time to encapsulate this idea into an internal DEI grant through IDC.  The IDC members suggested I attend three workshops on inclusion in introductory STEM courses given by the Cottrell Scholars Network.  I participated in all three of these workshops, and gained useful insights to apply to the project.  One of my advisees in computer science has volunteered to lead the app design.  We will recruit students from within the Whittier community to develop the digital storytelling aspect of the app.  We hope everyone will enjoy the results!
\\
\vspace{0.25cm}
My friends, it is my hope to be granted tenure at Whittier College.  I hope that the effort and passion I have given to Whittier College has made a positive impact, and that I have earned a place here.  I have created and taught new liberal arts and physical science courses that connected my research to social issues and history.  I have engaged in fruitful scholarship with my students, and several exciting research pathways have opened for us.  I have served committees that have tackled complex issues and built consensus.  I have served as a first-year advisor in three different semesters.  I have a track record of leading students to successful outcomes.  All the while, I have done my best to incorporate your suggestions into my work, and I am grateful for the work that you do.
\end{document}
