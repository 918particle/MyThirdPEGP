\documentclass[../../../main.tex]{subfiles}

\begin{document}
\label{sec:dei}

In Sec. \ref{sec:qsc}, I wrote about the challenges in introductory math instruction.  I would love to help improve results in introductory math courses, I have chosen to focus an internal DEI grant on introductory courses I teach personally.  I can control the pedagogy, and polish and test the central ideas within the proposal.  The inclusion and diversity committee (IDC) ultimately gave its blessing to my DEI grant\footnote{See supplemental material for details.}.  The objective is to develop a mobile application infused with machine-learning that will boost inclusion and belonging in foundational physics courses.  Based on examples developed at other institutions, we will create a customized tool that will strengthen our students’ skills and abilities. Diverse undergraduate experts in digital storytelling will include the voices, narrative themes, and imagery of the diverse students actually attending these courses. The result will be OER comparable to DuoLingo, but for physics.
\\
\vspace{0.15cm}
According to our institutional research, 25\% of all white male Whittier College students select major in STEM disciplines, but represent just 10\% of STEM majors at Whittier College (data selected for the period of 2019-2022). White men accounted for just 18\% of all majors in disciplines tied to engineering, 10\% of all KNS majors, and just 6.3\% of all biology majors. Introductory physics students are mostly biology and biochemistry majors who plan to attend medical school, KNS majors who plan to attend physical therapy school, and engineering students. Bolstering student success in these courses is intrinsically anti-racist, because this large group of students is on the pathway to join and diversify fields in medicine, biotechnology, and engineering.
\\
\vspace{0.15cm}
One major motivation for this proposal is to affirm the dignity of students of color and young women by providing them with a tool designed for them by myself and by their peers. There are two underlying problems, according to research, for those that struggle in introductory physics courses. (1) They perceive themselves to be less effective at science despite having the same grades as others, and (2) when they do struggle with physics concepts, they must be helped in a way that affirms their dignity. We propose to develop a tool that helps them automatically by showing them that (a) they are not alone in their anxiety about physics courses, and (b) peers have built tools designed to strengthen their skills. According to the social science research, differences in self-reported self-efficacy (1) vanish when one introduces (a).  My hope is that (2) will be addressed by (b) in the same iterative fashion that works in DuoLingo.  The program learns what concepts the students have mastered, and returns to those concepts less often than the ones the student needs to master.

\end{document}
