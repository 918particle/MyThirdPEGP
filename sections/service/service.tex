\documentclass[../../main.tex]{subfiles}
 
\begin{document}
\label{sec:service}

As Whittier College moves forward from the pandemic, service to our community has taken on new meaning.  We have given more of ourselves to sustain Whittier College than before.  In AY 2021-2022, Prof. Andrea Rehn recommended me for the Whittier Scholars Program council.  I was, however, asked to volunteer for the Educational Policy Committee, because new members were needed.  I decided to step up for Whittier and servce EPC.  In Sec. \ref{sec:committee_service}, I reflect on our accomplishments in EPC, and other committee service topics.  This Fall 2022 semester, I am serving as first-year advisor for the third time since I began teaching here.  I reflect on my INTD100 exeperiences in Sec. \ref{sec:first_year}.  I am also helping to form our new Quantative Success Center (QSC) (Sec. \ref{sec:qsc}).  Finally, I reflect in Sec. \ref{sec:dei} on a DEI service project I have begun to advance inclusion in introductory STEM courses.

\section{Committee Service}

\begin{flushleft}
\subfile{committee/committee}
\end{flushleft}

\section{First Year Orientation}

\begin{flushleft}
\subfile{orientation/orientation}
\end{flushleft}

\section{The Quantitative Success Center}

\begin{flushleft}
\subfile{qsc/qsc}
\end{flushleft}

\section{Inclusivity in Introductory STEM Courses}

\begin{flushleft}
\subfile{dei/dei}
\end{flushleft}

\section{Summary Reflection on Service Projects}

\begin{flushleft}
\subfile{summary/summary}
\end{flushleft}

\end{document}
