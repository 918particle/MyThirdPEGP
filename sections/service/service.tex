\documentclass[../../main.tex]{subfiles}
 
\begin{document}
\label{sec:service}

As Whittier College moves forward from the pandemic that began two years ago, service to our college community has taken on new meaning.  Many of us have given more of ourselves to sustain Whittier College.  In AY 2021-2022, I planned to serve the Whittier Scholars Program Council after Prof. Andrea Rehn agreed to recommend me.  I was, however, asked to volunteer for the Educational Policy Committee.  The EPC needed new members to continue the work, so I decided to step up for for Whittier and put my plans on hold.  In Sec. \ref{sec:committee_service}, I reflect on our accomplishments in EPC, and other committee service topics.  This Fall 2022 semester, I am again serving first year students through the INTD100 and orientation program.  I reflect on my INTD100 exeperiences in Sec. \ref{sec:first_year}.  In Spring 2022, I was asked to help with the formation of our new Quantative Success Center (QSC) by helping to write the job description for the QSC Director.  This semester, I will help with the candidate search (Sec. \ref{sec:qsc}).  Finally, I am pleased to share with you in Sec. \ref{sec:dei} my thinking behind a small grant proposal I submitted to the Inclusion and Diversity Committee (IDC) to advance inclusion within our introductory STEM courses.  The IDC has approved it and we begin work this semester.

\section{Committee Service}

\begin{flushleft}
\subfile{committee/committee}
\end{flushleft}

\section{First Year Orientation}

\begin{flushleft}
\subfile{orientation/orientation}
\end{flushleft}

\section{The Quantitative Success Center}

\begin{flushleft}
\subfile{qsc/qsc}
\end{flushleft}

\section{Inclusivity in Introductory STEM Courses}

\begin{flushleft}
\subfile{dei/dei}
\end{flushleft}

\end{document}
