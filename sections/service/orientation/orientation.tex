\documentclass[../../../main.tex]{subfiles}

\begin{document}
\label{sec:first_year}

This semester marks the third time I have volunteered to become advisor to first year students, and the second time I have taught College Writing Seminar (INTD100).  My course topics include the philosophy of science and how to identify and contrast \textit{pseudo-science} and \textit{scientific denialism} with real scientific thought.  To help improve our students' writing mechanics, I have created exercises on conciseness, hierarchy of details, elimination of ambiguous words or phrases, and overall organization and planning of writing.  Only one of my advisees has written a paper longer than ten pages, so this last topic is of special importance for articulating the results of scientific analysis or long-term projects.  I will help my students polish their technical writing skill, while stimulating conversation through readings on the philosophy of science.
\\
\vspace{0.15cm}
For our reading, I have chosen \textit{The Scientific Attitude}, by Lee McIntyre (MIT Press).  I learned of this book from Prof. Piner, who used it for his INTD100 section.  The book is meant for a wide audience of college level readers interested in the defense of reason and scientific thought.  There is a wave of science denialism and pseudo-science sweeping through our culture, and I felt this book would help our students develop the intellectual muscle to confront it.  The author begins with the \textit{demarcation problem} in the philosophy of science, which is an attempt to articulate the difference between science and pseudo-science.  From there, misconceptions about the truth of scientific claims are discussed, followed by defenses of the truth in claims within fields like physics, medicine, and the social sciences.  Given that Whittier College is a liberal arts school, my students will learn and discuss the logical framework of claims made by the social sciences, and how they fit into the larger framework of science.  Our discussions will conclude with the logical flaws of pseudo-scientific claims made by vaccine and climate skeptics, and creationists.

\end{document}
