\documentclass[../../../main.tex]{subfiles}

\begin{document}
\label{sec:committee_service}

Here I reflect primarily on my service within the Educational Policy Committee.  Since I have been accepted onto the Whittier Scholars Program (WSP) Council, I also reflect on our plans to enhance WSP.  Finally, I offer some ideas on how I can serve the college in the future through institutional research.

\subsection{Educational Policy Committee}

Though the charge to EPC in AY 2021-2022 contained many projects and goals, none required as much time and energy as finishing the course system proposal.  My honest reflection is that this work was challenging.  I trust you to read what I've written within the spirit of honest reflection and service rather than the outcome for the course system.
\\
\vspace{0.25cm}
When I joined EPC, I learned that the course system has been proposed various times in the last 20 years.  I asked about the original reasoning, and why my colleagues had spent such time advocating for it.  Having served at four institutions of higher education and three Catholic parishes, this debate felt similar to other entrenched debates I have encountered.  The original reasons have become blurred, those whose minds were once open have become more solidified, and new factors complicate the discussion.  Examples of the latter are the evolving curriculum and the faculty workload.  Though the workload is not strictly an issue of educational policy, it is so important to our faculty that it had to be included.  Entrenched debates lead to large amounts of accumulated time spent in meetings discussing the proposal.  Time \textit{itself} eventually becomes a motivation: ``It would be a shame not to pass this ... we've already spent so much time working on it.''  Given my experiences working at other institutions, these traits are the hallmarks of entrenched debate.
\\
\vspace{0.25cm}
In a lecture entitled ``Two Sacred Incompatible Values at American Universities,'' \cite{haidt} Prof. Jonathan Haidt (NYU Stern School of Business) defines \textit{motivated reasoning} in the following terms.  When someone wants to believe something, they ask internally ``\textit{Can} I believe this?'' and they search for a single fact to support the claim.  When someone does not want to believe something, they ask internally ``\textit{Must} I believe this?'' and they search for a single fact to reject it.  Psychological studies suggest that people engage in motivated reasoning when considering for/against propositions with moral or practical implications for their lives.  In discussions, the reasons why members \textit{wanted} or \textit{did not want} to support the course system became clear.  Some favored it because they felt it would help financially.  Others supported it because they felt it would help their advisees graduate given the loss of January term.  Some did not support it because they felt it would hurt a major.  Some did not favor it because it seemed to grant degrees to students who do not complete enough academic work.  
\\
\vspace{0.25cm}
At first, we were instructed to frame our discussions pedagogically, no matter what our internal motivated reasoning was.  One pedagogical task was to define a course.  Courses are as diverse as human thought: some are lighter, some more intense, more narrow, or broadly focused.  The goal was consensus for a solid course definition, and for the number of courses required to graduate.  We were instructed to use 32 courses as the industry standard.  A student could take courses that corresponded formerly to 3 credit-hours for a degree with 96 credit-hours, compared to the original 120 to graduate under the credit system.  The same calculation with 4 credit courses leads to 128 credits, and performing a weighted average given what students take still produces results lower than our current requirement.  The average is weighted because the ratio of 3 credit to 4 credit courses depends on the student.  It became necessary to compare the workload for 3 and 4 credit courses.  I had assumed that 4 credit courses were more work, but I learned it can depend on how work outside of class is counted\footnote{For example, a seminar that meets once per week could require 10 hours per week of reading.}.  We settled on a definition of about 150 minutes of class time, plus 7-9 hours of work outside of class\footnote{This is the definition for a ``full'' course, but four ``cumulative'' courses correspond to a full course.}.  This compromise felt, to me, like one born out of necessity to make progress.
\\
\vspace{0.25cm}
Given our experience with the course definition discussions, we took several logical steps to ensure students will graduate smoothly and on time under either system.  We raised the students' per semester credit limit to 17, and worked together to secure the approval of the faculty.  We also updated the definition of a credit-hour, after reviewing material from 4-year liberal arts colleges within and outside of WASC regarding total credits to graduation given their credit-hour definitions.  We began a sub-committee project, which I led, to understand and classify ``tracks,'' ``emphases,'' and ``options,'' within department majors.  The goal for that sub-committee project is to standardize language surrounding major programs in order to help students understand their options and plan for graduation.  I reflect on the outcome of this project below, after I complete my reflection on the course system proposal.
\\
\vspace{0.25cm}
Another component of the proposal was the proposed cap on courses per major.  Since a bedrock goal of the system is graduation in 4 years, there is some upper limit on the number of courses students can take.  With a finite number of courses to select, it would be unorthodox for a liberal arts college student to concentrate course selection in a single subject.  Some pressed for just 11-12 courses per major.  One third of the courses would be in the major, one third would be liberal arts requirements, and one third would be for ``exploration.''  There is a simple counter-argument: some major programs cannot exist with 11-12 courses.  Consider the KNS major with ``pre-physical therapy emphasis''\footnote{In 2021, 10.5\% of all students chose this major.  Compare this number to 14\% for BSAD.}.  There are two ways to estimate the number of courses: count the courses currently required, or total the credits and divide by an appropriate denominator.  The KNS department lists 21 courses required.  We also find about 21 courses if we total the credits and divide by 3.5 credits per course\footnote{This major contains 14 courses worth 4 credits each.}.  Thus, capping majors at 11-12 courses would eliminate one of our most popular majors.
\\
\vspace{0.25cm}
We did find a potential compromise: count only KNS courses (that is, the department of the major) towards the cap.  The total KNS courses in the major would stay under the cap (11), since the other courses come from CHEM, BIOL, MATH, PHYS, and PSYC.  This logic applies to any major requiring courses outside the major department.  Some EPC members eventually revealed their motivated reasoning behind supporting the cap on majors.  Some want to incentivize students to take courses in departments and subjects that need more students.  I consider my conscience well-formed on this point.  Though a central facet of a liberal arts education is experiencing diverse thought, we cannot make students take courses they don't wish to take.  The moral responsibility for course selection lies with the students and their families.  That being said, some appropriate cap must always be put in place, and even the current system has a credit-limit for majors.
\\
\vspace{0.25cm}
Consider one more example: ICS/Mathematics.  ICS/Math has experienced 950\% growth since 2018, and it has an average female-to-male ratio of 44\%.  The popularity and diversity of this major is an asset to Whittier.  Students are often unable to enroll in similar programs at larger schools due to overcrowding.  We have an opportunity to expand our undergraduate base by meeting student demand.  The compromise we devised above does not work, for this 17-course major is interdisciplinary between math and computer science and all 17 courses count.  Enforcing an 11-12 course cap significantly alters one of our fastest growing majors.  I also note that a female-to-male ratio of 44\% is more than double the ratio for all 3-2 Engineering majors over the last six years.  We should cherish and build on such diversity in the ICS/Math Program.  As a pre-tenured faculty member, I had to summon courage to voice my objection to the 11-12 course cap on majors.  It turns out that I was not alone, and we decided to table that part of the proposal..
\\
\vspace{0.25cm}
At one point, the discussion was steered back to financial reasoning.  We reviewed calculations that demonstrated a course system would save financial resources.  The average courses per student per semester would decline in the course system, with tuition remaining constant, yielding savings.  Though we were initially instructed to focus pedagogical reasoning, we were asked mid-year to reconsider financial reasoning.  There was a modest financial advantage to the course system over credit, but it arises students take fewer courses on average.  Some of us felt that reducing the courses per student per semester was akin to granting a degree for less work.  Alternatively, we can think of this as indirectly raising tuition by collecting the same tuition for fewer courses.  These are statistical conclusions drawn from the scenario we examined, which focused on averaging over hundreds of students.
\\
\vspace{0.25cm}
Towards the end of the year I helped resolve an impasse regarding the cap on the number of courses per semester students could take.  The original proposal restricted students to four full courses per semester (plus cumulatives).  Given my experience advising 3-2 Engineering, Physics, and ICS students, I believe this restriction would have lead to painful setbacks to graduation.  We examined results from other institutions that suggest a strong course load is correlated with student success.  Our own 2019 data indicates that 35\% of students took 15-16 credits with a GPA of 3.2, and 51\% of students took 13-14 credits with a GPA of 3.0.  From the data, allowing five courses poses less risk than initially imagined.  Two camps developed nonetheless: for caps, or against caps.  I devised a compromise: during the first semester, a cap of four courses, followed by the ongoing ability to take five with advisor approval.  Academic probation would remove the ability.  My past ESAC committee studies indicated a modest GPA decrease in transition between high school and the first semester of college (Sec. 4.1.1 of my prior PEGP).  The compromise solution checks the student GPA at the transition, and also respects the will of the students.  The compromise plan was adopted into the proposal.  Humbly, these are my honest reflections, and I believe my colleagues always had good intentions.  Throughout the year, we tried our best to discern what is best for our students.
\\
\vspace{0.25cm}
Finally, I mentioned above that I led a sub-committee to study the partioning of major programs through ``options,'' ``tracks,'' and ``emphases.''  I began by gathering data and compiling an executive summary (see Supplemental Material).  I chose the term ``partition'' as a neutral term for major sub-classification.  The average partitions per major across 27 majors Whittier College is $2.2 \pm 1.3$.  From website data, the divisional breakdown was $3.3 \pm 1.0$, $1.8 \pm 1.3$, and $1.9 \pm 1.2$ for the physical sciences, social sciences, and humanities, respectively (mean and standard deviation).  From data taken from DegreeWorks, the resuts are: $2.4 \pm 1.5$, $2.0 \pm 1.3$, and $1.9 \pm 1.1$, respectively.  Very few departments have just a single version of the major discipline.  Terms like track or emphasis are used to partition majors into sub-classified programs.  EPC considered that students could better understand the system if similar structures are common throughout the major programs within an overall curricular design.  The hope is that students can take better control of their educational planning.  For now, we have sent a departmental survey to department chairs, based on my executive summary.  We hope the survey will provide clarity such that the curriculum makes the most sense to first-generation students.

\subsection{The Whittier Scholars Program}

As I shared in my last PEGP report, I have already graduated one student through the Whittier Scholars Program.  This was a fantastic interdisciplinary experience.  Nicolas Bakken-French was an intrepid sophomore when he introduced himself to me, and he was interested in glaciers.  He had heard that at least part of my research involved Antarcticic science, and was curious if this connected to his plans.  He wanted to understand and document the disappearance of polar glaciers from a scientific and cultural perspective.  Our work sent him all over the world collecting data, gaining field experience and survival training, and photographs of ice fields and glaciers.  I was proud of the scientific and cultural report that we submitted in the end, and I felt that we produced a polished product.
\\
\vspace{0.25cm}
I have a new student in the Whittier Scholars Program, Jackson Diamond, who focuses on computer programming applications in medicine.  Jackson has been developing his innovation with me that is central to his final WSP project.  He is creating a position location system based on Bluetooth signal strengths and trilateration.  This would be useful for locating objects with Bluetooth tags with a precision of a few centimeters within a room or floor of a building.  Jackson shares that his first appliction of this technology would be patient tracking within a hospital.  Imagine a Bluetooth tag connected to a baby crib within a maternity ward that is integrated within a trilateration system.  The location of the baby could be remotely monitored for safety and security purposes.  I agreed to take on Jackson's program after his original advisor (Bill Kronholm) had to depart.  For me it was an easy service decision to make.  WSP needs new professors to step forward to maintain it, and it is a special privilege that we have such an interesting interdisciplinary program at our institution.
\\
\vspace{0.25cm}
Students like Jackson and Nicolas are why I have volunteered to serve on the WSP Council.  With Prof. Chihara now leading the council, our task this year is to streamline and update the program.  I look forward to working with Prof. Chihara by recruiting for WSP and enhancing the proposed interaction between INTD100 and WSP.  The basic idea is that some INTD100 sections would be integrated with introductory-level WSP courses for students who know they want to pursue interdisciplinary studies.  For my part, I will continue to serve as an INTD100 instructor over the years, and help strengthen the bond between the first-year experience and an introduction to WSP.  The goal, as Prof. Chihara says, is not necessarily to draw more students into WSP, but to show them the interdisciplinary nature of academia such that they choose a path for their major they find intriguing and impactful.  As part of this overall plan, I hope to strengthen ties between computer science as an interdisciplinary field and WSP.  Students like Jackson and Nicolas have shown me that there is still a hunger for a diverse liberal arts education among students in the physical sciences.  Through my advising and service, I hope my students and I will generate new ideas to serve our community.

\subsection{Future Proposals for Institutional Research}

I have noticed a common theme in my committee service roles.  I have the skills to help with institutional research, and I have noticed that some of our institutional research staff have retired.  For these reasons, I suggest a few ideas for ways I could serve Whittier College in the future.
\begin{enumerate}
\item In my writing, I often make use of the data from the dashboards created by Gary Wisenand that are available on Moodle.  Using these data, for example, I determined the growth rate and gender breakdown of the ICS/Math program.  These dashboards need to be updated with data from AY 2021-2022 and beyond.  Data projects like these would not be difficult given my experience level.  I am interested in questions like the demographic breakdown of introductory STEM courses, the participation of young women in STEM majors, and measuring the growth of the wide variety of major programs and the corresponding ``partitions.''
\item Since I served on ESAC for two years and EPC for one year (with ERC in between those), I have been exposed to many analyses of student data.  An idea for an interesting statistical analysis has been floating around in the back of my mind, related to boosting retention and graduation.  I hope to create a tool that visualizes the pathways students take through our major programs.  My hope is that the tool will help identify roadblocks to graduation on a statistical basis, and include end results like graduation and internships.
\end{enumerate}

\end{document}