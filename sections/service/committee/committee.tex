\documentclass[../../../main.tex]{subfiles}

\begin{document}
\label{sec:committee_service}

In this section I reflect primarily on my service within the Educational Policy Committee.  Since I have been accepted onto the Whittier Scholars Program (WSP) Council, I also reflect on our plans to enhance WSP.  Finally, I offer some ideas on how I can serve the college in the future through institutional research.

\subsection{Educational Policy Committee}

Though the charge to EPC in AY 2021-2022 contained many projects and goals, none required as much energy and thorough deliberation as finishing the course system proposal.  I must be honest when I share my reflections on these experiences: they were challenging.  I trust you to read what I've written within the context of honest reflection and service rather than the eventual fate of the course system proposal.
\\
\vspace{0.25cm}
When I joined EPC, I learned that people have been proposing a change to a course system for 20 years.  I asked about the original reasoning, and why my colleagues had spent large portions of their careers advocating for it.  Having served at a number of institutions\footnote{These include four institutions of higher education: UC Irvine, University of Kansas, The Ohio State University, and Whittier College, plus three councils of the Knights of Columbus at three Catholic parishes.}, I realized that the debate was similar to other entrenched debates I have encounted over the years.  The original reasons have become blurred, those whose minds were once open have become less so, and new factors complicate the discussion.  Examples of the latter are the evolving curriculum and the faculty workload.  Though these are not pedagogical issues, they are so important to faculty at Whittier College that they had to be included in some way.  Entrenched debates lead to large amounts of accumulated time spent in meetings discussing the proposal.  Time \textit{itself} eventually becomes part of the debate: ``It would be a shame not to pass this ... we've already spent so much time working on it.''  Given my experiences working at other Universities and within the Church, these traits are the hallmarks of debate that has become entrenched.
\\
\vspace{0.25cm}
In his lecture entitled ``Two Sacred Incompatible Values at American Universities,'' \cite{haidt} Prof. Jonathan Haidt (NYU Stern School of Business) defines \textit{motivated reasoning} in the following terms.  When someone wants to believe something, they ask internally ``\textit{Can} I believe this?'' and when someone does not want to believe something, they ask internally ``\textit{Must} I believe this?''  In the first scenario, someone already has a feeling as to why they favor a proposal, and they tend to search for one piece of factual evidence to support it.  In the second scenario, someone already has a feeling as to why they do not favor a proposal, and they tend to search for one piece of factual evidence to reject it.  Psychological studies suggest that most people think this way when presented with a for/against proposition.  In our EPC discussions, the reasons why members \textit{wanted} or \textit{did not want} to believe the course system would benefit Whittier College eventually became clear.  Some wanted it because they felt it would help our finances.  Some wanted it because they felt it would help their advisees graduate given the loss of January term.  Some did not want it because they felt it would hurt a major degree within their department.  Some did not want it because it seemed to grant degrees for less than the traditional amount of work.  The EPC was instructed, however, to frame our discussions along pedagogical lines, and in light of the faculty workload proposal that we received mid-year.
\\
\vspace{0.25cm}
One of our tasks was to define a standard course, even though courses can be as diverse as human thought.  Some courses are more intense, some more narrow or broadly focused.  We updated the definition of a credit-hour, and we reviewed material from 4-year liberal arts colleges within and outside of WASC regarding total credits to graduation given their credit-hour definitions.  The goal was consensus for the total number of \textit{courses} required to graduate, given a solid course definition.  Motivated reasoning lead each of us to different numbers, but justifying those numbers pedagogically proved challenging.  We were instructed to use 32 courses as the standard.  With a total of 32, a student could take courses that corresponded formerly to 3 credit-hours for a degree with the equivalent of 96 credits compared to the original 120 to graduate under the credit system.  A similar calculation with 4 credit courses leads to 128 credits for a degree, and performing a weighted average of the extremes still produces results that reduce student credit-hours relative to the former system.  The ratio of 3 credit courses to 4 credit courses depends on the student, and we examined those statistics as well.  At one point, the discussion focused on the work comparison for 3 and 4 credit courses.  Some felt they are the same amount of work, and some did not.  The details depend on counting work done outside class.  We took a number of steps
\\
\vspace{0.25cm}
Another facet of the proposal that required consensus was the cap on courses per major.  My understanding of the argument is the following: with a specified number of courses to graduate (32), most students will target that number, and few will deliberately exceed it.  The number of courses per major needs to be capped such that majors do not constitute a significant proportion of 32.  Whittier College is a liberal arts school, and granting a degree to someone who chose 80 \% of their classes from one subject would defeat the purpose.  Some EPC members were pressing for 11-12 courses as the maximum by mid-year.  The counter-argument is very simple: some major programs simply cannot exist with such a low number of courses.  A classic example is the KNS major with pre-physical therapy emphasis.  One can estimate the number of courses under the proposed system in two ways: count the courses currently required, or total the credits and divide by an appropriate denominator.  The KNS department website lists \textbf{21 courses required to graduate} with this emphasis.  We also find about 21 courses if we total the credits and divide by 3.5 credits per course\footnote{This major contains 14 courses worth 4 credits each.}.  
\\
\vspace{0.25cm}
The discussion then turned to financial reasoning.  We reviewed a series of calculations that demonstrated a course system with 32 courses to graduate would save the school some money.  The mechanism seemed to be that the average courses per student per semester would decline, with tuition remaining the same.

My sub-project, the survey.  Framing the issue of the course system, understanding it.  Mathematical analyses of the proposals: (a) financial implications (b) pedagogical implications (c) curricular implications (d) the compromise i offered that was accepted regarding maximum course loads

\subsection{The Whittier Scholars Program}

Another year, another advisee, acceptance to join the WSP board

\subsection{Future Proposals for Institutional Research}

Full utilization of the Tableau dashboards left by Gary Wisenand, growth of the ICS/Math major etc.

\end{document}
