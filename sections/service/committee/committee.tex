\documentclass[../../../main.tex]{subfiles}

\begin{document}
\label{sec:committee_service}

Here I reflect primarily on my service within the Educational Policy Committee.  Since I have been accepted onto the Whittier Scholars Program (WSP) Council, I also reflect on our plans to enhance WSP.  Finally, I offer some ideas on how I can serve the college in the future through institutional research.

\subsection{Educational Policy Committee}

Though the charge to EPC in AY 2021-2022 contained many projects and goals, none required as much energy and thorough deliberation as finishing the course system proposal.  I must be honest when I share my reflections on these experiences: they were challenging.  I trust you to read what I've written within the context of honest reflection and service rather than the eventual fate of the course system proposal.
\\
\vspace{0.25cm}
When I joined EPC, I learned that people have been proposing a change to a course system for 20 years.  I asked about the original reasoning, and why my colleagues had spent large portions of their careers advocating for it.  Having served at four institutions of higher education and three Catholic parishes, I realized this debate was similar to other entrenched debates I have observed at my previous institutions.  The original reasons have become blurred, those whose minds were once open have become more solidified, and new factors complicate the discussion.  Examples of the latter are the evolving curriculum and the faculty workload.  Though the workload is not strictly a pedagogical issue, it is so important to faculty at Whittier College that it had to be included.  Entrenched debates lead to large amounts of accumulated time spent in meetings discussing the proposal.  Time \textit{itself} eventually becomes part of the debate: ``It would be a shame not to pass this ... we've already spent so much time working on it.''  Given my experiences working at other institutions, these traits are the hallmarks of debate that has become entrenched.
\\
\vspace{0.25cm}
In his lecture entitled ``Two Sacred Incompatible Values at American Universities,'' \cite{haidt} Prof. Jonathan Haidt (NYU Stern School of Business) defines \textit{motivated reasoning} in the following terms.  When someone wants to believe something, they ask internally ``\textit{Can} I believe this?'' and they tend to search for a single fact to support the claim.  When someone does not want to believe something, they ask internally ``\textit{Must} I believe this?'' and they tend to search for one piece of evidence to reject it.  Psychological studies suggest that people engage in motivated reasoning when considering for/against propositions with moral or practical implications for their lives.  In our EPC discussions, the reasons why members \textit{wanted} or \textit{did not want} to believe the course system would benefit Whittier College eventually became clear.  Some wanted it because they felt it would help our finances.  Others wanted it because they felt it would help their advisees graduate given the loss of January term.  Some did not want it because they felt it would hurt a major degree within their department.  Some did not want it because it seemed to grant degrees to students who do not complete enough work.  
\\
\vspace{0.25cm}
No matter what our internally motivated reasons were, we had to frame our discussions pedagogically as we were instructed.  One of our tasks was to define a standard course.  Courses can be as diverse as human thought: some are lighter, some more intense, and some are more narrow or broadly focused.  The goal was a solid course definition, and consensus for the total number of \textit{courses} required to graduate.  Motivated reasoning leads us to different results, but justifying the numbers pedagogically is hard.  We were instructed to use 32 courses as the standard.  With a total of 32, a student could take courses that corresponded formerly to 3 credit-hours for a degree with 96 credit-hours, compared to the original 120 to graduate under the credit system.  The same calculation with 4 credit courses leads to 128 credits, and performing a weighted average of the extremes still produces results that are lower than we currently require.  The ratio of 3 credit courses to 4 credit courses depends on the student, and we examined those statistics as well.  We actually compared the workload for 3 and 4 credit courses.  I had assummed a 4 credit course is more work, but I learned the importance of defining and measuring work outside of class.  I found that most of us were relying on personal experiences to judge the workload outside of class.
\\
\vspace{0.25cm}
In response to the difficulty of developing consensus around courses, we took several logical steps to ensure students will graduate smoothly and on time under the credit system.  We raised the students' per semester credit limit to 17, and worked together to secure the approval of the faculty.  We also updated the definition of a credit-hour, after reviewing material from 4-year liberal arts colleges within and outside of WASC regarding total credits to graduation given their credit-hour definitions.  We began a sub-committee project, which I led, to understand and classify ``tracks,'' ``emphases,'' and ``options,'' within department majors.  The goal for that sub-committee project is to standardize language surrounding major programs in order to help students understand their options and plan for graduation.  I reflect on the outcome of this project below, after I share my reflection on the course system proposal.
\\
\vspace{0.25cm}
Another component of the proposal was the proposed cap on courses per major.  Since a bedrock goal of the system is graduation in 4 years, there is some upper limit on the number of courses even the most intrepid students can take.  With a finite number of courses to select, it would be unexpected for a student at a liberal arts college to chose 80 \% of their classes from one subject.  Some of us were pressing for a cap of 11-12 courses, with the thought being one third of the courses chosen for the major, one third core liberal arts requirements, and one third for ``exploration.''  The counter-argument is very simple: some major programs cannot exist with 11-12 courses.  Consider the KNS major with ``pre-physical therapy emphasis''\footnote{In 2021, 10.5\% of all students chose this major.  Compare this number to 14\% for BSAD.}.  I can imagine two ways to estimate the number of courses in the major: count the courses currently required, or total the credits and divide by an appropriate denominator.  The KNS department lists 21 courses required to graduate with this emphasis.  We also find about 21 courses if we total the credits and divide by 3.5 credits per course\footnote{This major contains 14 courses worth 4 credits each.}.
\\
\vspace{0.25cm}
Thus, capping majors at 11-12 courses would eliminate one of our most popular majors.  Our discussions did yield a potential compromise: counting only courses beginning with KNS (that is, the department of the major) towards the cap.  The total KNS courses in the major would stay under the cap (11), since the other courses come from CHEM, BIOL, MATH, PHYS, and PSYC.  This does not guarantee that students would take more courses in the humanities and social sciences outside of PSYC and KNS. Through our EPC discussions, it became clear to me that the motivated reasoning behind the cap on major courses was to encourage students to take courses in the humanities and social sciences that need more students.  I consider my conscience well-formed on this point, and I cannot ignore my own moral instincts.  While a central facet of a liberal arts college is that students receive a well-rounded education, we cannot make students take courses they don't wish to take.  The moral responsibility for course selection lies with the students and their families.  That being said, some appropriate cap must always be put in place, and even the current system has a credit-limit for majors.
\\
\vspace{0.25cm}
Consider one more example: ICS/Mathematics.  ICS/Math has experienced 950\% growth since 2018, and it has an average female-to-male ratio of 44\%.  The popularity and diversity of this major at Whittier College is an asset to us, and students are often unable to enroll in similar programs at larger schools due to overcrowding.  We have an opportunity to expand our undergraduate base by meeting the needs of students.  However, this major has 17 courses.  The compromise we devised above does not apply, for this major is interdisciplinary between math and computer science and therefore all 17 courses count towards the major.  Enforcing a cap to encourage students to take other courses would significantly alter one of our fastest growing majors.  I also note that a female-to-male ratio of 44\% is more than double the result for all 3-2 Engineering majors over the last six years.  Thus, we should cherish and build on such diversity in the ICS/Math Program.  Thus, I've shown how including a major cap as part of the course system proposal would have negatively impacted one of our most popular majors (KNS with pre-physical therapy), and one of our fastest growing ones (ICS/Math).
\\
\vspace{0.25cm}
At one point, the discussion of the course system proposal was steered back to financial reasoning.  We reviewed a series of calculations that demonstrated a course system with 32 courses to graduate would save financial resources.  The mechanism appeared to be that the average courses per student per semester would decline, with tuition remaining constant.  What we found strange was that we were initially instructed to focus on strictly pedagogical reasoning for the course system proposal.  I agree that there was a modest financial advantage to the course system over credit, but it arises because on average students are taking fewer courses under the specific scenario we examined.  Some of us felt that reducing the courses per student per semester was akin to granting a degree for less work\footnote{Alternatively, we can think of this as indirectly raising tuition by collecting the same tuition for fewer courses.  These are statistical statements drawn from the scenario we examined, which focused on averaging over hundreds of students.}.  Once again, these are my honest reflections and I do not mean to suggest that my colleagues wanted to do the wrong thing.  We all had good intentions, and we tried really hard to understand the effects a complex issue would have on our students.
\\
\vspace{0.25cm}
Once we considered the financial argument, we tried to resolve a lingering impasse.  The proposal from AY 2020-2021 contained caps on the number of courses per semester.  The original plan barred a first-year student from taking five courses in either semester, and any student would have had to obtain special permission to take five courses.  Given my experience advising in 3-2 Engineering, Physics, and ICS, I knew that would cause painful setbacks to graduation.  I also recognized that my colleagues were trying to protect students from failing out of college.  We examined articles covering data from other institutions that suggests a strong load is actually correlated with student success.  We also noticed in our own data that students who take five courses tend to have higher grades, but that these students do not represent a majority of the student body.  Thus, two poles developed: no per-semester caps at all, or the original proposal.  I offered a compromise: during the first semester, a cap of four courses, followed by the ability to apply for and acquire the ability to take five courses in any semester given an acceptable GPA.  Academic probation would remove the ability.  The compromise allows us to measure the college GPA of a student in the first semester, which we know is lower thann high school GPA from my ESAC committee studies (Sec. xxx of my prior PEGP).  The compromise also respects the will of the students.  My idea satisfied both sides and was adopted into the final draft.

My sub-project, the survey.  Framing the issue of the course system, understanding it.  Mathematical analyses of the proposals: (a) financial implications (b) pedagogical implications (c) curricular implications (d) the compromise i offered that was accepted regarding maximum course loads

\subsection{The Whittier Scholars Program}

Another year, another advisee, acceptance to join the WSP board

\subsection{Future Proposals for Institutional Research}

Full utilization of the Tableau dashboards left by Gary Wisenand, growth of the ICS/Math major etc.

\end{document}
