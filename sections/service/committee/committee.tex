\documentclass[../../../main.tex]{subfiles}

\begin{document}
\label{sec:committee_service}

In 2017, my department had arranged my schedule such that I did not serve on a committee for the first year.  By Fall 2018, I had developed the idea that I could serve Whittier College through data analysis.  I was interested in the connection between the high school preparation of our students and their ability to pass introductory courses required for their major.  On the Enrollment and Student Affairs Committee (ESAC), in Fall 2018, I learned that this is a topic with which many administrators and intructors had been struggling.  I spent two years working on ESAC, and I watched as our committee carefully approached consensus while remaining respectful of the diverse perpectives that included athletics, student life, and instructors.  In the second year, we began discussions with Falone Serna, Vice President of Enrollment Management, to implement the policy result of the prior year.  On ESAC, I also learned about first year orientation, for which I volunteered in 2019 and 2020.
\\
\vspace{0.15cm}
In 2020-21, having served two years on ESAC, we decided it would be good for me to experience service with other types of committees.

\subsection{Educational Policy Committee}

My sub-project, the survey.  Framing the issue of the course system, understanding it.  Mathematical analyses of the proposals: (a) financial implications (b) pedagogical implications (c) curricular implications (d) the compromise i offered that was accepted regarding maximum course loads

\subsection{The Whittier Scholars Program}

Another year, another advisee, acceptance to join the WSP board

\subsection{Future Proposals for Institutional Research}

Full utilization of the Tableau dashboards left by Gary Wisenand, growth of the ICS/Math major etc.

\end{document}
