\documentclass[../../../main.tex]{subfiles}

\begin{document}
\label{sec:qsc}

In Spring 2022, I was recruited by Prof. Gil Gonzalez to serve in a group of colleagues whose goal is to create a Quantitative Success Center (QSC).  To say that Whittier College has struggled with student success in introductory math courses in recent years is an understatement.  I have been confronted in my years at Whittier with situations I did not know occurred in higher education.  A senior in algebra-based physics approached me mid-exam and asked ``what does solving for x mean?'' I witnessed a student fail College Algebra (MATH076) four times, \textit{and then} enroll in my Algebra-based Physics course.  I decided out of curiosity at the beginning of last semester to observe how MATH074 is taught, but I found every single chair empty and the instructor working on a laptop.  The instructor has since left Whittier College.
\\
\vspace{0.15cm}
My teaching practice is always informed by equity, and I incorporate PER when I teach physics and mathematics.  I participated in three workshops on inclusion in introductory STEM courses (see Sec. \ref{sec:dei} below).  I wrote extensively about how I address classroom equity in my prior PEGP (see Sec. 2.2 of my prior PEGP).  I hope that I have earned some credibility within the area of teaching physical science courses to a diverse group of students with varying preparation.  I have taught $\approx 500$ diverse STEM students here and regularly attend workshops on how to teach them better.  The majority of my students in physics are students of color and young women\footnote{For this semester: 60\% people of color and 56\% young women in PHYS135A, and 64\% people of color and 55\% young women in PHYS150.}.  When discussing DEI issues with me, my colleagues usually point out that the majority of instructors at Whittier College in Mathematics, Physics, Computer Science, and Chemistry are white men.  They also state that DFW rates in these subjects are higher than others (particularly in math).  Yet, my course evaluation data and learning outcomes show that my approach is generating student success in STEM at Whittier.
\\
\vspace{0.15cm}
%If I, a white man, am unable to ``address their needs,'' then a majority of my students should not succeed.  \textbf{My teaching data demonstrates the exact opposite reality.}  When lecturing me about DEI issues, some colleagues point out that the majority of STEM \textit{professors} in Mathematics, Physics, Computer Science, and Chemistry are white men, and that DFW rates in these subjects are higher than others (particularly in math).  So why is it not an open-and-shut case at Whittier, just like the situation at much larger institutions?  Anecdotally, my colleagues in my department also demonstrate success with diverse students.  The simplest explanation is that it has nothing to do with ``whiteness,'' but teaching style.

%So trust me when I share, in sincerity, how alarming and frustrating it is to hear the rhetoric some of our colleagues use to describe how they believe STEM courses are taught at Whittier College.  As I was editing my DEI internal grant (Sec. \ref{sec:dei}) with IDC colleagues, some wrote the following:
%
%\begin{quotation}
%We do not see how, for example, a system designed by white men addresses the needs of students of color and women who are marginalized in STEM.
%\end{quotation}
%
%I was shocked to read that.  White professors are \textit{incapable} of helping students from a different background?  I chose to respond with logic:
%
%\begin{quotation}
%Consider a proof by contradiction: introductory STEM courses at Whittier College are ``systems'' often created by white men, so by this logic no students of color or female students should succeed in such courses.  But many do, as evidenced by course evaluation data and in our learning outcomes. If they did not, a large fraction of introductory physics students would fail.
%\end{quotation}
%
%I have taught $\approx 500$ diverse STEM students here and regularly attend workshops on how to teach them better.  The majority of my students in physics are students of color and young women\footnote{For this semester: 60\% people of color and 56\% young women in PHYS135A, and 64\% people of color and 55\% young women in PHYS150.}.  If I, a white man, am unable to ``address their needs,'' then a majority of my students should not succeed.  \textbf{My teaching data demonstrates the exact opposite reality.}  When lecturing me about DEI issues, some colleagues point out that the majority of STEM \textit{professors} in Mathematics, Physics, Computer Science, and Chemistry are white men, and that DFW rates in these subjects are higher than others (particularly in math).  So why is it not an open-and-shut case at Whittier, just like the situation at much larger institutions?  Anecdotally, my colleagues in my department also demonstrate success with diverse students.  The simplest explanation is that it has nothing to do with ``whiteness,'' but teaching style.

In appears to me that introductory mathematics pedagogy at Whittier College relies more heavily on ``traditional lecture'' style, to use the terminology from Sec. \ref{sec:teaching_philosophy}.  This is not the case with introductory physics.  I note that traditional style might be more appropriate for advanced math courses as it is for advanced physics courses.  A QSC focused on pedagogical practices that generate success among a diverse set of students will be a positive influence for our introductory math instructors.  To that end, I argued for including qualities like administrative experience in the future QSC director.  Our group drafted with consensus a job description that called for the following characteristics: proven teaching experience in mathematics at the college level, knowledge of equitable teaching strategies, and administrative experience.  The latter is the most important.  The QSC director will be responsible for helping mathematics instructors become more successful.  These interactions will not be easy, but necessary.  I will contribute to the QSC by giving lectures about PER in mathematics instruction, because I have shown that MATH080 can be taught this way successfully.  I hope to share good news about QSC development in my next PEGP report.

\end{document}
