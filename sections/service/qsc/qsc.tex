\documentclass[../../../main.tex]{subfiles}

\begin{document}
\label{sec:qsc}

In Spring 2022, I was recruited by Prof. Gil Gonzalez to serve in a group of physical science and education professors and staff whose goal was to create a Quantitative Success Center (QSC).  To say that Whittier College has struggled with student success in introductory math courses in recent years is an understatement.  In my first years here, I was confronted with situations I did not know occurred in higher education.  A senior student approached me mid-exam and asked ``what does solving for x mean?'' I witnessed a student fail College Algebra (MATH076) four times, \textit{and then} enroll in my Algebra-based Physics course.  Even this semester, I decided to peak into a section of MATH074, only to find every single chair empty and the professor working on the laptop.
\\
\vspace{0.25cm}
Let me begin by stating unequivocally that my teaching practice is always informed by equity.  I incorporate PER when I teach physics and mathematics (for example, MATH080: Elementary Statistics).  I participated in three workshops on inclusion in introductory STEM courses (see Sec. \ref{sec:dei} below).  I wrote extensively about how I address DEI issues in my prior PEGP report (see Sec. 2.2 of my prior PEGP).  I hope that I have earned some credibility within the area of teaching physical science courses to students with a wide variety of preparation levels.  So trust me when I share, in sincerity, that there are few words to describe how astonishing and disheartening it is to hear some of our colleagues complain about STEM at Whittier.  In feedback I received about one of my projects, several colleagues shared that ``''

In a lecture on social psychology \cite{haidt}, Prof. Jonathan Haidt (NYU Stern School of Business) points out that many people jump to conclusions in social science analyses of STEM in higher education.  The moment 

\end{document}
