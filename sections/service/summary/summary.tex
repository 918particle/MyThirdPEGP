\documentclass[../../../main.tex]{subfiles}

\begin{document}
\label{sec:service_summary}

Because this report represents my application for tenure, I include a representative list of service projects I have completed at Whittier College over the years, and proposed future acts of service.  This list is for the benefit of those who have not encountered my work before, or who have not read my prior reports.
\\
\vspace{0.15cm}
\subsubsection{Past Projects}
\small
\begin{enumerate}
\item I involved my student, Cassady Smith in my research, and we earned a 2018 Keck summer research fellowship.  I mentored Cassady, and wrote a letter of recommendation that helped Cassady earn a research fellowship on LIGO (the Nobel prize-winning gravity-wave detection experiment).  Cassady is currently a graduate student at Yale Unviersity.
\item I involved my student, John Paul G\'{o}mez-Reed in my research, and we earned a 2018 Keck summer research fellowship, and later an Ondrasik-Groce Fellowship.  I mentored John Paul, and wrote for John Paul letters of recommendation for graduate schools.
\item I gave a lecture on our research at Whittier College at Science Night at Los Nietos Middle School in Fall 2018, and promoted Whittier College.
\item I joined the undergraduate organization CRU at Whittier College, acting as a mentor to young Christian men.  We created a Bible study group.  Some of the CRU members were also my students in my courses.  I was later awarded The Outstanding Organization Advisor Award, in the inaugural year of that award.
\item In 2018, I helped organize and execute a faculty search for a tenure-track position in the Department of Mathematics.  This included one-on-one interviews with candidates, and sifting through applications.
\item In 2018, Prof. Zorba invited me to join the Artemis Program.  Funded through the CEC, the Artemis Program recruits and trains promising female students from local high schools to apply to Whittier College.  We organize faculty-led research projects with the students, in the hope that they will join STEM fields at Whittier College in the future.  My first Artemis group joined me in Spring 2019.  I would later work with a second Artemis cohort in Spring 2020.  We learned to write computer code, control digital circuits, conduct experiments, and perform data analysis.
\item In my service to the ESAC committee in 2018-2019, I joined a sub-committee tasked with analyzing student admissions data.  I discovered a disparity in financial aid gap (put simply: grants and scholarships minus tuition) that correlated with student success.  The discovery led to reform in our admission of students.
\item In 2018-2019 I was advisor to Nicolas Clarizio, a double major in business and physics.  Together, we designed and built a fully functional 3D-printed quadrotor drone.  The final goal of the project is to add solar charging to the drone, so that it can aid in polar exploration and glaciology.
\item In Fall 2018, I joined Prof. Lagan for first-year orientation. I helped Seamus with ice-breaker activities witrh the new students, and with discussions about Whittier College as a liberal arts institution.
\item I have created two CON2 courses (one also is designated CUL3): INTD255 and INTD290.  The students reported that these courses were exciting and interesting.
\item I have taught INTD100 twice, including this current semester.
\item I have given two lectures to my colleagues at workshops organized through Wardman Libary on Open Educational Resources (OER).  A variety of professors attended these workshops to learn how we use OER to make our courses more equitable and inclusive.
\item I have learned to implement OER in all of my courses for purposes of equity.  Whenever possible, I use OER textbooks and other resources to minimize the cost to students.  This includes advanced courses.
\item Every Summer since I joined Whittier College, I have hosted and mentored Whittier undergraduates through fellowship programs such as the Keck Fellowship, Fletcher Jones Fellowship, and the Ondrasik-Groce Fellowship.  Occasionally, I have used my start-up grant to fund student workers to continue research projects started during the semesters.
\end{enumerate}

\subsubsection{Future Projects}
\small
\begin{enumerate}
\item The physical sciences wing of Whittier College maintains the \textbf{3-2 Engineering Program}.  This program was created by Prof. Lagan, in which students complete PHYS, COSC, and MATH courses at Whittier College for three years.  The students complete two years of engineering courses at (for example) USC.  They receive two Bachelor's Degrees: physics, computer science, chemistry, or math from Whittier College, and an engineering degree from the next institution.  Currently, Prof. Zorba is assumming responsibility for the program while I serve on WSP.  I have shared with my department that I can assume responsibility once my time on WSP concludes.
\item In January 2022, I was asked by Prof. John Bak to help draft an engineering proposal for the Motorola Solutions Foundation (MSF).  We were invited to submit a pre-proposal to MSF, and I reacted quickly to help raise money for the school.  I chose to draft a proposal based on my CEM and radar research (see Sec. \ref{sec:scholarship}).  Though we made it to the final round, we were not chosen as finalists.  However, this shows that I continue to participate in fundraising efforts for Whittier College, which is something you can expect from me going forward.  See Supplemental Material for details.
\end{enumerate}
\end{document}
