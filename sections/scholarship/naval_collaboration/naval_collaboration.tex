\documentclass[../../../main.tex]{subfiles}

\begin{document}
\label{sec:naval_collaboration}

According to an economic analysis by the Los Angeles County Economic Development Corporation in 2016

\section{Building Student Success after Whittier College}

When I began to interact with the Office of Naval Research, my contacts raised the possibility of a more formal partnership with Whittier College.
\\
\vspace{0.25cm}
To date, I have advised five physics and engineering students toward graduation (not counting my WSP student)\footnote{Students: Cassady Smith, John Paul G\'{o}mez-Reed, Nicolas Clarizio, Nicolas Bakken-French (WSP), Raymond Hartig, and Adam Wildanger.}  Three of them are attempting 
 
\section{Equipping Whittier College Laboratories}

My colleagues at NSWC have already provided our laboratory in SLC with equipment
\section{Financial Support}

From Tab. ... I can receive \$16.5k per summer as a Summer Faculty Research Fellow through ONR.  At the next level, Senior Fellows receive \$19.0k per summer.  To qualify for Senior Fellow, one must have been awarded tenure as an Associate Professor at an institution accredited by the U.S. Department of Education.  One also must have published one paper per year since receiving a doctoral degree.  If awarded tenure in academic year 2022-23, I would meet both requirements for the 2024 application round.  Regarding sabbatical, the ONR does have another program designed for professors to complete research projects while on sabbatical for 1 or 2 semesters.  This funding is important for my family, and we are proud to work hard for Whittier College and our students as they help serve ONR.  Regarding student financial support, I am hoping to coordinate that this year as a team effort between Whittier administrators and ONR personnel.

\end{document}
