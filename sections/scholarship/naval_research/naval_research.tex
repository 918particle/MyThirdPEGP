\documentclass[../../../main.tex]{subfiles}

\begin{document}

\section{Collaborations with the Office of Naval Research (ONR) and NSWC Corona Division}
\label{sec:naval_research}

Running in parallel with my UHE-$\nu$ physics research has been a fruitful engineering collaboration with the US Navy.  There is a long history of collaborative engineering research between the military and national laboratory system and the physical sciences in academia.  The United States maintains strategic security by collaborating with engineering researchers in academia, who in turn benefit via funding and collaborative opportunities that further research goals.  The research, in turn, benefits the people via technological breakthroughs.  One of the most famous examples is the Global Positioning System (GPS).  The connection to my UHE-$\nu$ research is that both the Navy and IceCube Gen2 require me to engage in RF engineering research.  This past summer, I have also begun to study the GPS system modernization.
\\
\vspace{0.25cm}
In Sec. 8 of my prior PEGP, I wrote about the opportunity our collaboration with the US Navy represents for our students.  The collaboration is funded by the Office of Naval Research (ONR) and takes place at a US Navy laboratory called NSWC Corona Division in Corona, CA.  The research has touched on radar and RF component design via computational electromagnetism (CEM), GPS modernization, RF field engineering training, and will in the future include reliability engineering and high-performance computing (HPC).  In Sec. \ref{sec:cem_paper}, I share how this collaboration has already produced award-winning research that has involved three different Whittier College undergraduates.  This research led to an invitation for me to speak at a national CEM conference at MIT\footnote{See supplemental material, and \url{https://news.mit.edu/2022/meepcon-comes-to-mit-0817}.}.  In Sec. \ref{sec:fellow}, I share how I have been helping NSWC Corona with workforce development using my teaching skills, and how the ONR Summer Faculty Research Program (SFRP) will boost our research in Summer 2024.  In Sec. \ref{sec:epa}, I share our plans to form a lasting partnership called an Educational Partnership Agreement with NSWC Corona, and the benefit to future engineering students at Whittier College.

\subsection{Radio-Frequency (RF) Engineering, with Applications to Radar and IceCube Gen2}
\label{sec:cem_paper}

I received my first Summer Faculty Research Fellowship in the Summer of 2020.  Miraculously, the ONR decided that the fellows could conduct research remotely.  After a consultation with Gary Yeakley, Dr. Christopher Clark, and Jeffery Benson of NSWC Corona, we concluded I could best serve our mutual interests by designing RF phased arrays.  RF phased arrays are groups of individual RF antennas that work together to generate a radar beam that can be steered with no moving parts.  This technology is highly useful in field environments that require AESAs: active electronically scannable antennas.  IceCube Gen2 and RNO-G will use them as receivers to detect UHE-$\nu$ (see Sec. \ref{sec:neutrino}).

\subsubsection{Publication: Broadband RF Phased Array Design with MEEP}
this was all me baby
\subsubsection{Undergraduate Research and the Fletcher-Jones Fellowship}
Include Adan Wildanger and 3-2 program
\subsubsection{Undergraduate Research with Physics Research (PHYS396)}
Include Dane Goodman and Andrew Householder
\subsubsection{Invited Lecture: First Annual MeepCon at MIT (Summer 2022)}
first invited lecture 45 min killed it
\subsection{Research Grants: ONR Summer Faculty Research Fellowships (SFRP)}
\label{sec:fellow}
t
\subsubsection{Workforce Development: Active Learning in the Armed Services}
course 1 and 2
\subsubsection{Grants Received: SFRP Status}
money, equipment, and include mention the MSF as well
\subsection{Future Plans: Educational Partnership Agreement (EPA) with NSWC Corona Division}
\label{sec:epa}
mention equipment, high-performance computing, reliability engineering, internships
\end{document}
