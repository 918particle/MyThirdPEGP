\documentclass[../../../main.tex]{subfiles}

\begin{document}

\section{Ultra-High Energy Neutrino Research with IceCube Gen2}
\label{sec:neutrino}

\textit{Cosmic rays} are high-energy protons, electrons, and nuclei propagating through space near the speed of light.  They carry information from other regions in the galaxy, and in some case, other galaxies.  Since the discovery of extremely energetic cosmic rays more than a half century ago, the quest to uncover their sources continues.  Despite progress in experimental capabilities and theoretical insight, we do not yet know the acceleration mechanism for those particles with energies that have been measured in excess of $10^{20}$ electron-volts \cite{10.1088/1742-6596/1766/1/012002}.  Being electrically charged, cosmic rays paths are curved by electromagnetic fields in space.  By the time the cosmic ray arrives at Earth, the arrival direction no longer points back to the origin.  In addition, interactions with cosmic microwave background (CMB) photons can block ultra-high energy cosmic rays from reaching Earth \cite{PhysRevLett.16.748} \cite{1966JETPL...4...78Z}.
\\
\vspace{0.25cm}
Neutrino astronomy offers a powerful tool to discover the physics associated with cosmic ray acceleration, which is not accessible with other \textit{messengers}: cosmic-rays, gamma-rays, and  optical photons. Charged cosmic rays which interact with gas, dust, or radiation near an accelerating object produce gamma-rays and high-energy neutrinos.  These neutrinos are called \textit{astrophysical} neutrinos. Whereas gamma-rays can be absorbed in dense environments, astrophysical neutrinos can escape and travel unimpeded to a detector (\cite{Astro2020_1} and references therein). Neutrinos travel at the speed of light in straight lines, undeflected by electromagnetic fields.  This allows for identification of sources, as well as the potential for finding sources that emit both neutrinos and gravitational waves \cite{10.3847/2041-8213/ab9d24}.
\\
\vspace{0.25cm}
The most energetic cosmic rays that do escape their source can interact with the CMB en route to the Earth, generating \textit{cosmogenic} neutrinos with a characteristic energy distribution peaking at $10^{18}$ electron-volts \cite{10.1007/bf00645585} \cite{BERESINSKY1969423}.  Astrophysical and cosmogenic neutrinos offer a window into regions of the Universe far beyond what is possible with other messengers.  A flux of neutrinos originating outside the solar system with energies between $10^{13}$ and $10^{15}$ electron-volts has been measured by the IceCube collaboration \cite{PhysRevLett.111.021103}. Previous analyses have shown that the discovery of ultra-high energy neutrinos (UHE-$\nu$, energy greater than $10^{15}$ electron-volts) will require a detector with a larger volume because the flux decreases with energy \cite{PhysRevD.98.062003}. UHE-$\nu$ are the ones that could potentially explain the origin of cosmic rays.  UHE-$\nu$ also provide the chance to study quantum mechanical interactions never probed before \cite{Astro2020_1} \cite{Astro2020_2}.
\\
\vspace{0.25cm}
Utilizing the \textit{Askaryan effect}, in which UHE-$\nu$ creates a radio-frequency pulse, greatly expands the effective volume of UHE-$\nu$ detector designs.  This effect is important for UHE-$\nu$ detection, because radio pulses travel more than 1 kilometer in Antarctic and Greenlandic ice \cite{10.3189/2015jog14j214} \cite{10.3189/2015jog15j057} \cite{10.1002/2015rs005849} \cite{10.1016/j.astropartphys.2011.11.010}.  Stations can be placed 1 km apart, and the volume of the overall array of stations is large enough to capture UHE-$\nu$.  The large volume is necessary for two reasons.  First, cosmic rays and neutrinos at these energies are rare enough that a $10 \times 10$ km$^2$ target is necessary.  Second, neutrinos do not always interact in dense matter quantum mechanically, unlike protons or heavy ions.  To ensure that we record the UHE-$\nu$ that do interact, we must construct a large detector.

\subsection{Publication: A Fully Analytic Model of the Askaryan Effect}
t
\subsubsection{Undergraduate Research and the Fletcher-Jones Fellowship}
t
\subsubsection{Undergraduate Research and the Ondrasik-Groce Fellowship}
t
\subsubsection{Four Applications of the Askaryan Model and Future Plans}
t
\subsection{Publication: Greenlandic Ice and the Radio Neutrino Observatory, Greenland (RNO-G)}
t
\subsection{Grant Applications Submitted, and Future Plans}
t
\subsubsection{Machine Learning in High-Energy Astrophysics}
t
\subsubsection{Supercomputers and the Wisconsin IceCube Particle Astrophysics Center (WIPAC)}
t
\end{document}
