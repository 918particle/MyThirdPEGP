\documentclass[../../../main.tex]{subfiles}

\begin{document}

\section{Ultra-High Energy Neutrino Research with IceCube Gen2}
\label{sec:neutrino}

\textit{Cosmic rays} are high-energy protons, electrons, and nuclei propagating through space near the speed of light.  They carry information from other regions in the galaxy, and in some case, other galaxies.  Since the discovery of extremely energetic cosmic rays more than a half century ago, the quest to uncover their sources continues.  Despite progress in experimental capabilities and theoretical insight, we do not yet know the acceleration mechanism for those particles with energies that have been measured in excess of $10^{20}$ electron-volts \cite{10.1088/1742-6596/1766/1/012002}.  Being electrically charged, cosmic rays paths are curved by electromagnetic fields in space.  By the time the cosmic ray arrives at Earth, the arrival direction no longer points back to the origin.  In addition, interactions with cosmic microwave background (CMB) photons can block ultra-high energy cosmic rays from reaching Earth \cite{PhysRevLett.16.748} \cite{1966JETPL...4...78Z}.
\\
\vspace{0.15cm}
Neutrino astronomy offers a powerful tool to discover the physics associated with cosmic ray acceleration, which is not accessible with other \textit{messengers}: cosmic-rays, gamma-rays, and  optical photons. Charged cosmic rays which interact with gas, dust, or radiation near an accelerating object produce gamma-rays and high-energy neutrinos.  These neutrinos are called \textit{astrophysical} neutrinos. Whereas gamma-rays can be absorbed in dense environments, astrophysical neutrinos can escape and travel unimpeded to a detector (\cite{Astro2020_1} and references therein). Neutrinos travel at the speed of light in straight lines, undeflected by electromagnetic fields.  This allows for identification of sources, as well as the potential for finding sources that emit both neutrinos and gravitational waves \cite{10.3847/2041-8213/ab9d24}.
\\
\vspace{0.15cm}
The most energetic cosmic rays that do escape their source can interact with the CMB en route to the Earth, generating \textit{cosmogenic} neutrinos with a characteristic energy distribution peaking at $10^{18}$ electron-volts \cite{10.1007/bf00645585} \cite{BERESINSKY1969423}.  Astrophysical and cosmogenic neutrinos offer a window into regions of the Universe far beyond what is possible with other messengers.  A flux of neutrinos originating outside the solar system with energies between $10^{13}$ and $10^{15}$ electron-volts has been measured by the IceCube collaboration \cite{PhysRevLett.111.021103}. Previous analyses have shown that the discovery of ultra-high energy neutrinos (UHE-$\nu$, energy greater than $10^{15}$ electron-volts) will require a detector with a larger volume because the flux decreases with energy \cite{PhysRevD.98.062003}. UHE-$\nu$ are the ones that could potentially explain the origin of cosmic rays.  UHE-$\nu$ also provide the chance to study quantum mechanical interactions never probed before \cite{Astro2020_1} \cite{Astro2020_2}.
\\
\vspace{0.15cm}
Utilizing the \textit{Askaryan effect}, in which UHE-$\nu$ creates a radio-frequency pulse, greatly expands the effective volume of UHE-$\nu$ detector designs.  This effect is important for UHE-$\nu$ detection, because radio pulses travel more than 1 kilometer in Antarctic and Greenlandic ice \cite{10.3189/2015jog14j214,10.3189/2015jog15j057,10.1002/2015rs005849,10.1016/j.astropartphys.2011.11.010}.  Stations can be placed 1 km apart, and the volume of the overall array of stations is large enough to capture UHE-$\nu$.  The large volume is necessary for two reasons.  First, cosmic rays and neutrinos at these energies are rare enough that a $10 \times 10$ km$^2$ target is necessary.  Second, neutrinos do not always interact in dense matter quantum mechanically, unlike protons or heavy ions.  To ensure that we record the UHE-$\nu$ that do interact, we must construct a large detector.

\subsection{Publication: A Fully Analytic Model of the Askaryan Effect}

Given the importance of Askaryan radiation for UHE-$\nu$ science \cite{Askaryan:1962hbi,1962JPSJS..17C.257A,ask1,ask2,PhysRevLett.86.2802,PhysRevLett.99.171101,PhysRevD.101.083005}, I embarked in 2020 on a mission to be the first to publish a fully analytic model of Askaryan radiation.  A fully analytic model differs from semi-analytic and fully simulated models because the latter two cases involve computations that cannot be written mathematically.  I describe four advantages of fully analytic models below.  Making this project fully analytic represented new exploration for me in mathematical physics.  My normal role in my department is to be the engineer and experimental physicist.  During quarantine, however, I could not access my lab.  In truth, the idea had been on my mind since 2017, when I first published on the subject as a CCAPP Fellow at The Ohio State University \cite{10.1016/j.astropartphys.2017.03.008}.  I am pleased to share that my final results have been published in \textit{Physical Review D} \cite{PhysRevD.105.123019}.  Physical Review D (PRD) is one of the most impactful journals in elementary particle physics, field theory, gravitation, and cosmology\footnote{Physicists use a metric called an impact factor to gauge the strength of a physics journal.  The impact factor provides a functional approximation of the mean citation rate per citable work.  An impact factor of 1.0 means that, on average, the articles published 1-2 years ago have been cited one time.  PRD had an impact factor of 5.4 in 2021, placing it in the top 5 APS journals, and the top position for high-energy particle physics journals.  For comparison, \textit{Astroparticle Physics Journal}, another journal frequently used by my colleagues, has an impact factor of 2.7.}.  In Secs. 3.5 of prior PEGP, I shared how this work was submitted to PRD and favorably reviewed (published on the physics arXiv: \url{https://arxiv.org/}).  After over a year in review, the editor was told to add another reviewer.  I ultimately succeeded after rewriting the manuscript in 2021-2022.
\\
\vspace{0.15cm}
The publication has marked a milestone in my career, for three reasons.  First, this work is the first I have published in PRD, which is considered the top journal in my field.  Second, this is the first time I have helped a Whittier College undergraduate publish in a peer-reviewed physics journal.  Raymond Hartig, a double-major in physics and mathematics, has been an enormous help in verifying my calculations and incorporating our equations into IceCube Gen2 detector simulations.  I have published other papers in my time at Whittier College, and I have always involved undergraduates in my research.  But my work with Raymond represents the first time we have published undergraduate research in a \textit{physics} journal, as opposed to an engineering journal.  Third, this work represents the first time I have published a paper on theoretical or mathematical physics, as opposed to experimental or applied physics.  As I described in Sec. 3.5 of my prior PEGP, this work represents my third published paper in a peer-reviewed journal in my time as a Whittier College physics professor in satisfaction of departmental guidelines for tenure.

\subsubsection{Undergraduate Research and the Fletcher-Jones Fellowship}

My first child was born in May 2020, just as the first wave of the pandemic began exponential growth.  Suddenly, all laboratory research halted, and I needed a project I could complete at night after the baby was asleep.  Thankfully Raymond Hartig, a student I had been preparing for some mathematical physics project during the Spring 2020 semester, won a Fletcher-Jones Fellowship from Whittier College.  Using pen and paper, and communicating over Discord, we sought to expand my work from Ohio State into a fully-analytic time-domain model \cite{10.1016/j.astropartphys.2017.03.008}.  Raymond explored one path while I explored another, until we found the right path.  To say he worked hard is an understatement.  We arrived at the central results, and began a manuscript during January term of 2021.  The results are a formula for the electromagnetic field versus time, emitted by a neutrino hitting ice at the speed of light.  This field will be a critical signal for IceCube Gen2.  Raymond applied for the Ondrasik-Groce Fellowship in the Summer of 2021.  After we learned we were unsuccessful, I paid him from my research startup grant so he could continue helping with the paper.

\subsubsection{Undergraduate Research and the Ondrasik-Groce Fellowship}

This Summer, Raymond and I won the Ondrasik-Groce Fellowship to advance the Askaryan model research.  Our idea is to use the model to measure the energy of the UHE-$\nu$ in IceCube Gen2.  Imagine detecting a radio pulse deep within Antarctic ice, potentially caused by a UHE-$\nu$.  The actual data collected is a waveform captured by RF channels built from antennas, amplifiers, and filters.  Using the waveform to deduce UHE-$\nu$ properties requires us to perform a simulation using some Askaryan model and knowledge of the RF channels.  Our first task this Summer was to incorporate our newly published model into NuRadioMC, the flagship simulation package for IceCube Gen2.  I shared in Sec. 3.5 of my prior PEGP that I created one of the pillars of NuRadioMC by solving the ray-tracing problem of RF pulses in Antarctic ice \cite{10.1140/epjc/s10052-020-7612-8} \cite{horizPaper}.  Raymond has done a good job of incorporating our model into NuRadioMC, and checking that the results match other (non-analytic) models.  Now that this job is done, we are working out how to measure the UHE-$\nu$ energy using simulated neutrinos in IceCube Gen2.  The analytic nature of our model makes this possible because one of the parameters in the model is derived from the UHE-$\nu$ primary energy.  One the protocol is established, I believe this will represent our second PRD publication.  Raymond plans to apply to graduate school to become a physicist, and these works have helped him to form a wonderful portfolio for his application.

\subsubsection{Three Applications of the Askaryan Model and Future Plans}
\label{sec:ask_app}

There are at least three main advantages of analytic time-domain models.  The first pertains to our Ondrasik-Groce Fellowship project: UHE-$\nu$ properties like total energy may be derived directly from the waveforms if they are matched to analytic ones.  Second, evaluating a fully analytic model provides a software speed advantage compared to other models. Third, analytic
models may be embedded in detector firmware to form a filter that blocks noise and enhances the UHE-$\nu$ detection probability. This feature will greatly enhance IceCube Gen2 prospects because of the rare nature of UHE-$\nu$ signals.  My vision for the future of this work involves three tracks.
\\
\vspace{0.15cm}
The first track involves UHE-$\nu$ template analysis. NuRadioMC, is broken into four pillars. Currently, UHE-$\nu$ are simulated first as events (NuRadioMC pillar (1)) and the RF emissions (Askaryan signals) are generated next (NuRadioMC pillar (2)).  Our ability to match simulated waveforms to potential UHE-$\nu$ waveforms from the detector is limited because we cannot scan through properties of the simulated cascades of particles created by the UHE-$\nu$, only the UHE-$\nu$ properties. For example, two UHE-$\nu$ with the same energy could generate different cascades with different shapes of electric charge. The cascade shape affects the waveform shape and therefore the interpretation of future IceCube Gen2 data. Conversely, if the effect of the cascade shape is understood, one can measure the UHE-$\nu$ energy using templates \cite{PhysRevD.105.123019}.
\\
\vspace{0.15cm}
The second track involves embedding the model itself in detector firmware. The detector cannot distinguish small signals from noise. Noise and signal data that trigger detectors and are both saved. We try to isolate UHE-$\nu$ signals in large data sets comprised mostly of radio noise once the data has been transmitted to the USA. All data has to be shipped with limited bandwidth, and we cannot ship it continuously. Embedding the model on the detector would allow the detector itself to locate and flag priority events that appear to match UHE-$\nu$. The physics community expects IceCube Gen2 to provide this type of alert system so that other detectors could search for UHE-$\nu$ sources identified by IceCube Gen2. Thus our approach makes possible an approach that is fast enough to solve the problem.
\\
\vspace{0.15cm}
The third track involves the connection between computational electromagnetism (CEM) and our Askaryan model. In NuRadioMC the simulated signal is created by code in pillars (1)-(4) sequentially. That is, the Askaryan model is mixed with detector response after ray-tracing. In reality, the radiation flow does not have to follow strictly ray-tracing, but the 3D index of refraction of the ice. It is a wave that generally follows ray-tracing, but also reflects from internal ice layers, propagates horizontally, and can change shape. All the effects not captured by the smooth index of refraction function, $n$, will affect the signal.  Fully-analytic Askaryan models have a unique advantage: analytic equations can be implemented as CEM sources, and CEM codes can account for all those effects. The analytic model is in a unique position to provide advanced insight when combined with CEM.

\subsection{Publication: Greenlandic Ice and the Radio Neutrino Observatory, Greenland (RNO-G)}

During my time in graduate school at UC Irvine, and as a post-doctoral fellow at the University of Kansas, I conducted expeditions to Antarctica to gather data and deploy UHE-$\nu$ detector prototypes.  We collected radio sounding data to measure the RF transparency of the ice, to ascertain whether the ice could be used as a detection medium for IceCube Gen2 and its associated prototypes.  A radio sounding experiment is when an RF pulse is transmitted through ice and the echo is received.  The echo time is related to the ice thickness, and the strength of the returning RF pulse is related to the ice transparency.  I published a paper in the Journal of Glaciology on the RF transparency of the Ross Ice Shelf \cite{10.3189/2015jog14j214}.  In 2017-2018, my colleagues and I wrote a second paper revealing that classically forbidden RF ray-tracing behavior can occur in the same ice \cite{Barwick:2018497}.  The Radio Neutrino Observatory, Greenland (RNO-G) project was formed in the hope that Greenlandic ice would also have sufficient RF transparency for a UHE-$\nu$ detector.  Initial measurements from a 2014 expedition to Summit Station, Greenland, were conclusive enough to begin RNO-G development \cite{10.3189/2015jog15j057}.  The 2014 expedition, however, was limited to just a single RF frequency, and the raw data was not easily distinguished from RF background noise.
\\
\vspace{0.15cm}
Thus, a new mission to Greenland was undertaken to collect data at a wider range of frequencies with larger signal strength.  I am a member of the RNO-G collaboration, and I have helped RNO-G to submit grant proposals for hardware development.  I was asked to be an internal reviewer for the peer-reviewed article once the expeditioners returned with the new RF transparency data.  An internal reviewer in physics is someone who serves a scientific collaboration through critiquing and revision of journal articles before they are submitted for formal peer review.  Given the size of physics collaborations in my field (sometimes we have $\approx 100$ scientists in one collaboration), internal reviewers play a vital role in maintaining the flow of polished, professional results.  I helped to fix errors and analysis issues in the manuscript.  I am happy to share that we have published the results in the Journal of Glaciology this Summer \cite{aguilar_2022}.  The results indicate that Greenlandic ice is almost as transparent as the ice at the South Pole, and suitable for RNO-G to move forward.  This is the second time I have reviewed an article as a Whittier College professor, but the first time I've been trusted with the responsibility as an internal reviewer.

\subsection{Grant Applications Submitted, and Future Plans}

In my scholarship related to IceCube Gen2, we have been active in the area of grant-writing for computational physics and artificial intelligence (AI).  I have two ongoing efforts to share with you.  If either of the projects below is successful, it will empower our Askaryan field research and enable new endeavors with IceCube Gen2 simulations and data analysis.

\subsubsection{Supercomputers and the Wisconsin IceCube Particle Astrophysics Center (WIPAC)}

In the Spring and Summer of 2022, members of the IceCube Collaboration submitted a grant proposal that included Whittier College\footnote{See supplemental material for details.}.  We worked with the Wisconsin IceCube Particle Astrophysics Center (WIPAC: \url{https://wipac.wisc.edu/}) to develop a proposal that will advance high-performance computing (HPC) to serve our field and many other forms of science.  The proposal is entitled ``MRI: Acquisition of the Heterogeneous Accelerator Lab (HAL) at University of Wisconsin-Madison.''  MRI stands for Major Research Instrumentation, and the MRI program through the NSF is used to build resources and tools for science at universities and labs.  If successful, this grant will provide resources for the construction of HPC resources augmented by special systems known as computational accelerators.  Thus, the proposed system is known as HAL, or Heterogenous Accelerator Lab.
\\
\vspace{0.15cm}
Accelerators are specialized instruments: different scientific problems can have different optimal hardware deployment.  Having a diverse array of computational hardware within an HPC system enables exploration of artificial intelligence (AI), simulation, and data processing techniques across a wide variety of scientific fields.  To maximize the utility of a new HPC system in such a quickly evolving technological landscape, HAL will consist of heterogeneous hardware and co-processors designed to serve all of these fields. HAL personnel will focus on the deployment, management, and operation of a dynamic and heterogeneous pool of accelerator capacity.  In addition to the diverse computational capability, HAL will broaden participation by diverse groups of researchers at the undergraduate level if the NSF approves it.
\\
\vspace{0.15cm}
The HAL team has made a committment to the empowerment of Whittier College students.  If funded, access to HAL will be provided to students and researchers at Whittier College as if they were UW-Madison researchers.  By sharing its resources via the
Open Science Pool (OSP) compute federation, HAL will enable Whittier College students to take advantage of new computing power in their projects.  For example, Whittier College students involved in machine learning and AI projects in the ICS/Math and 3-2 Engineering/Computer Science areas could have access to systems capable of completing projects we cannot currently complete with our in-house resources.  Further, our students would benefit by learning to connect to and interact with the Open Science Grid.

\subsubsection{Machine Learning in High-Energy Astrophysics}

During the Spring of 2022, my colleagues at UC Irvine, included Whittier College in an NSF grant application geared towards AI applications to high-energy astrophysics\footnote{See supplemental materials for project summary.  I chose not to include all 28 pages of the grant proposal, but just enough to show the main goals of the project.}.  The project is entitled ``NRT-WoU: Team Science and Science Communication for AI and Multimessenger Astrophysics.''  The project is an effort to build interdisciplinary cooperation between scientists who care about advancing AI research, and scientists interested in tackling the diverse digital data streams that flow from multi-messenger astrophysics experiments.  The term multi-messenger astrophysics means using all available astrophysical messengers to learn about the Universe: gamma-rays, optical photons, cosmic rays, neutrinos, and gravitational waves.
\\
\vspace{0.15cm}
The project has three goals.  The first goal is to produce transformational astrophysics research and innovative new AI algorithms.  Progress in AI is often driven by people attempting to solve a problem in physics that has practical applications in other sectors.  The first goal is tied to a proposal theme stating that the exchange should be more of an interdisciplinary collaboration.  Researchers in AI can contribute and collaborate with astrophysicists in the same way astrophysicists drive progress in AI.  The second goal is to increase the training of a diverse group of students with transferable science communication and collaboration skills.  To that end, the grant proposal discusses the active training for students in the areas of ``team science'' and science communication.  The third goal is to build new institutional support for interdisciplinary research beyond 5 years after grant acceptance.  One example is the creation of a professional MS in Data Science, run jointly by the UCI Schools of Physical Sciences and Information and Computer Sciences.  This effort shares traits with Data Science programming begun within our Lux Program.
\\
\vspace{0.15cm}
Whittier College would participate in the UCI NRT proposal via the neutrino science.  As founding members of the ARIANNA collaboration \cite{Barwick:2014pca,barwick2016radio,10.1109/tns.2015.2468182,4_5} and IceCube members, my UCI colleagues and I have been working toward the implementation of my analytic Askaryan model on-board Antarctic stations as described in section \ref{sec:ask_app}.  This task is an AI-driven one, with two stages.  First, we will implement my analytic equations as a digital model in detector firmware.  Second, we will train the firmware with AI to use the model to reconstruct UHE-$\nu$ properties from detector data.  If a reconstruction of UHE-$\nu$ properties is not made in the firmware, we can reject RF background noise in real time.  These tasks will improve the sensitivity of IceCube Gen2 to UHE-$\nu$ events because the detectors themselves will automatically reject noise in favor of signal.  Currently, this process limits the size of the detector array, because humans have to download the data and reject noise offline.  In Summer 2018, I began this project with an Ondrasik-Groce Fellowship student named John Paul G\'{o}mez-Reed using ARIANNA firmware.  With the analytic Askaryan model now published, this project becomes much stronger.

\end{document}
